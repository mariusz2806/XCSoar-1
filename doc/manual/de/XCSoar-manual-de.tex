\documentclass[german,a4paper,10pt]{refrep}
\usepackage[utf8]{inputenc}
\usepackage[german]{babel}
\DeclareUnicodeCharacter{00B0}{$^{\circ}$}
\usepackage[T1]{fontenc}
\usepackage[right]{eurosym}%EuroSymbol 
\usepackage{color}
\usepackage{times}
\usepackage{multicol}
\usepackage{booktabs}
\usepackage{longtable}
\usepackage{rotating}
\usepackage{todonotes}%		Gestattet Kommentare für Sachen, welche noch zu erldigen sind  .
\usepackage{paralist}%		etwas geringere Abstände in der item und enumerate...}
\usepackage{enumitem}%      Listen selber gestalten ... 
\usepackage{url}
\usepackage{hyperref}
\makeindex
\makeatletter
\title{Benutzerhandbuch}%	Titel der ersten Seite 
\newcommand{\mycontent}[0]{%
\begin{tabular}{l}\hspace*{-4cm}{\em XCSoar-Handbuch  } \vspace*{2pt}\end{tabular}}% Kopfzeile
%
\usepackage{color}
\usepackage{times}
\usepackage{booktabs}
\usepackage{longtable}
\usepackage{rotating}
\usepackage{todonotes}
\usepackage{multirow}
\usepackage{makeidx}\makeindex
\makeatletter
\usepackage{fancyhdr}
\pagestyle{fancy}
\maxipagerulefalse
%
% Include XCSoar header and footer settings and used buttons 
\input{xcsoar-headers.sty}
\input{buttons.sty}
%
\widowpenalty=1000
\clubpenalty=1000
%
%  XCSoar - Website Einf�gen
\newcommand{\xcsoarwebsite}[0]{\url{www.xcsoar.org}}
%
% Define command to insert tip image
\newcommand{\tip}[0]{\marginlabel{\parbox{1.1cm}{\includegraphics[width=0.7cm]{figures/reminder.pdf}}}}

% Define command to insert gesture image
%\newcommand{\gesture}[1]{\marginlabel{{\it#1{\phantom{aa}}}\parbox{1.1cm}{\hspace{-3mm}\includegraphics[width=0.8cm]{figures/gesture.png}}}}
\newcommand{\gesture}[1]{\marginlabel{{\it#1{\phantom{aa}}}\parbox{1.1cm}{\hspace{-3mm}\includegraphics[width=0.8cm]{figures/gesture.pdf}}}}
%
% Define command to insert warning image
\newcommand{\warning}[0]{\marginlabel{\parbox{1.3cm}{\includegraphics[width=0.9cm]{figures/warning.pdf}}}}
%
% Define command to insert Achtung image
\newcommand{\achtung}[0]{\marginlabel{\parbox{1.3cm}{\includegraphics[width=2.5em]{figures/warning.pdf}}}}
%
% Define command to insert a flash image
\newcommand{\blitz}[0]{\marginlabel{\parbox{1.3cm}{\includegraphics[height=2.0em]{figures/reminder.pdf}}}}
%
% Define command to insert a stop
\newcommand{\halt}[0]{\marginlabel{\parbox{1.3cm}{\includegraphics[height=2.0em]{figures/warning.pdf}}}}
%
% Define command to reference a configuration item
\newcommand{\config}[1]{\marginlabel{\ref{conf:#1}\parbox{1.3cm}{\includegraphics[width=0.8cm]{figures/config.pdf}}}}
%
% Potentially overdue ``InfoBox'' style macro
\newcommand{\InfoBox}[0]{{InfoBox}}
%
% Define command to put a menu label on the margin
\newcommand{\menulabel}[1]{\marginpar{\parbox{5.0cm}{\raggedright #1}}}
%
% Define command to draw a sketch on the margin
\newcommand{\sketch}[1]{\marginpar{\parbox{3.750cm}{\includegraphics[angle=0,width=0.9\linewidth,keepaspectratio='true']{#1}}}}
%
% Define command to draw a small sketch on the margin
\newcommand{\smallsketch}[1]{\marginpar{\includegraphics[angle=0,keepaspectratio='true']{#1}}}
%
% Enumerated todo's for the todonotes package
\newcounter{todocounter}
\newcommand{\todonum}[2][]{\stepcounter{todocounter}\todo[#1]{\thetodocounter: #2}}
%
% dies nette Makro bringt mir die aktuelle Version der bearbeiteten XCSoar-Verrion aufs Papier 
\newcommand{\version}{\begingroup\catcode`\_=\active\input{VERSION.txt}\endgroup}%
%
%Anf�hrungszeichen per Tastatur einfach so eingeben -> "
%\shorthandoff{"}%

\input{title-xcsoar.sty}
%
\newcommand\xc{\textsf{XCSoar }}%  ein paar Abkürzungen gegen die nervige Handarbeit ..  
\newcommand\fl{\textsf{Flarm }}
\newcommand\al{\textsf{Altair }}
\newcommand\trdvvs{Vega}
\begin{document}
\sloppy% ja ich weiß, soll man nicht mehr machen .. dennoch ..
\shorthandoff{"}%
\maketitle
\begingroup\setlength{\parskip}{0.1\baselineskip}\tableofcontents\endgroup
%%%%%%%%%%%%%%%%%%%%%%
\chapter*{Vorwort}
\section*{Warnungen und Vorsichtshinweise}
%
\warning
\textsc{Es liegt in der ausschließlichen Verantwortung des Benutzers, diese Software vernünftig
und umsichtig einzusetzen. Diese Software ist gedacht als Unterstützung zur Navigation
und darf \emph{auf keinen Fall} als Ersatz für professionelle, dedizierte Navigationsgeräte benutzt
werden. \emph{Auf gar keinen Fall} darf diese Software zur Bestimmung von Distanzen, Koordinaten
oder aber Richtungen benutzt werden. Diese Software darf \emph{nicht} als Antigeländekollisions- Software benutzt werden.
Diese Software darf \emph{auch nicht} als ein Antikollisionssystem benutzt werden.}
%
\section*{Offizielle Hinweise}
\subsection*{Software Lizenz Einverständnis}
%
Diese Software wird gemäß der GNU General Public License Version~2 veröffentlicht. Der ausführliche,
komplette Text des Einverständnisses hierzu und Anmerkungen zur Garantie befindet sich
unter~\ref{cha:gnu-general-public}.
%
\subsection*{Beschränkte Haftung}
In keinem Fall übernimmt XCSoar oder dessen Vorsitzende, Teilhaber, Programmierer, Angestellte  oder aber sonstige Mitwirkende  oder aber  Tochterunternehmungen irgendeine Haftung für den Gebrauch oder aber Mißbrauch und auch zufälliges Fehlverhalten dieses Programmes, egal zu welchem Zweck es eingesetzt wird.
%
\subsection*{Erklärung}
Dies Programm, alle dazugehörigen Dateien, Daten und Material wird verteilt "so,wie es ist" und gewährt keinerlei Garantie auf irgendeine Art und Weise -  weder erwähnt noch unerwähnt. Dies Programm wird komplett und \emph{ausschließlich auf ureigenes Risiko des Benutzers} installiert, ausgeführt und benutzt. Obwohl mit größtmöglicher Sorgfalt und Achtung während der Entwicklung dieses Produktes gearbeitet wurde, kann nicht ausgeschlossen werden, daß das Programm 100\% fehlerfrei ist. Auf gar keinen Fall können daher Ansprüche bzgl. Korrektheit, Fehlerfreiheit, Zuverlässigkeit für irgendeinen Teil dieser Software gestellt werden. Das XCSoar Projekt-Team und die Verteiler sind in keiner Weise verantwortlich für Fehler in diesem Programm oder für Fehlverhalten, Datenverlust oder aber körperliche Verletzungen die aus der Verteilung, Benutzung, Installation oder Anwendung dieser Software herrühren könnten.


%%%%%%%%%%%%%%%%%%%%%%
\input{ch01_introduction.tex}
%%%%%%%%%%%%%%%%%%%%%%
\input{installation.tex}
%%%%%%%%%%%%%%%%%%%%%%
\chapter{Benutzeroberfläche Interface}\label{cha:interface}

Dies Kapitel beschreibt das grundlegende  Konzept der Benutzeroberfläche, wie sie von \textsf{XCSoar} benutzt wird und ist als erste Übersicht zu verstehen. Mehr Details werden in den folgenden Kapiteln erörtert.

\begin{center}
\includegraphics[angle=0,width=\linewidth,keepaspectratio='true']{figures/plain.png}
\end{center}

Die  \textsf{XCSoar} Anzeige besteht aus mehreren Teilen:
\begin{description}
\item[Kartenanzeige] Die größte Fläche des Bildschirmes wird vom GPS-moving map beansprucht.
Wichtige Symbole, welche Informationen des Segelflugrechners darstellen, werden auf dieser Karte
eingeblendet (z.B. die Über/Unter Gleitpfad-Pfeile am linken Rand oder das Thermik-Höhenband).
Einige Icons und Textinformationen  erscheinen  (bei Bedarf) am unteren Rande des Displays,
welche über den Zustand des Rechners (verbunden mit GPS, Kurbelmodus, Endanflugmodus) etc.\  informieren.
%
\item[{\InfoBox}en] Das sind kleine rechteckige Fenster, in denen die vom Piloten gewünschten Daten angezeigt werden. Diese InfoBoxen können statische Informationen beinhalten, als auch dynamische, d.h. man kann in Ihnen auch wählen (z.B. den MC-Wert einstellen). Je nach gewähltem Inhalt lassen sich durch z.B.\ längeres drauftippen die Werte ändern. 

Die InfoBoxen sind zu sog. ''InfoBoxSeiten'' zusammengefasst (von denen es bis zu 8 Stück  gibt)  und können in mehreren Anordnungen dargestellt werden. Z.B.\ am oberen und unteren Rand, rechts oder links, oder auf der rechten Seite des Displays (im Quer- bzw. landscape-Modus). 

Bislang sind diese Anordnungen fix, in zukünftigen Versionen ist jedoch angestrebt, diese Infoboxen auch frei verschiebbar auf dem Display anordnen zu können.
%
\item[Anzeigen] Anzeigen stellen Instrumenten Displays dar (z.B. ein Vario). Alle Anzeigen sind optional und einige
haben lediglich informellen  Character, wenn \textsf{XCSoar} an ein unterstütztes Instrument angeschlossen ist.
%
\item[Button Beschriftungen und Menüs] Hardware-Knöpfe und die Wippe des Gerätes, auf dem \textsf{XCSoar} läuft, können benutzt werden, um Menüs bzw.\ Hauptmenüs aufzurufen, welche typischerweise den Hardwareknöpfen und /oder der Wippe fix zugeordnet sind. Dies erlaubt mitunter einen schnelleren Zugriff als das Bedienen allein des berührungsempfindlichen Bildschirmbereiches (''TouchScreen'').
Falls das entsprechende Gerät über einen solchen TouchScreen verfügt (nahezu alle bis auf \al), können alle Menüs auch
hierüber geöffnet und bedient werden. Diese Buttons sind in schwarzem Text auf hellgrünem/grauen Hintergrund gestaltet.
%
\item[Status Meldungen] Hinweise zum Status des Fluges und auch des Flugzeuges  (z.B.\ Annäherung an einen Luftraum, vergessenes Fahrwerk, Startbestätigung bei Überfliegen der Start/Ziellinie etc\dots ) wird in einer Textbox  über dem Bildschirm eingeblendet.
Dieser Status Meldungen werden angezeigt, um detaillierte, wichtige Informationen zu geben, wenn gewisse, zum Teil konfigurierbarer Ereignisse, auftreten.
%
\item[Dialog Fenster] Die -größeren- Dialogfenster enthalten normalerweise Grafiken und Buttons. Diese werden verwendet, um dem Piloten  detaillierte Daten bezüglich Wegpunktdetails, Statistik und Analyse des bisherigen Fluges, Polaren etc\dots  zu übermitteln.

\item[Hauptmenü] das Hauptmenü wird entweder über einen Doppelklick auf die Bildschirmoberfläche oder aber über Gesten \gesture{Runter - Hoch} aufgerufen.

   Wenn die erscheinenden Buttons nicht innerhalb einer gewissen (einstellbaren) Zeit  betätigt  werden, verschwinden sie von sich aus wieder, um die Kartendarstellung des weiteren Fluges nicht zu behindern.
\end{description}

Die Bedienung von \textsf{XCSoar} ist auf verschiedene Wege möglich:

\begin{itemize}
\item Anklicken eines speziellen Kartenelements
\item Anklicken einer InfoBox und anschließend den Bildschirmmenüs
\item Benutzen einer ''Geste'', z.B. Zeichnen eines Striches von links nach rechts auf dem TouchScreen
 (siehe dazu  \ref{sec:gestures} weiter unten).
\item ''Verschieben'' (dragging) des Bildschirms (berühren und verschieben auf dem TouchScreen unter Fingerdruck).
\item Drücken der Hardware-Knöpfe des  Gerätes.
\item Drücken der Wippe des Gerätes.
\item Drücken  oder Schalten  an einem an \textsf{XCSoar} angeschlossenen Gerät.
\end{itemize}

Je nachdem, welcher Hardware bzw.\ welches Gerät mit \textsf{XCSoar}  benutzt wird, stehen nicht alle dieser Methoden zur Verfügung. Die Belegung der jeweiligen Hardware-Knöpfe können daher unterschiedlich sein.

Bei der \textsf{PC}-Version von \textsf{XCSoar} ist das Anklicken eines Bildschirmelements es mit der Maus identisch der Berührung mittels Touchscreen.

Da der \al nicht über einen TouchScreen verfügt, ist hier die Belegung der Menüs und Elemente grundlegend anders und wird über die dort vorhanden Hardware-knöpfe und Drehgeber vorgenommen.


\section{Button-Beschriftungen und -Menüs}
Das Button-Menü besteht aus einem Satz von Buttons und Menüs und ist wird -bis auf die o.g. genannten Ausnahmen- mittels des TouchScreens aufgerufen (Doppelklick auf die berührungsempfindliche Oberfläche).

Die Benutzung der Buttons und des Button-Menüs  stellt die bevorzugte Art der Interaktion mit \textsf{XCSoar} dar.

\begin{center}
\includegraphics[angle=0,width=\linewidth,keepaspectratio='true']{figures/buttonmenu.png}
\end{center}

\subsection*{Grundlegende Info zur Bedienoberfläche}
Das Menü ist in vier Hauptgruppen mit hoffentlich aussagekräftigen Namen (Nav, Anzeige, Konfig. und Info) unterteilt, welche diverse Untermenüs besitzen. Das spezielle Layout hängt vom jeweils benutzten Gerät ab und kann ebenso vom Bediener angepasst werden.

 \textsf{XCSoar} kann ebenso über an das Gerät angeschlossene externe Eingabegeräte wie Joysticks, externe Tastaturen, game-pads und Multifunktionsgriffe bedient werden. Die hierüber aufzurufenden Funktionen können in großem Rahmen eben falls frei konfiguriert  werden.


Beim  \al  gibt es vier Hauptmenüs, welche durch drücken auf die Knöpfe an der linken Seite aufgerufen werden können. 

Sowie ein Menü aktiviert wurde, erscheinen Untermenüs mit entsprechenden Beschriftungen über der unteren Druckknopfreihe. 
Hiermit kann man sich dann durch die entsprechenden Menüs und Funktionen weiter hindurchmanöverieren. 
Auf der letzten Seite eines jeden Menüs erscheint wieder der Menü-Knopf, mit dem man schließlich wieder auf den 
Bildschirm zurückkommt und alle Menüs verschwinden wieder vom Schirm. 
 


Auf dem \textsf{PC} werden diese Hauptmenü-Funktionen über die Ziffertasten 1, 2, 3 und  4 aufgerufen. Die Tasten 6, 7, 8, 9 und  0 sind dabei mit den
entsprechenden Untermenü-Funktionen der horizontal bzw.\ vertikal angeordneten Reihe von Buttons  zugeordnet.

Bei der PDA Version können die Hauptfunktionen direkt über die vier Hardware-Knöpfe über/unter (je nach Gerät) neben der Schaltwippe aufgerufen werden.

Falls ein Menü aufgerufen wird und nach einer gewissen Zeit keinerlei Eingabe mehr erfolgt, wird das angewählte Menü automatisch wieder geschlossen.  Diese Verzögerungszeit ist frei definierbar.  Auf dem \textsf{PC} kann die ESC- Taste hierzu benutzt werden, beim  \al wird hierzu der PWR/ESC- Knopf benutzt.

Wenn die Menü-Buttons grau erscheinen, ist die entsprechende Funktion nicht verfügbar.

So ist zum Beispiel der Button \textcolor{white}{\button{Wegpunkt Liste}} grau hinterlegt, falls keine Wegpunkt-Liste (besser: keine Wegpunkt-Datei) zur Verfügung steht oder geladen ist.

Einige der Menü-Buttons sind mit dynamischem Text versehen, d.h. die Beschriftung ändert sich entsprechend des Zustandes (gedrückt/nicht gedrückt). Folgendes Verhalten ist hierbei grundlegend festgelegt:


\textbf{Die Beschriftung des Buttons zeigt an, \textcolor[rgb]{0.72,0.03,0.20}{ was geschehen wird}, wenn er \warning gedrückt wird, er zeigt also nicht den aktuellen Zustand an!}\index{Schaltflächen!Anzeigeverhalten}

Wird zum Beispiel auf der Schaltfläche  \bmenu{MC Auto} angezeigt, dann wird ein Klick hierauf  ''Auto McCready'' \textcolor[rgb]{0.72,0.03,0.20}{ einschalten}, und die Beschriftung wechselt zu\bmenu{MC Manuell}\hspace{-3pt}.
In der unten aufgelisteten Liste sind die Standard Beschriftungen aufgeführt.

\subsection*{Übersicht der Menüs }
Im Folgenden wird beschrieben, wie das Menü allgemein aufgebaut ist.
Dies gilt für alle Plattformen (\textsf{PC}, Android, \al PDA, etc\dots

Die entsprechenden Funktionen werden in nachfolgenden Kapiteln ausführlicher beschrieben.

\textsf{\al}:

Die vier primären Menübuttons (Hauptmenü) werden beim \al durch Drücken der vertikal angeordneten Knöpfe an der linken Seite wie folgt von unten nach oben aktiviert:

\begin{jspecs}
\item[\bmenu{\strut Nav}]       Kontrolle und Einstellungen bezüglich Navigation, hauptsächlich gedacht für Tasks (Aufgaben).
\item[\bmenu{\strut Anzeige}] Einstellungen der Anzeige von \textsf{XCSoar}
\item[\bmenu{\strut Konfig.}]  Konfiguration von \textsf{XCSoar}, angeschlossener Geräte,
                                      Einstellungen während des Fluges (Wind, Ballast, MC %etc.\
\item[\bmenu{\strut Info}]       Aufruf diverser Informations Fenster,  Checkliste, Lufträume, Wetter etc.\
\end{jspecs}


In der \textsf{\textsf{PC}}-Version werden hierfür die Tasten  1, 2, 3 und 4 verwendet.

\subsection*{Navigationsmenü (Nav)}\index{Men:Navigations-Menue}
\jindent{\bmenus{Aufgabe}}{ Blendet Aufgabenverwaltung und - Rechner ein. }
\jindent{\bmenut{Vorheriger}{Wegpunkt}}{ Wählt den vorherigen Wegpunkt innerhalb der Aufgabe an. }
\jindent{\bmenut{Nächster}{Wegpunkt}}{ Wählt den nächsten Wegpunkt innerhalb der Aufgabe an.}
\jindent{\bmenut{Wegpunkt}{Liste}}{ Zeigt die Wegpunkt - Auswahl an.}
\jindent{\bmenus{Alternativen}}{ Zeigt eine  Schnellauswahl-Liste der nahesten landbarer und erreichbarer Plätze in der Nähe an.}
%
%
%
\jindent{\bmenut{Aufgabe}{Abbruch}}{ Bricht die Aufgabe ab und geht über in den ''Lustflug-Modus''.}
\jindent{\bmenus{Gehezu}}{ Blendet die Wegpunkt-Auswahl 1ein und aktiviert den ''Gehezu''-Modus für den ausgewählten Wegpunkt. }
\jindent{\bmenus{Target}}{ Zeigt den Target-Dialog. Entscheidend für AAT-Aufgabe. Hier können u.a. AAT-Wegpunkte verschoben werden. }
\jindent{\bmenut{Wegpunkt}{Details}}{ Blendet Details zum aktuell gewählten Wegpunkt ein.}


\subsection*{Anzeige Menü}\index{Men:Anzeige-Menue}
\jindent{\bmenut{Zoom}{herein}}{ Zoomt die Karte herein (Kartenausschnitt kleiner, mehr Details. }
\jindent{\bmenut{Zoom}{heraus}}{ Zoomt  die Karte heraus (Kartenausschnitt größer,weniger Details. }
\jindent{\bmenut{Zoom}{Auto}}{ Schaltet zwischen  Zoom Auto und  Zoom Manuell hin und her. }
\jindent{\bmenut{Marke}{Setzen}}{ Setzt einen Marker der aktuellen Position auf der Karte. }
\jindent{\bmenut{Verschieben}{Ein}}{ Schaltet den Verschiebe-Modus der Karte ein. }
%
%\jindent{\bmenut{Beschriftungen}{Aufgaben\& Landeplätze}}{ Blendet die Beschriftungen der Wegpunkte innerhalb Aufgaben ein oder aus.. }
\jindent{\bmenud{Beschriftungen}{Aufgaben}{\& Landeplätze}}{ Blendet die Beschriftungen der Wegpunkte innerhalb Aufgaben ein oder aus.. }
\jindent{\bmenut{Spur}{Kurz}}{ Schaltet die Spur des bisherigen Fluges (kurz, lang, aus). }
\jindent{\bmenut{Gelände}{Aus}}{ Schaltet die Darstellung de Geländes auf der Karte ein oder aus. }
\jindent{\bmenut{Topo.}{Aus}}{ Schaltet die Topologie ein oder aus. }
\jindent{\bmenut{Info}{Auto}}{ Schaltet die Infoboxen (Kurbeln, AAT, Endanflug, Auto). }
%
%
%
\subsection*{Konfigurationsmenü (Konfig.\ )}\index{Men:Konfig-Menue}
\jindent{\bmenut{MC}{$+$}}{ McCready größer. }
\jindent{\bmenut{MC}{$-$}}{ McCready kleiner. }
\jindent{\bmenut{MC}{Auto}}{ Schaltet zwischen McCready Auto oder und McCready manuell. }
\jindent{\bmenut{Flug}{Einstellungen}}{ Zeigt das Flug-Einstellungsfenster  (Mücken, Ballast, QNH). }
\jindent{\bmenut{Wind}{Einstellung}}{ Ruft das Windeinstellungsmenü auf.}
%
%
%
\jindent{\bmenus{Vario}}{ Kontrolle des VEGA - Varios, dies enthält ein Untermenü.  Nur wählbar, wenn auch ein VEGA angeschlossen ist. }
\jindent{\bmenut{System}{Einstellungen}}{ Öffnet die \textsf{XCSoar}-System-Einstellungen. Hierunter liegen 21 weitere Dialogfenster zur kompletten Konfiguration \textsf{XCSoar}s. }
\jindent{\bmenut{Luftraum}{Einstellungen}}{ Öffnet den Luftraum-Dialog. }
\jindent{\bmenut{Logger}{Start}}{ Schaltet den internen \textsf{XCSoar}- IGC-Logger an oder aus.. }
\jindent{\bmenus{Wiedergabe}}{ Startet das Wiedergabe-Fenster aufgezeichneter Flüge. }
%
\jindent{\bmenut{Roh}{Logger}}{ Schaltet den NMEA-Rohdaten-Logger ein oder aus (nur benutzt zur Entwicklung/Debuggen von \textsf{XCSoar}). }
\jindent{\bmenus{NMEA}}{ Öffnet den NMEA (z.B. GPS-Anschluß) Dialog. }
\jindent{\bmenus{Flugzeug}}{ Öffnet den Dialog zur Eingabe von flugzeugspezifischen Details. }
%
%
%
\subsection*{Informations Menü  (Info)}\index{Men:Informations-Menue}
\jindent{\bmenus{FLARMRadar}}{ Öffnet das Flarm-Radar-Fenster (Nur, wenn Flarm angeschlossen und erkannt) }
\jindent{\bmenut{METAR}{TAF}}{ Zeigt das METAR/TAF Fenster. }
\jindent{\bmenut{Naher}{Luftraum}}{ Zeigt die Details des nächstgelegenen Luftraumes. }
\jindent{\bmenut{Check}{Liste}}{ Öffnet die Checkliste.\footnote{Nur, wenn \texttt{\textsf{XCSoar}-checklist.txt} im Verzeichnis \texttt{\textsf{XCSoar}Data} vorhanden}}
\jindent{\bmenus{Analyse}}{ Öffnet den Analyse/Statistik-Dialog. }
%
\jindent{\bmenus{Status}}{ Öffnet den Status-Dialog. }
\jindent{\bmenus{Weather}}{ Öffnet das Wetter-Dialog-Fenster. }
\jindent{\bmenut{Team}{Code}}{ Einblendung des TEAM-Code Dialoges.}
\jindent{\bmenut{FLARM}{Details}}{ Öffnet das FLARM-Detail-Fenster.}
\jindent{\bmenus{Zentrierhilfe}}{ Öffnet die Zentrierhilfe.}
%
\jindent{\bmenus{Mitwirkende}}{ Zeigt Versionsnummer und eine Reihe der Mitwirkenden an \textsf{XCSoar} an. }
\jindent{\bmenut{Naher}{Wendepunkt}}{ Zeigt Details des nächst gelegenen Wegpunktes an. }
\jindent{\bmenut{Wiederhole}{Meldung}}{ Wiederholt die zuletzt ausgegebene Meldung \textsf{XCSoar}s. }
%
%
%
\subsection*{Variometer-Untermenüs (Vario) des Konfigurationsmenüs (Konfig.\ )  }
 Die Funktionen dieses Menüs sind ausschließlich anwählbar, wenn \warning das VEGA Vario als Gerät an \textsf{XCSoar} angeschlossen ist.
\begin{center}

\end{center}
\jindent{\bmenut{Airframe}{Switches}}{Displays airframe switch values. }
\jindent{\bmenut{Setup}{Audio}}{ Adjusts volume of sounds produced by \textsf{XCSoar} as well as certain speech announcements by the VEGA intelligent variometer. }
\jindent{\bmenut{Manual}{Demo}}{ Activates VEGA variometer manual tone demo. }
\jindent{\bmenut{Setup}{Stall}}{ Opens VEGA stall monitor setup dialog. }
\jindent{\bmenut{ASI}{Zero}}{ Zeros the airspeed indicator. }
\jindent{\bmenut{Accel}{Zero}}{ Levels/zeros the accelerometers. }
\jindent{\bmenus{Speichern}}{Speichert die Einstellungen des VEGA im EEPROM. }
\jindent{\bmenut{Cruise}{Demo}}{ Activates VEGA variometer cruise tone demo. }
\jindent{\bmenut{Climb}{Demo}}{ Activates VEGA variometer climb tone demo. }
\subsection*{Verschieben-Untermenü des Anzeige-Menüs}
\jindent{\bmenut{Verschieben}{aus}}{ Schaltet Verschiebemodus aus. }
\jindent{\bmenut{Zoom}{herein}}{ Zoomt Karte herein. }
\jindent{\bmenut{Zoom}{heraus}}{ Zooms Karte heraus. }
\jindent{\bmenut{Naher}{Wegpunkt}}{ Zeigt das Wegpunkt-Detail Fenster, oder aber, falls die Karte verschoben wurde, das Wegpunkt-Detail Fenster zu jenem Wegpunkt, an dem sich das Fadenkreuz (sichtbar beim Verschieben, immer in der Mitte der Karte), befindet. }
\subsection*{Standard-Buttons }
Wenn keinerlei Menü aktiv ist, (der sog.\ \textsl{default-mode}) dann sind beim \al bzw.\ beim PC den horizontalen Knöpfen die folgenden Funktionen zugeordnet: 

\begin{center}
\begin{tabular}{c c c c c}
 6 & 7 & 8 & 9 & 0 \\[5pt] 
 F5 & F6 & F7 & F8 & F9\\[5pt] 
\smenut{Flug}{Einstellung} & \smenut{Task}{Calc} & \smenut{Task}{Edit} &
\smenut{Arm}{Advance} & \smenut{Marker}{setzen} \\
\end{tabular}
\end{center}

In der oberen Reihe die \al-Tasten, in der unteren Reihe die \textsf{PC}-Tasten.
Unten die entsprechende Standard-Funktionsbelegung.

Wenn der ESC-Knopf des \al kurz gedrückt wird, erscheint die Beschreibung der Knöpfe.
Bei allen anderen Geräten / Plattformen wirken die Cursor-Tasten wie folgt:


\begin{itemize}
\item[Hoch Taste] Zoom herein
\item[Runter Taste] Zoom hinaus
\item[Links Taste] Nächster InfoBox-Set
\item[Rechts Taste] Letzter InfoBox-Set
\item[Enter] Bestätigen/Löschen von Status-Mitteilungen oder aber Flarm-Meldungen unterdrücken, wenn diese erscheinen und keine Warnung aktiv ist
\end{itemize}
Der Drehknopf des \al  ist im default-mode wie folgt belegt:
\begin{itemize}
\item[Äußerer Knopf gegen Uhrzeigersinn ] Zoom herein
\item[Äußerer Knopf im Uhrzeigersinn      ]  Zoom heraus
\item[Innerer Knopf gegen Uhrzeigersinn  ] (Nicht belegt)
\item[Innerer Knopf im Uhrzeigersinn       ] (Nicht belegt)
\item[Drücken des Knopfes] Bestätigen von Status Meldungen und Luftraum Warnungen/Hinweisen
\end{itemize}


In Dialog Fenster sind den Drehknöpfen des \al folgende , Cursor-ähnliche Funktionen zugewiesen:
\begin{itemize}
\item[Äußerer Knopf gegen Uhrzeigersinn       ] Cursor hoch
\item[Äußerer Knopf im Uhrzeigersinn             ] Cursor runter
\item[Innerer Knopf gegen Uhrzeigersinn        ] Cursor links
\item[Innerer Knopf im Uhrzeigersinn              ] Cursor rechts
\item[Drücken des Knopfes ] Bestätigen/OK-Taste
\end{itemize}

Beim \al können alternativ die folgenden Knöpfe entlang des Gehäuserandes für die Navigation innerhalb von Dialogen benutzt werden:

Der F4-Knopf kann innerhalb von Dialogen als Alternative zur ENTER-Taste (Drücken des Drehknopfes) benutzt werden.
F6 und F7 können benutzt werden um die nächste bzw.\ letzte Seite innerhalb von Dialogseiten anzuwählen.
\subsection*{Dynamische Menü Beschriftungen}
Die meisten Menüs haben dynamische Beschriftungen, um klarer hervorzuheben, welche Aktion diesem Menü zugeordnet ist, wenn der jeweilige Button bedient wird.

Falls ein Menü oder Button grau hinterlegt bzw.\ die Schrift nicht schwarz sondern hellgrau ist, \textcolor{white}{\button{so wie hier}} so steht diese Funktion zum jeweiligen Zeitpunkt \emph{nicht} zur Verfügung und ein Klick darauf bewirkt \textbf{gar nichts}. 

Folgende Konvention zur Belegung bzw. Beschriftung der Buttons und Menüs wird  innerhalb \textsf{XCSoar} verwendet:

Beispiel:

Erscheint z.B.:  ''Licht an'' als Beschriftung, so  wird das Licht \emph{an}geschaltet und  es erscheint dann als Beschriftung z.B. ''Licht aus''.  Ein erneute Drücken schaltet dann das Licht wieder \emph{aus}.

Stehen mehrere Funktionen eines Knopfes zur Verfügung, (z.B.\ zusätzlich noch ''Licht gedimmt'', dann kann durch Drücken des Buttons zyklisch durch diese Funktionen durchgewählt werden:


\begin{center}
''Licht an'' $\rightarrow$ ''Licht gedämmt'' $\rightarrow$ ''Licht aus'' $\rightarrow$'' Licht an'' $\rightarrow$ ''Licht gedämmt'' und so weiter\dots
\end{center}

Eine Auswahl von dynamisch beschrifteten Buttons ist als Beispiel hier unten aufgelistet:

\begin{description}
\item[\bmenu{Nächster Wendepunkt}]
  Grau, wenn keine  Aufgabe aktiv ist, oder aber wenn der nächste Wendepunkt gleich dem Zielpunkt (also dem letzten Wegpunkt) ist.

 Wenn der nächste Wegpunkt gleich dem Ziel ist, dann erscheint '' Wendepunkt Ziel''.
\item[\bmenu{Letzter Wendepunkt}]
  Grau, wenn keine  Aufgabe aktiv ist, oder aber wenn der vorherige Wendepunkt gleich dem Startpunkt ist und es keine verlegten Startpunkte gibt.

  Falls verlegte Startpunkte vorhanden sind und der aktive Wendepunkt gleich dem Start ist, wird ''zyklischer Start'' angezeigt um aus den diversen Startpunkten einen entsprechenden auszuwählen.

Wenn der aktive Wegpunkt gleich dem ersten Wegpunkt nach dem Start ist, dann erscheint  ''Wendepunkt Start''.
\item[\bmenu{Beschriftungen}]
Angezeigt wird hierbei folgendes: '' Beschriftung Aufgabe \& Landeplätze'', ''Beschriftung Keine'', Beschriftung Alle.

\item[\bmenu{Nächster Wendepunkt}]
 Grau,  wenn keine Wegpunkt-Datenbank hinterlegt ist.
 \item[\bmenu{vorheriger Wendepunkt }]
  Grau,  wenn keine Wegpunkt-Datenbank hinterlegt ist.
\item[\bmenu{Wegpunktliste }]
  Grau,  wenn keine Wegpunkt-Datenbank hinterlegt ist.
\end{description}

\section{{\InfoBox}en}\index{InfoBoxen}

\textsf{XCSoar} stellt nahezu alle wichtigen Werte mittels  sog.\  {\InfoBox}en dar. 
Diese {\InfoBox}en können zu InfoBoxfenstern (s.u.) zusammengestellt werden und mit
\emph{gewissen Einschränkungen \footnote{In Planung ist, in kommenden Versionen einige oder alle 
InfoBoxen auch frei verschiebbar darzustellen}} beliebig auf dem Bildschirm plaziert werden. 
Die Informationen, welche in den  {\InfoBox}en dargestellt werden sollen,  können aus einer großen Anzahl (gelistet im Kapitel~\ref{cha:infobox}) ausgewählt werden.

Die mögliche Anzahl und das Layout der {\InfoBox}en  hängen von der Ausrichtung des Bildschirmes und von der Größe bzw. Auflösung des Bildschirms des entsprechende Gerätes ab. Für ein typisches 320$\times$240 Pixel-Display (Pocket-\textsf{PC} im ''Hochkant''-Modus\footnote{Aufruf mit \texttt{xcsoar.exe --portrait}}) stehen jeweils vier {\InfoBox}en am oberen und unteren Rand des Displays zur Verfügung.
Im Querformat, sind neun {\InfoBox}en an der rechten Seite des Displays vorgesehen. Siehe Bild unten:
\begin{center}
\includegraphics[angle=0,width=0.35\linewidth,keepaspectratio='true']{figures/infoboxes.png}
\end{center}
\menulabel{\bmenut{Konfig.}{2/3}\blink~\bmenus{System}}
die entsprechende Anordrnung ist auswählbar über
 \button{\strut Aussehen}\blink~\button{\strut Anordnung}
%\begin{quote} \smenus{Konfig.\ }\blink\smenus{Konfig.\  }\blink%
%\smenut{System}{Einstellung}\blink\seite{19}
%\end{quote}
\index{Infobox!Seiten!Anordnung}

\subsection*{Bildschirmanzeige-Modi / InfoBox-Seiten}\index{InfoBoxen!Seiten}

\textsf{XCSoar} gestattet dem Piloten einen ganzen Satz an individuell belegten Infobox-Seiten zu erstellen.
Dies dient dazu, bestimmte Werte, die zum Beispiel vor allem beim Kurbeln  benötigt werden übersichtlich und schnell darzustellen.
Ein anderer Satz an Infoboxen könnte zum Beispiel für den Endanflug oder aber während Wettbewerbsaufgaben erstellt und angezeigt werden.

\textsf{XCSoar} so kann so konfiguriert werden, daß es automatisch auf die jeweilige Seite umschaltet, oder aber ein manuelles Umschalten erforderlich ist. Der Name der aktuellen InfoboxSeite kann ganz unten links am Rande der Karte abgelesen werden.

Es ist ebenfalls möglich, überhaupt keine Infoboxen auf der Karte darzustellen, in diesem Fall erscheint ausschließlich die Landkarte mit Topologie bzw. Karte auf dem Bildschirm.

Um durch die diversen Infobox-Seiten hindurch zu schalten, kann der TouchScreen benutzt werden:
\gesture{Right} oder \gesture{Left}

Auf PDA ist es ebenfalls möglich,  mittels der Schaltwippe  (nach rechts) durch die diversen Seiten hindurch zu schalten.

\subsection*{Ändern von Werten in den InfoBoxen}\index{InfoBoxen!Werte Ändern}
(Dieser Abschnitt ist ausschließlich gültig, wenn ein Touchscreen oder eine Maus zur Verfügung steht.)

Einige {\InfoBox}-Werte können durch drücken auf den Touchscreen oder aber mit der Maus vom Piloten geändert
Hierzu müssen die entsprechenden {\InfoBox}en  \emph{langanhaltend} gedrückt werden.
Ein kurzes Klicken reicht dazu nicht.  Nach dem loslassen erscheinen weitere {\InfoBox}en:

\begin{description}
\item[\bmenu{Edit}]
Erlaubt dem Piloten den Wert der entsprechenden {\InfoBox} zu ändern (zum Beispiel den McCready Wert).

\item[\bmenu{Setup}]
hiermit kann das Verhalten der entsprechenden {\InfoBox} geändert werden. Beispielsweise kann der McCready Wert von Auto auf Manuell geschaltet werden.
Weiterhin kann hier durch Klicken auf  \bmenu{Setup InfoBox} innerhalb dieser {\InfoBox}  eine Liste mit allen verfügbaren {\InfoBox}en dargestellt werden, welche an dieser Stelle der InfoboxSeite erscheinen soll. hierzu erscheint ein Button ''Infobox Einstellungen''

\item[\bmenu{Schließen}] Na, was wohl\dots
\end{description}

{\InfoBox}en, deren Werte geändert werden können, sind zum Beispiel der McCready wert, oder aber die Windgeschwindigkeit und Richtung. Im Simulator-Modus, zählt auch die Höhe, sowie die Übergrundgeschwindigkeit hierzu.

\subsection*{Ändern von {\InfoBox}en}
{\InfoBox}en können entweder über das Konfigurationsmenü geändert werden 
\menulabel{\bmenut{Konfig.}{2/3}\blink~\bmenus{System}}
%\begin{quote}
%\smenus{Konfig.\ }\blink\smenus{Konfig.\  }\blink\smenut{System}%{Einstellung}\blink\seite{21}
%\end{quote}  
oder durch einen länger anhaltenden Druck auf die entsprechende  {\InfoBox}, welche geändert werden soll.

Anschließend erscheint ein kleines Menü, in welchem mittels  \button{\strut Einstellung}\blink~ \menulabel{\bmenus{Aussehen}\blink~\bmenus{InfoBox Sets}}  \button{\strut InfoBox wechseln} eine beliebige Box ausgewählt werden kann, welche anschließend  an der Stelle der angetippten erscheinen soll. 
Eine kurze Beschreibung der Infoboxen ist hinterlegt.



\section{Status Meldungen}
Status-Meldungen erscheinen im oberen Bereich des Bildschirms über die gesamte Kartenbreite und sind für die kurzfristige Anzeige von Meldungen gedacht. Nach einer einstellbaren Zeit verschwinden diese Meldungen von selbst. Verschiedene Meldungen können mit verschiedenen Anzeigezeiten belegt werden.

Zusätzlich können bestimmte Statusmeldungen mit einer Bestätigungs-Aufforderung versehen sein.
Die Bestätigung erfolgt entweder durch Drücken des Drehknopf es beim \al, Drücken der Statusmeldung auf dem TouchScreen oder aber Anklicken mit der Maus (\textsf{PC}).  Zusätzlich können benutzerdefinierte Buttons definiert werden, welche die zuletzt gezeigte Status Meldung wiederholt.

Hier ein paar typische Statusmeldungen:
\begin{itemize}
\item Luftraum Warnungen
\item Meldungen der Programmes an sich zur Optik wie z.B. Ändern des Anzeigemodus (hochkant/quer, Auflösung)
\item Meldungen des Segelflugrechners wie z.B. Start, Startlinie überflogen, Erreichen eines Wendepunktes etc\dots  )
\end{itemize}

Beachte, daß Statusmeldungen nicht erscheinen, während ein Dialog auf dem Bildschirm erscheint. Diese Meldung wird gespeichert und erst dann angezeigt, nachdem das Dialogfenster verlassen wurde.

\section{Dialog Fenster}\label{sec:dialog-windows}

\textsf{XCSoar} verfügt über etliche Dialogfenster, welche aufgerufen werden können um zusätzliche Informationen darzustellen. Weiterhin werden Dialogfenster benutzt, um zum Beispiel Konfigurationseinstellungen vorzunehmen, Aufgaben (tasks) zu erstellen, usw.\

Manche Dialoge zeigen einfach nur Informationen und fordern keinerlei Aktion des Piloten, andere dagegen zeigen Felder, deren Werte geändert werden können und erfordern eine Eingabe vom Piloten oder einfach nur das Drücken eines anderen Buttons.

Ein vom \textsf{PC} gewohnter Cursor erscheint über dem aktiven Button oder Datenfeld.
Wenn die hoch/runter Tasten gedrückt werden, oder aber der äußere Drehknopf auf dem \al, rückt der Cursor zum nächsten bzw.\ vorherigen Listenelement.

Für lange Listenelemente und scrollbaren Text, werden die Hoch- und Runter-Tasten zur Navigation durch den Text benützt, und die links- und Rechts-Tasten bewegen jeweils eine komplette Seite vor oder zurück.

Für die PDA und \textsf{PC}-Version, können solche Listen geändert werden, indem das jeweilige Listenelement angeklickt wird.
Wenn ein Listenelement angewählt wurde, ist ein weiterer Klick gleichbedeutend mit der Enter-Taste bzw. einem O.K.

Durch Drücken der rechts/links Tasten Innerhalb eines gewählten Listenelement des (bzw. Drehen des inneren Knopfes auf dem \al) werden die Werte dieses Listenelement des geändert. Durch Drücken der OK- bzw. Enter-Taste dies (oder Drücken des Drehknopf das auf dem Alter) wird der Button aktiviert oder ein Listenelement aus einer Liste ausgewählt.

Dialoge werden typischerweise vom Button Menü gestartet. Viele der Dialog Fenster haben mehrere Seiten mit Information, und sind innerhalb \textsf{XCSoar} immer gleich zu bedienen:

Die \button{$<$} und \button{$>$} Buttons wählen die vorherige bzw. die nächste Seite aus und der
\button{Close} Button dient  zum Verlassen des Dialogs. (Gleichbedeutend ist die ESC-Taste auf dem \textsf{PC} bzw. die ESC/PWR-Taste  auf dem \al.)

Die Dialogfenster müssen explizit vom Piloten geschlossen werden. Wenn ein Dialogfenster geöffnet ist, bleibt es so lange geöffnet, bis der Dialog geschlossen wird. In manchen Dialogfenstern werden bestimmte  Werte nicht dargestellt, sofern sie irrelevant sind. (so zum Beispiel die Delta-Zeit einer AAT-Aufgabe, wenn keine AAT-Aufgabe geflogen wird).

Eine Liste der wichtigsten Dialoge ist hier aufgelistet:
\begin{description}
\item[Flug Einstellung] Einstellung des QNH, der Polare, Mückenbeladung, Wasserballast
\item[Wind Einstellung] Einstellungen des Windes (Richtung, Stärke und Art der Berechnung )
\item[Wegpunkt Details] Details zu Wegpunkten wie Peilung (Bearing), Höhe, Frequenz (falls vorhanden), Sonnenuntergang  etc\dots
\item[Wegpunkt Auswahl] Auswahl eines Wegpunktes aus der Datenbank
\item[Analyse] Mehrere Seiten mit analytischen Details zum bisherigen Flugweg (Strecke, mittleres Steigen, echte, erflogene Polare u.v.m.)
\item[Status] Dieser Dialog bringt eine Zusammenfassung der aktuellen Situation/Konfiguration  des Flugzeuges, des Systems, der Aufgabe von Staat und Zeiten der Aufgabe.
\item[Checkliste] Eine Checkliste, welche vom Piloten selber erstellt werden kann und mehrere Seiten umfassen darf
\item[Konfiguration] Konfiguration von  \textsf{XCSoar} und angeschlossenen Geräten
\item[Luftraum] Konfiguration und Darstellung der Farben und Erscheinung von Lufträumen
\item[Airspace filter] Controls enabling and disabling the display and warnings
of each airspace class
\item[Team code] Einstellungen und Übermittlung von Koordinaten für den Teamflug
\item[Devices] Allows setup of various external devices (e.g. glide computers, FLARM, etc.).
\item[Setup Plane] Allows easy reconfiguration of the plane-dependant settings (e.g. polar, competition ID, etc.) by choosing from a list of previously-created plane profiles.
\end{description}

Diese Dialoge werden in nachfolgenden Kapiteln behandelt werden. Ausgenommen hiervon sind der Checklisten-, der Status-  und der Texteingabe-Dialog, welche hier unten im Anhang behandelt werden.

\subsection*{Checklisten Dialog}\index{Checkliste}

%\begin{quote}
%\smenus{Info}\blink\smenus{Checkliste}
%\end{quote}

Der Checklisten-Dialog ist gedacht, um sogar mehrere Seiten von Erinnerungen und frei \menulabel{\bmenus{Info}\blink~\bmenus{Checkliste}}definierbarem Text des Piloten aufzunehmen.
Die typische Anwendung hierfür ist eine Checkliste vor dem Start. Diese Checkliste ist ein reines Textfile und kann solche Dinge enthalten wie z.B.:\


täglicher Check, Vorflugcheck, Außenlandungscheck, Pinkeltüten Einpacken, Erinnerung an Treibstoff, Klappen beim Start verriegelt usw.\ je nach Gusto des Piloten.
Die  \button{$<$} und \button{$>$} Buttons können die jeweils nächste Seite der Checkliste anwählen.

Damit die Checkliste entsprechend aufgerufen werden kann, muß ein entsprechendes Textfile \textsf{\textsf{XCSoar}-checklist.txt} im Verzeichnis \textsf{\textsf{XCSoar}Data} existieren!


\begin{center}
\includegraphics[angle=0,width=0.8\linewidth,keepaspectratio='true']{figures/checklist.png}
\end{center}

Wenn kein solches File existiert, erscheint eine Meldung im Fenster: \verb|Erstelle xcsoar-checklist.txt|

\subsection*{Status Dialog}

%Dieser Dialog kann aufgerufen werden über
%\begin{quote}
%\smenus{Info}\blink\smenus{Info}\blink\smenus{Status}
%\end{quote}

Der Statusdialog ist ein Dialog mit mehreren Seiten, welcher eine Übersicht über das \menulabel{\bmenut{Info}{2/3}\blink\bmenus{Status}}Flugzeug, das System, die Aufgabe, Regeln zur Aufgabe und Zeiten darstellt. Ein Drücken auf \button{$<$} bzw.  \button{$>$} wählt die vorherige bzw.\  nächste Seite an.

\textbf{Achtung!}

Beachtet, daß die Werte in diesem Dialog nicht aktualisiert werden, solange die Seite dargestellt wird.
\warning Das bedeutet, daß Positionen, Zeit, usw.\ nicht aktualisiert werden. Um die aktualisierten Werte dieser Seite zu betrachten, ist notwendig einen anderen Dialog zu öffnen, und anschließend wieder in diesen Dialog zu klicken.

\begin{description}
\item[\button{Flug}] Zeigt die aktuelle Position des Flugzeuges (Breiten - und Längengrad), die Höhe, den maximalen Höhengewinn, den nächstgelegenen Wegpunkt, dessen Entfernung  und die Peilung (bearing) dorthin.
\begin{center}

\includegraphics[angle=0,width=0.5\linewidth,keepaspectratio='true']{figures/status-aircraft.png}
\end{center}

\item[\button{System}] Zeigt den Status von angeschlossenen Geräten, Satellitenempfang,  Batterielevel, Verfügbarkeit von \fl und  Logger an. 

\begin{center}
\includegraphics[angle=0,width=0.5\linewidth,keepaspectratio='true']{figures/status-system.png}
\end{center}

\item[\button{Aufgabe}] Anzeige von Aufgabenzeit, verbleibender Zeit, Entfernungen und Aufgabengeschwindigkeit
\begin{center}
\includegraphics[angle=0,width=0.5\linewidth,keepaspectratio='true']{figures/status-task.png}
\end{center}

\item[\button{Regeln}] Zeigt Gültigkeit des Abfluges, Gültigkeit der Überquerung von Start/Ziellinien gemäß der eingegebenen Regeln der Aufgabe, Startpunkt, Abfluggeschwindigkeit, minimal zulässiger Höhe am Ziel, Gültigkeit der Ankunft am Ziel
\begin{center}
\includegraphics[angle=0,width=0.5\linewidth,keepaspectratio='true']{figures/status-rules.png}
\end{center}

\item[\button{Zeiten}] Anzeige aller Zeiten wie z.B.: Lokalzeit, GPS-Zeit, Flugdauer, Start-  und Landezeit sowie den Sonnenuntergang.
\begin{center}
\includegraphics[angle=0,width=0.5\linewidth,keepaspectratio='true']{figures/status-times.png}
\end{center}
\end{description}

\subsection*{Eingabe von Text} \label{sec:textentry}\index{Texteingabe}\index{Eingabe von Text}



Zwei Möglichkeiten der Eingabe sind möglich und vorgesehen:  ''Ranglisten''  und ''Tastatur''.
\menulabel{\bmenut{Konfig.}{2/3}\blink~\bmenus{System}} Voreingestellt ist ''Tastatur''-welche für fast alle Endgeräte -bis auf \al sinnvoller und erheblich schneller und bequemer erscheint.

Um die entsprechenden  Buchstaben einzugeben, werden im Ranglisten-Stil die A+ und  A- \menulabel{\button{Aussehen}\blink~\bmenut{Sprache}{Eingaben}} Buttons benutzt.
Drücken auf  \button{$<$} bzw.\   \button{$>$} schiebt den Cursor an die entsprechenden Stelle.

Andern kann man den Eingabestil dann wie folgt: \button{Texteingabestil}

%\begin{quote}
%\smenus{Konfig.\ }\blink\smenut{System}{Einstellung}\blink\seite{18}
%\end{quote} Unterpunkt 


\begin{center}
\includegraphics[angle=0,width=0.6\linewidth,keepaspectratio='true']{figures/textentry.png}
\end{center}

Um Text mit dem TouchScreen einzugeben, ist es ratsam auf die ''Tastatur''- bzw. Voreinstellung zu wechseln, und einfach einen Buchstaben nach dem anderen mit der TouchScreen Tastatur eingeben.

 In einigen Dialogen (z.B. beim Editieren und oder Eingeben von Wegpunkten) schlägt \textsf{XCSoar} sofort den/die nächsten passenden Einträge aus der Datenbank vor bzw. \tip  unterdrückt nicht passende/nicht vorhandene Einträge, sodaß die Eingabe sehr schnell und komfortabel von sich geht.

Um den letzten Buchstaben zu löschen, drücke den  \button{$<-$} button.

\begin{center}
\includegraphics[angle=0,width=0.6\linewidth,keepaspectratio='true']{figures/textentry_keyboard.png}
\end{center}

Drücke auf  \button{Schließen} um die Eingabe zu übernehmen.

\section{Warntöne}\index{Warntöne}\index{Hinweistöne}

\textsf{XCSoar} erzeugt Geräusche für diverse Ereignisse, welche vom Bediener/Piloten  nach Belieben konfiguriert werden können.
Hierzu siehe~\ref{sec:status} für detaillierte Konfigurierung.

Wenn \textsf{XCSoar} an das VEGA-Variometer angeschlossen ist,  sendet \textsf{XCSoar} Kommando-Strings an das Subsystem des VEGA Sprachsystems, um von diesem Sprachausgaben und Warnungen erzeugen zu lassen, z.B.:

\begin{itemize}
\item Endanflug durch Gelände
\item Anfliegen/erreichen eines Wegpunktes
\item Luftraum Warnungen
\end{itemize}

\section{Bildschirm}

Ein paar Einstellungen der Bildschirmdarstellung und der Details auf der Karte können \menulabel{\bmenut{Konfig.}{2/3}\blink\bmenus{System}}gewählt werden. Der meistverwandte und wichtigste Punkt ist hierbei wohl, ob die {\InfoBox}en weiß auf schwarz oder schwarz auf  weiß - \button{\strut invertierte InfoBox} genannt, dargestellt werden sollen. \menulabel{\button{\strut Aussehen}\blink\button{\strut Anordnung}}

Die Einstellung der Helligkeit beim \al erfolgt gemäß des {\em \al User's Manual}.


\section{Hilfe System}
  Für die meisten Dialoge ist nun ein Hilfesystem vorhanden.
  Wenn eine Eigenschaft bzw.\ ein Fenster aufgerufen wurde, drücke auf \button{Hilfe} und eine entsprechender Hilfstext wird erscheinen, der die Auswahlmöglichkeiten  beschreibt.

\section{Gestures}\label{sec:gestures}\index{Gesten}
Seit Version 6.0 unterstützt \textsf{XCSoar} auch sog.\  ''Gesten''.

Im nachfolgenden Bild sind die derzeit verfügbaren Gesten gelistet: 
\begin{center}
 \includegraphics[angle=0,width=0.75\linewidth,keepaspectratio='true']{figures/how2gestures.png}
\end{center}

Um diese zu aktivieren muß diese Funktion im Konfigurationsmenü unter System Setup/Interface aktiviert werden. \index{Gesten!Aktivieren}

Es handelt sich hierbei um eine ganz normale Arbeitsweise, wie sie sie jeder kennt, der mit einem Touchscreen-Gerät

\renewcommand{\gesture}[2]{\marginlabel{{\it#1{\hspace{1em}}}\parbox{1.1cm}
{\hspace{-3mm}\includegraphics[width=0.7cm]{figures/gesture.png}\hspace{-7.75em}\it#2}
}}

\subsection*{Gesten auf der Karte}
Folgende Gesten sind derzeit verfügbar:

%
\begin{itemize}\itemsep1.25em
\item[]\gesture{Runter}{}\textbf{Zoom:}\\  Zoom größer - Kartendarstellung erfolgt in einem größeren Maßstab  
\item[]\gesture{Hoch}{} \textbf{Zoom:}\\ Zoom kleiner - Kartendarstellung erfolgt in einem größeren Maßstab  
\item[]\gesture{Links}{} \textbf{Nächste Seite:}\\ die nächste Seite der InfoBoxSets bzw.\  -Seiten wird angezeigt.  
\item[]\gesture{Rechts}{} \textbf{Vorherige Seite:}\\ Die vorherige Seite der InfoBoxSets bzw.\ -Seiten wird angezeigt.  
\item[]\gesture{Runter Rechts}{} \textbf{Wegpunkte:}\\ Die Wegpunkauswahl mit dem Wegpunktfilter tliste wird angezeigt.   
\item[]\gesture{Runter Links}{} \textbf{Alternative:}\\ Die alternativen Wegpunkte werden angezeigt    
\item[]\gesture{Rechts Runter}{} \textbf{Aufgaben Verwaltung:}\\ Die Aufgabenverwaltung mit allen Untermenüs wird angezeigt.   
\item[]\gesture{Runter Hoch}{} \textbf{Hauptmenü:}\\ Das Hauptmenü mit den vier Auswahlboxen wird angezeigt  
\item[]\gesture{Hoch Rechts}{Runter} \textbf{Analyse:}\\ Der Analyse-Dialog erscheint 
\item[]\gesture{Hoch Rechts}{Runter Links} \textbf{Verschieben:}\\ Verschieben-Modus wird aktiviert. Dieser Modus kann auch folgendermaßen aufgerufen werden:\\ Spreize (oder ziehe zusammen) mit Zeigefinger und Daumen gleichzeitg auf dem Bildschirm, ähnlich dem Zoomen bei anderen Programmen auf Android oder PDA.
\end{itemize}

\subsection*{Gesten für den \fl-Radarschirm}
Wenn das \fl-Radar auf dem Bildschirm aktiv ist, sind folgende Gesten verfügbar: 

%Gesten, die im FLARM Radar-Fenster verfügbar sind:
\begin{itemize}\itemsep1.25em
\item[]\gesture{Hoch Runter}{} \textbf{Auto Zoom:}\\  Auto Zoom skaliert das \fl-Fenster so, daß die ziele optimal sichtbar sind. Wenn Auto Zoom nicht angeklickt ist, muß das Fenster eventuell manuell nachjustiert werden. 
\item[]\gesture{Rechts Links}{} \textbf{Steigen/Höhe:}\\ Schaltet zwischen angezeigtem mittlerem Steigen oder Höhe des Zieles hin und her
\item[]\gesture{Runter Rechts}{} \textbf{Details:}\\ Hiermit wird ein Detail-Dialog  aufgerufen, in dem alle verfügbaren Daten zum Ziel dargestellt sind.  
\item[]\gesture{Hoch Runter}{} \textbf{Manueller Zoom:}\\ Ändern des Zoom-Levels zwischen 500 und 10.000m
\item[]\gesture{Links Rechts}{} \textbf{Ziele:}\\ Auswahl des nächsten oder vorherigen Zieles auf dem \fl-Radarschirm
\end{itemize}
\index{Gesten!im FLARM Radar}

%fixed 07/03/2013 


%                           fertig ...
%%%%%%%%%%%%%%%%%%%%%%
\chapter{Navigation}\label{cha:navigation}

In diesem Kapitel wird die ''moving map'' beschreiben, welche einen elementarer Bestandteil von \xc zur Navigation darstellt und heute selbstverständlich ist. Weiterhin werden hier die entsprechenden Elemente beschrieben, welche auf der Karte eingeblendet sind um die Navigation so schnell und einfach wie möglich zu machen.

\section{Elemente der Kartenanzeige}

\begin{maxipage}\centering
\includegraphics[angle=0,width=0.55\linewidth,keepaspectratio='true']{figures/fig-map.png}
\end{maxipage}

Die moving-map zeigt folgende Bestandteile:

\begin{enumerate}[itemsep=-0.25ex]
  \item Das Segelflugzeug Symbol
  \item Die Wegpunkte
  \item Die aktive Flugaufgabe (task)
  \item Den Kurs zum gewählten Wegpunkt (bearing)
     (\footnote{Der Kurs zum nächsten Wegpunkt kann auch eine {\em Route} sein, wie es im Abschnitt~\ref{sec:route} beschrieben wird.})
    \item Die Lufträume
  \item Die Landkarte und die Topologie (Straßen, Flüsse, Seen)
  \item Markierungen (selbst erstellte ''markers'')
  \item Den bisher zurückgelegten Flugweg (''trail'')
  \item Den Gleitbereich/Reichweite \footnote{Der Gleitbereich wird auch beschrieben im Abschnitt~\ref{sec:reach}.}
  \item Die aus der aktuellen Höhe erreichbaren Plätze
  \item Einen Nord-  und einen Windpfeil
\end{enumerate}

Die Karte ist in einer speziellen Projektion dargestellt, nicht in Längen- und Breitengraden. Sie kann herein- und herausgezoomt werden als auch verschoben werden. Sämtliche Navigationsberechnungen von \xc finden unter Berücksichtigung der Erdkrümmung statt.
%%%%%%%%%%%%%%%%%%%%%%%%%
\section{Segelflugzeugsymbol und Kartenausrichtung}\label{Segelflugzeugsymbol}\index{Segelflugzeugsymbol}\index{Kartenausrichtung}

Das Segelflugzeugsymbol zeigt die Position des Flugzeuges auf der Karte. Die Ausrichtung des Symboles gibt die ungefähre Lage des Segelflugzeuges in Bezug auf die Karte wieder.

Die Karte kann in Abhängigkeit des Flugmodus und der gewählten Einstellungen in der Konfiguration auf drei Arten dargestellt werden:

\begin{description}
\item[\emph{Kurs nach oben}]Kartendarstellung so, daß der Kurs nach oben zeigt.
\item[\emph{Flugrichtung oben}] Bei dieser Darstellung wird das Flugzeug immer nach oben ausgerichtet und die Karte dreht sich  mit. Hierbei wird außerdem der Nordpfeil eingeblendet, der die Richtung nach Nord angibt (\emph{true north})
\item[\emph{Norden oben}] Hier wird die Karte immer in Nordrichtung (\emph{true north}) dargestellt. Das Segelflugzeugsymbol bewegt sich gemäß seinem Kurs unter Berücksichtigung des Windes.
\item[\emph{Ziel nach oben}] Hierbei ist immer das Ziel am oberen Rande der Karte ausgerichtet.
\item[\emph{Wind oben}] Ausrichtung der Karte gemäß der Windrichtung: Der Wind kommt auf der Karte von oben nach unten
\end{description}

Über die Konfigurationseinstellungen \config{orientation} können die jeweiligen Einstellungen an die Flugsituation (Kurbeln, Vorflug, Endanflug) angepasst werden.

In den Konfigurationseinstellungen kann weiterhin eingestellt werden, ob während des Kurbelns wahlweise das Ziel oben anzuzeigen ist (\emph{Ziel nach oben}). Dies ist evtl.\ sinnvoll, wenn während des Kurbelns kurzzeitig die Orientierung zum Ausleiten verlorengegangen ist. Nach Beendigung des Kurbelns springt die Anzeige dann wieder in \emph{Norden oben} den Modus.

Wenn die Darstellung \emph{Norden oben} oder \emph{Ziel nach oben} gewählt ist, wird das Segelflugzeugsymbol in der Karte zentriert dargestellt. Normalerweise dagegen ist das Symbol ca.\ 20\% vom unteren Rand der Karte positioniert, um einen guten Überblick in Flugrichtung zu gewährleisten. Auch diese Position und eine evtl.\ Verschiebung relativ zum Ziel oder zum Kurs kann in den Konfigurationseinstellungen geändert werden.
%%%%%%%%%%%%%%%%%%%%%%%%
\section{Zoom und Kartenmaßstab}\label{zoom}\label{kartenmasstab}\index{Zoom}\index{Kartenmaßstab}\index{Maßstab}
Um den Kartenmaßstab zu ändern  (für \textsf{PC}  und  \textsf{Pocket PC} ):
\begin{enumerate}
\item Es kann eine Geste benutzt werden, Die Geste \gesture{Hoch/Runter}
     ''Hoch'' zoomt herein,  ''Runter'' zoomt heraus.
  \item Tippe auf eine leere Stelle des Displays, um die Karte zu aktivieren. Falls Du Dich in irgendeinem Menu befindest, funktioniert es nicht! 

  Hiernach die Schaltwippe des \textsf{PDA}s hoch oder runter drücken, um herein- bzw.\ herauszuzoomen.
  
     \item Als letzte Möglichkeit kann über das Menü gezoomt werden
\menulabel{\bmenut{Anzeige}{1/2}\blink\bmenut{Zoom}{herein}}
%\begin{quote}
% \smenus{Anzeige}\blink\smenut{Zoom}{Herein} bzw. %\smenus{Anzeige}\blink\smenut{Zoom}{Heraus}
%\end{quote}
\end{enumerate}

Beim Altair wird der Drehknopf zum ein- und auszoomen verwendet werden.

Der Kartenmaßstab (siehe Bild links) \index{Kartenmaßstab}\index{Maßstab} ist in der linken unteren Ecke als von links nach rechts weisender Pfeil eingeblendet.  \menulabel{\includegraphics[angle=0,width=0.3\linewidth,keepaspectratio='true']{figures/zoom.png}}

Die Zahl gibt hierbei die Gesamtbreite des Bildschirmes an.

Für \textsc{Compaq Aero} Benutzer: Wenn Ihr die \textsl{Compaq Aero Game Keys} aktiviert (im Q-Menü), können auch diese für das Hinein- und Herauszoomen benutzt werden.\index{Zoom!Maßstab}

\subsection*{Steigflug Zoom}
Es besteht die Möglichkeit, zwei Zoom Einstellung zu haben: einer, wenn der Segelflieger beim Kurbeln ist, und ein weiterer welcher für den Überlandflug bzw. Endanflugmodus gültig ist.

Dies ist die ''Steigflug Zoom'' Einstellung, welche unter \config{circlingzoom} eingestellt werden kann.\index{Zoom!Steigflug}

Als Standardmaßstab für die ''Steigflug Zoom'' Einstellung ist abhängig  von der Displaygröße ein Kartenmaßstab von 2,5 - 5 km voreingestellt.
Wenn während des Kurbelns hinein oder herausgesucht wird, so beeinflußt dies nicht den vorher im Grabe Ausflug eingestellten zu den Zoom-Maßstab. Sowie das kurbeln beendet wird, wird automatisch auf den vorher eingestellten Maßstab zurückgesetzt.
\subsection*{Auto Zoom}
Die ''Auto-Zoom'' Funktion zoomt automatisch herein, wenn man sich einem Wendepunkt nähert.\index{Zoom!Auto}
\menulabel{\includegraphics[angle=0,width=0.35\linewidth,keepaspectratio='true']{figures/zoomauto.png}}
Wenn Auto-Zoom aktiv ist, erscheint und links am Kartenrand ein Text ''Auto-Zoom''
Es kann trotzdem weiterhin hinein und heraus gesucht werden, die Auto-Zoom Funktion wird in diesem Falle automatisch auf manuell zurückgesetzt.

Um Auto-Zoom einzuschalten, benutze links angegebenes Menü
%\begin{quote}
%\smenus{Anzeige}\blink\smenut{Zoom}{Auto}
%\end{quote}
\menulabel{\bmenut{Anzeige}{1/2}\blink\bmenut{Zoom}{Auto}}
Nachdem ein Wegpunkt sich geändert hat (automatisch, über die Wegpunkt Auswahl, oder durch manuelles Umschalten auf der Karte) wird Auto-Zoom den Maßstab automatisch so wählen, daß der nächste Wegpunkt sichtbar ist. Während des Kurbelns, wenn ein Bart entdeckt wurde, wird die Karte zentriert bezüglich des Bartes dargestellt, sodaß das Flugzeugsymbol dennoch sichtbar bleibt.

%%%%%%%%%%%%%%%%%%%%%%%%%%%%%%%%%
\section{Verschieben der Karte - Verschiebe-Modus}\index{Karte!Verschieben}\index{Pan-Mode}

Der Verschiebe-Modus (''Pan-Mode'') erlaubt es dem Benutzer die Karte auf der moving map zu verschieben. Damit ist er in der Lage einen weit größeren Bereich zu sehen als auf den Kartenausschnitt dargestellt wird. Insbesondere bei der Aufgabenplanung ist dies sinnvoll und nützlich.


\menulabel{\bmenut{Anzeige}{1/2}\blink\bmenut{Verschieben}{Ein}}

%\begin{enumerate}
%\item Einschalten des Verschiebe-Modus wie folgt:
%\begin{quote}
%\smenus{Anzeige}\blink\smenut{Verschieben}{Ein}
%\end{quote}
 
Die Karte kann nun mit auf den Touchscreen gedrückten Finger verschoben werden. Auf dem \textsf{PC}  erfolgt dies mit der Maus bei gedrückt gehaltener Maustaste; beim \al werden der innere und äußere Drehknopf hierzu benutzt. Der Verschiebemodus wird verlassen mit 
%\menulabel{\bmenut{Verschieben}{Aus}} 
\button{Verschieben Aus}
%\begin{quote}
%\smenut{Verschieben}{Aus}
%\end{quote}
%\end{enumerate}

Es erscheint ein spezielles Verschiebe-Menü (s.u.) sowie ein Fadenkreuz, welches immer in der Mitte der Karte fixiert bleibt.

Der Verschiebe-Modus kann ebenso über die Geste P \gesture{P} aufgerufen werden.

\begin{center}
\includegraphics[angle=0,width=0.7\linewidth,keepaspectratio='true']{figures/pan.png}
\end{center}\index{Was ist hier?}
Mit der Auswahl von \button{Was ist hier} öffnet sich ein Menü, welches entsprechende Kartenelemente wie z.B.\ Wegpunkte und Lufträume die sich in der Nähe des Fadenkreuzes befinden, sowie die eigene Position anzeigt und zur Auswahl/Bearbeitung darstellt. Zu den jeweiligen angezeigten Elementen werden Entfernung und ggf.\ benötigte Höhe angezeigt.

Parallel dazu werden in der oberen rechten Bildschirmecke die Koordinaten des Fadenkreuzes und die dazugehörige Höhe des Geländes über MSL angegeben.  

%%%%%%%%%%%%%%%%%%%%%%%%%%%%%%%%%%%
\section{Wegpunkte} \label{sec:waypoint-schemes}
Die Darstellung und Einfärbung von Wegpunkten erfolgt je nach Situation und Art des jeweiligen Punktes.\menulabel{\bmenut{Konf}{2/3}\blink\bmenus{System}} Hauptmerkmal ist die hierbei Landbarkeit bzw. Erreichbarkeit eines Wegpunktes.  Es gibt verschiedene Sätze für die Darstellung von landbaren Wegpunkten  (Lila Punkt, Schwarz Weiß und Ampel), die unter \index{Wegpunkte!Farben}
%\begin{quote}
%\smenus{Konfig.}\blink\smenus{Konfig.}\blink\smenut{System}%{Einstellung}\blink\seite{4}
%\end{quote}
\button{Landbare Symbole} eingestellt werden können.\menulabel{\button{Kartenanzeige}\blink\button{Wegpunkte}}\index{Wegpunkte!Darstellung}

Die Symbole sind in untenstehender Tabelle aufgelistet. 
\begin{description}
   \item[Lila Punkt] Bei dieser Darstellung werden die landbaren Plätze als lila Punkt angezeigt. Wenn der Wegpunkt erreichbar ist, bekommt der lila Punkt einen grünen Kreis.  Diese Darstellung ist sehr ähnlich der in WinPilot
   \item[S/W] \config{waypointicons} Landbare Wegpunkte werden in der karte weiß/grau dargestellt, ähnlich wie oben bekommen erreichbare Plätze einen grünen Punkt, ist er durch einen Berg versperrt, wird er rot. 
   \item[Verkehrsampel] Flugplätze und Außenlandefelder werden in den Farben einer Verkehrsampel dargestellt: Grün, wenn erreichbar, Orange, wenn z.B.\ durch einen Berg versperrt und rot, wenn er überhaupt nicht erreicht werden kann.
\end{description}

\tip Je nach verwendeter Datenbank kann auch die Länge und Richtung der Landebahn mit angezeigt werden. 
%\begin{center}
%\includegraphics[angle=0,width=0.5\linewidth,keepaspectratio='true']%{figures/possiblewaypoints.png}
%\end{center}

%\begin{tabular}{c|c|cc|cc|}
%Icon set &\begin{sideways}Einfacher Wegpunkt \end{sideways}
%&\begin{sideways}Landbares Feld\end{sideways}
%&\begin{sideways}Erreichbar\end{sideways}
%&\begin{sideways}Flugplatz\end{sideways}
%&\begin{sideways}Erreichbar\end{sideways}\\
%\hline
%Purple Circle &
%\includegraphics[width=0.5cm]{icons/map_turnpoint.pdf} &
%\includegraphics[width=0.8cm]{icons/winpilot_landable.pdf} &
%\includegraphics[width=0.8cm]{icons/winpilot_reachable.pdf} &
%\colorbox{white}{\includegraphics[width=0.8cm]{icons/winpilot_landable.pdf}}
%& \includegraphics[width=0.8cm]{icons/winpilot_reachable.pdf} \\
%\hline
%B/W Icon &
%\includegraphics[width=0.5cm]{icons/map_turnpoint.pdf} &
%\includegraphics[width=0.9cm]{icons/alt_landable_field.pdf} &
%\includegraphics[width=0.9cm]{icons/alt_reachable_field.pdf} &
%\colorbox[rgb]{0.94,0.94,0.94}{\includegraphics[width=0.9cm]{icons/alt_landable_airport.pdf}}
%& \includegraphics[width=0.9cm]{icons/alt_reachable_airport.pdf} \\
%\hline
%Orange Icon &
%\includegraphics[width=0.5cm]{icons/map_turnpoint.pdf} &
%\includegraphics[width=0.9cm]{icons/alt2_landable_field.pdf} &
%\includegraphics[width=0.9cm]{icons/alt_reachable_field.pdf} &
%\colorbox{white}{\includegraphics[width=0.9cm]{icons/alt2_landable_airport.pdf}}
%& \includegraphics[width=0.9cm]{icons/alt_reachable_airport.pdf} \\
%\hline
%\end{tabular}
%

Die Wegpunkte können nach mehreren Abkürzungsregeln beschriftet werden, um auf \config{labels} der Karte nicht so viel Platz zu verschwenden und den Bildschirm  übersichtlicher darstellen zu können. 

%\menulabel{\bmenut{Konf}{2/3}\blink\bmenus{System}}
Weiterhin können je nach Sichtbarkeit bzw. Erreichbarkeit die Farben und/oder Formen geändert werden.%\menulabel{\button{Kartenanzeige}\blink\button{Wegpunkte}} 
\xc berechnet kontinierlich und unter Berücksichtigung des Windes, welche Wegpunkte sich innerhalb des Gleitpfades befinden und stellt dies entsprechend der eingestellten Optionen dar.

Die angenommene Ankunftshöhe {\em oberhalb der Sicherheitshöhe} der erreichbaren  Wegpunkte wird neben dem Wegpunkt  angezeigt. Diese Ankunftshöhe wird berechnet aus der entsprechenden Leistung des Segelflugzeuges und dem MacCready Wert. 

\config{reachpolar} Hierbei kann gewählt werden, ob die Berechnung mit einem Sicherheits-MacCready Wert, oder aber streng nach der Polare erfolgt.
%%%%%%%%%%%%%%%%%%%%%%%%%%%%%%5
\section{Aktive Aufgabe}

Die Linie der aktiven Aufgabe wird als eine grüne, gestrichelte Linie auf der Karte dargestellt.

Bei AAT-Aufgaben werden die Sektoren der Aufgabe gelb transparent hinterlegt dargestellt.
Die Start und Zielpunkte werden als Kreise dargestellt, Linien werden nur 
gezeichnet, wenn es sich hierbei um Start bzw. Ziellinien handelt.
Wendepunktsektoren werden als Segmente dargestellt (auch der DAEC-Schlüssellochsektor kann dargestellt werden).


Vom Segelflugzeugsymbol wird eine dicke schwarze Linie direkt zum nächsten aktiven Wendepunkt gezeichnet. Diese Linie stellt normalerweise den direkten Weg zum nächsten Wegpunkt dar, kann aber auch eine {\em Route} darstellen, welche um zum Beispiel Berge und/oder Lufträume herum führt. Nähere Details hierzu werden im Abschnitt Route~\ref{sec:route}  beschrieben.
\begin{center}
\begin{tabular}{c c c}
{\it Start/Ziel} & {\it Sektor} & {\it Zylinder} \\
\includegraphics[angle=0,width=0.3\linewidth,keepaspectratio='true']{figures/cut-startfinish.png} &
\includegraphics[angle=0,width=0.3\linewidth,keepaspectratio='true']{figures/cut-sector.png} &
\includegraphics[angle=0,width=0.3\linewidth,keepaspectratio='true']{figures/cut-barrel.png} \\
\end{tabular}
\end{center}
%%%%%%%%%%%%%%%%%%%%%%%%%55
\section{Gelände und Topologie}

Die folgenden topologischen Details werden auf der Karte dargestellt:
\begin{itemize}
\item Hauptstraße, dargestellt als rote Linien
\item Flüsse, dargestellt als blaue Linien
\item Große Wasserflächen (Seen), dargestellt als blaue Flächen
\item Große Städte, dargestellt als gelbe Flächen
\item Kleine Siedlungen, dargestellt als gelbe Diamanten, je nach verwendeter Datenbank wird auch nur der Name angezeigt 
\end{itemize}
Großstädte und kleine Siedlungen werden in Schrägschrift beschriftet.

Das Gelände wird gemäß der Höhe eingefärbt, optional kann es schattiert werden -  entweder gemäß der Sonneneinstrahlung oder aber entsprechend des Windeinfalles (Luv oder Lee).

Unbekanntes Gelände, oder Gelände das unterhalb der Meereshöhe liegt, wird blau dargestellt.

Die Schattierung des Geländes erhöht die Erkennbarkeit der Struktur und ist derzeit so gestaltet, daß helle Flächen eines Berges die Luvseite darstellen \config{shading}. und dunkle Flächen die Leeseite.
Die Stärke und Helligkeit der Schattierung kann konfiguriert werden 

Die Unterstützung der Schattierung für Sonneneinstrahlung in Abhängigkeit vom \menulabel{\bmenut{Anzeige}{2/2}\blink\bmenut{Gelände}{An}}Sonnenstand ist in Arbeit\dots.
Gelände- und Topologiedarstellung kann über das Menü ein und ausgeschaltet werden:
%\begin{quote}
%\smenus{Anzeige}\blink\smenus{Anzeige}\blink\smenut{Gelände}{An} \\[1em]
%\smenus{Anzeige}\blink\smenus{Anzeige}\blink\smenut{Topologie}{An}
%\end{quote}

\menulabel{\bmenut{Anzeige}{2/2}\blink\bmenut{Topologie}{An}}
\begin{center}
\begin{tabular}{c c}
Topology & Terrain \\
\includegraphics[angle=0,width=0.4\linewidth,keepaspectratio='true']{figures/cut-topo.png} &
\includegraphics[angle=0,width=0.4\linewidth,keepaspectratio='true']{figures/cut-terrain.png} \\
\end{tabular}
\end{center}

Wenn keine Geländedaten zur Verfügung stehen (oder die Gelände-Anzeige ist abgeschaltet) erscheint der Hintergrund der Karte weiß. Jedes Gelände unterhalb des Meeresspiegels wird blau dargestellt!
Flüge außerhalb des in der Datenbank bekannten Geländes werden weiß dargestellt.

\menulabel{\bmenut{Anzeige}{2/2}\blink\bmenut{Beschriftung}{Alle}}
Um die Darstellung auf der Karte übersichtlicher zu machen, können Beschriftungen und Labels der Wegpunkte ein und ausgeschaltet werden, oder aber auf die aktive Aufgabe beschränkt werden.


%\begin{quote}
%\begin{center}
%\smenus{Anzeige}\blink\smenus{Anzeige}\blink\smenut{Beschriftung}%{Keine}\blink\smenut{Beschriftung} {Aufgabe}\blink \dots
 %\\[0.75em] \smenut{Beschriftung}{Alle}\blink\bmenud{Beschriftungen}{Aufgabe\&}
%{Landeplätze}
%\end{center}
%\end{quote}

Folgende Auswahlen stehen zur Verfügung:

\jindent{\bmenud{Beschriftungen}{Aufgabe \&}{ Landeplätze}}{ Zeigt die Beschriftung der Wegpunkte  der aktiven Aufgabe und einige landbare Felder (basierend auf den Wegpunkt-Details in der geladenen Wegpunkt Datei).  Andere Wegpunkte werden angezeigt, aber nicht beschriftet. }
\jindent{\bmenut{Beschriftung}{Aufgabe}}{ Es werden nur die in der aktiven Aufgabe befindlichen Wegpunkte beschriftet. }
\jindent{\bmenut{Beschriftung}{Alle}}{ Anzeige aller Beschriftungen für alle Wegpunkte. }

Die Beschriftung und das Aussehen der Beschriftung ist über das Menü \config{labels} konfigurierbar.
%%%%%%%%%%%%%%%%%%%%%%55
\section{Flugspur (trail)}\label{sec:trail}

Wenn gewünscht, kann die bisher geflogen Flugstrecke auf der Karte dargestellt werden. Die Farbe und die Breite dieser Spur hängt von der Höhe oder von Vario-Wert ab und kann entsprechend gewählt werden. \config{snailtype}

\begin{center}
\includegraphics[angle=0,width=0.8\linewidth,keepaspectratio='true']{figures/snail.pdf}
\end{center}

Wenn ein VEGA-Variometer angeschlossen ist und die Netto-Luftmassenbewegung anzeigt, dann können die Farben und Dicken dieser Spur auch die Netto-Luftmassenbewegung darstellen.

Die Flugspur kann wie folgt ausgewählt werden: \button{Kurz} (zeigt ungefähr 10 min), \menulabel{\bmenut{Anzeige}{2/2}\blink\bmenut{Spur}{Kurz}} \button{Lang}, (zeigt ca. eine Stunde), \button{Voll} (zeigt die gesamte bisherige Flugstrecke) und aus. Die Einstellung kann jederzeit über das Menü wie folgt vorgenommen werden:

%\begin{quote}
%\smenus{Anzeige}\blink\smenus{Anzeige}\blink\smenut{Flugspur}%{Lang}\blink\smenut{Flugspur}{ Kurz}
%\end{quote}

Alternativ kann die Einstellung auch im Menü  \config{snailtrail} als Voreinstellung eingestellt werden.
Beachtet, daß wegen der Übersichtlichkeit der Kartendarstellung beim Kurbeln die Flugspur grundsätzlich kurz ist.
%%%%%%%%%%%%%%%%%%%%%%%%%%%%%%
\subsection*{Spurdrift (Windversatz-Kompensation)}
Um den Windversatz beim Kurbeln darzustellen und so das Zentrieren zu unterstützen, kann die Spurdrift-Funktion bei der Kartendarstellung eingeschaltet werden.

 \textcolor{blue}{ Die Spurdrift bzw. Windversatz-Kompensation hat \textbf{nichts}  mit der Funktion ''geschätzte Windabdrift'' \achtung aus der Endanflugskonfiguration nach Kap.~\ref{sec:final-glide} zu tun!}
 
In diesem Fall wird die Flugspur relativ zum Wind aufgezeichnet und nicht mehr absolut über Grund. Das bedeutet, daß  \xc den erflogenen Bart während des Fluges mit dem ermittelten Wind versetzt ganz ähnlich wie auch das Flugzeug versetzt wird. Hiermit wird eine bessere Wiederauffindbarkeit und  Anzeige der Bewegung des Flugzeuges relativ zum Bart ermöglicht.

Zur Veranschaulichung ist nachfolgend ein Beispiel beigefügt. 

Beachte, daß, wenn die Spurdriftfunktion aktiviert ist (rechtes Bild), sich das Flugzeug mehr in einem parallelversetzten Kreis  bewegt, anstelle einer gekrümmten, langgezogenen Spirale (linkes Bild).

\begin{center}
\includegraphics[angle=0,width=0.8\linewidth,keepaspectratio='true']{figures/traildrift.png}
\end{center}

\config{traildrift} Die Funktion \button{Spurdrift} kann über die Konfiguration Einstellung ein und ausgeschaltet werden.  Die Kompensation ist ausschließlich in Kurbelmodus aktiv; im normalen Geradeausflug wird eine Kompensation über das Wind-Menü ermöglicht.
\menulabel{\bmenus{Konfig}\blink\bmenus{Wind}}
Die Anzeige des Windversatzes zeigt sehr schön, wenn Bärte in der Höhe stark versetzt werden - zum Beispiel durch Windscherungen.

Die Breite der Flugspur  kann optional anhand des Vario-Wertes gesetzt werden (Steigen, Saufen) \config{trailscaled}.
%%%%%%%%%%%%%%%%%%%%%%%5
\section{Marker}

Marker werden als kleine blaue Flaggen auf der Karte dargestellt. 
Die Marker können entweder automatisch oder aber durch Knopfdruck gesetzt werden.

\menulabel{\bmenut{Nav}{2/2}\blink\bmenut{Marke}{Setzen}}Ein Beispiel für die Nutzung von automatischem Setzen von Markern ist z.B. die 
Markierung, wann in den Kurbelmodus gewechselt wurde. Hiermit kann sehr einfach dargestellt werden, wann und wo gekurbelt wurde.\index{Marker}
\menulabel{\includegraphics[angle=0,width=0.25cm,keepaspectratio='true']{icons/map_flag.pdf}}

Marker werden nicht gespeichert, wenn \xc beendet wird, die Koordinaten aller Marker werden jedoch im File \verb|xcsoar-marks.txt| gespeichert.

%%%%%%%%%%%%%%%%%%%%%%%%%%%%%%
\section{Anzeige des Gleitbereiches}\label{sec:reach}

Der aktuelle Gleitbereich \button{Erreichbar Anzeige}kann wahlweise in der Karte auf zwei Weisen dargestellt werden:\index{Gleitbereich}\index{Gleitbereich!Darstellung}\index{Reichweite}
\menulabel{\bmenut{Konfig.}{2/3}\blink\bmenus{System}}
\begin{itemize}
    \item als schwarz-weiß gestrichelte Linie, 
    \item als schattierter Bereich.
\end{itemize}
\menulabel{\button{Endanflug}\blink\button{Routenplaner}}
Der Gleitbereich schließt alle erreichbaren Gelände ein und markiert - entsprechend der Höhe des Geländes - auch Bereiche, welche evtl.\ umflogen werden müssen.  Dies kann sehr nützlich sein, wenn in niedrigen Höhen nach Steigen gesucht wird, oder man sich in bergigem Gelände aufhält.

Die Berechnung der Reichweite \button{Reichweite Modus} kann in zwei Detailstufen \config{turningreach} wie folgt konfiguriert werden:

\begin{description}
\item[Aus] Wenn ausgeschaltet, wird die Reichweite nicht berechnet.
\begin{center}
\includegraphics[angle=0,width=0.8\linewidth,keepaspectratio='true']{figures/reach1.png}
\end{center}
\item[Direkt] Wenn eingeschaltet, erfolgt die Berechnung der Reichweite geradeaus vom Flugzeug zum Ziel führenden Pfad. Es werden weder Lufträume noch sonstige Hindernisse in die Kalkulation einbezogen.\index{Gleitbereich!Berechnung}
 \item[mit Umweg] Wenn dies eingeschaltet ist, wird die Reichweite unter Berücksichtigung von Hindernissen und Umwegen berechnet (Gelände und/oder Lufträume, s.o.)
\begin{center}
\includegraphics[angle=0,width=0.8\linewidth,keepaspectratio='true']{figures/reach2.png}
\end{center}
\end{description}

Der Weg des Endanfluges wird permanent auf evtl.\ Kollisionen mit einer Geländeerhebung geprüft -- sollte sich eine Kollisionsgefahr ergeben, so wird der vorausgesagte Kollisionspunkt\index{Geländekollision} mit dem Gelände als kleines rotes Kreuz auf der Karte dargestellt.

Wenn der \button{Reichweite Modus} aktiviert ist, dann wird dieser bei Abbruch einer Aufgabe  automatisch benutzt, um landbare Plätze (in allen Richtungen) auf der Karte darzustellen (Im Alternativen - Modus).


\textcolor{blue}{Beachte, daß bei aktiven Aufgaben die ''Reichweite'' Funktion   \textbf{nicht} in den InfoBoxen aktiv ist, hier wird die Funktion \textbf{nicht} verwendet!}\warning 


\textcolor{blue}{Sie ist ausschließlich bei der Darstellung auf der Karte und bei abgebrochenen Aufgaben aktiv!}

Die Leistungsdaten des Segelflugzeuges (Polare) bzw.\ die MacCready Einstellung, die für diese Berechnung verwendet werden sollen, können unter %\ref{sec:final-route}
\button{Erreichbare Polare} eingestellt werden. \config{reachpolar}
\begin{description}
\item[Aufgabe] Es wird der MC benutzt, welcher in der Aufgabe gesetzt ist.
\item[Sicherheits  MC] Hier wird ein voreingestellter, i.d.R.\ niedriger MC Wert gewählt, den der Pilot vorher festsetzt. Normalerweise etwas schlechter als das beste Gleiten. (s.~\ref{safety-MC}) 

Die ''Größe'' der Sicherheit bei der Reichweitenkalkulation ist dementsprechend   die Differenz aus bestem Gleiten und dem Gleiten entsprechend der Geschwindigkeit des Sicherheits MC-Wertes.
\end{description}
%%%%%%%%%%%%%%%%%%%%%%%%%%%%%%%%%%%%%%%%%5
\section{Status-Fenster}\label{sec:aircr-stat-dial}\index{Status}

Das Status-Fenster ist ein reines Informationsfenster und hat fünf Unterfenster. \button{Flug}, \button{System}, \button{Aufgabe}, \button{Regeln} und \button{Zeiten} Es 
\menulabel{\bmenut{Info}{2/3}\blink\bmenus{Status}} und dient vor allem zur Beobachtung von Zeiten, Koordinaten, Regeln etc.\  
  
\achtung Beachte, daß die hier aufgeführten Daten nicht aktualisiert werden, während dies Fenster geöffnet ist. erst nach Schließen und wieder aufrufen sind die Daten wieder auf dem laufenden! 


%\begin{center}
%\includegraphics[angle=0,width=0.5\linewidth,keepaspectratio='true']{figures/XCS64-StatusWindow.png}
%\end{center}

Mit dem Button \button{Flug} kann z.B. schnell Deine Position an andere 
weitergegeben werden, hier wird Deine Position angezeigt, Deine Peilung und Entfernung auf den nächsten aktiven Wegpunkt etc. 


Derzeit werden als ''nächste Wegpunkte'' ausschließlich Punkte aus der Wegpunkt-Datenbank genommen, in zukünftigen Versionen ist geplant, evtl.\  auch Dörfer und Städte aufzunehmen.\menulabel{\includegraphics[angle=0,width=0.8\linewidth,keepaspectratio='true']{figures/XCS64-StatusWindow.png}}

Unter \button{Regeln} kannst Du z.B.\ erkennen, ob Dein Abflug gültig war, mit welcher Geschwindigkeit er erfolgte etc.\ 

\button{Zeiten} gibt allerlei Zeiten wieder, z.B.\ die UTC-Zeit, den Sonnenauf- und Untergang (Danke, Helmut\dots), Start- Lande- und Flugzeit. 

\button{System} informiert u.a.\ über die Versorgungsspannung des Geräte, wie der GPS-Empfang ist, ob \fl und Logger angeschlossen sind. 

In \button{Aufgabe} werden Daten und Zeiten zur aktuellen Aufgabe gelistet: aktueller Schnitt, Distanz, bei AAT: zugeteilte und verbleibende Zeit etc\dots 


%%%%%%%%%%%%%%%%%%5
\section{Routen}\label{sec:route}

\xc kann Flugwege planen, die um Lufträume und z.B.\ Gebirge herumführen, sowohl in \menulabel{\bmenut{Konfig.}{2/3}\blink\,\bmenus{System}}
horizontaler als auch in vertikaler Richtung. Solch ein Flugweg wird hier als ''Route'' bezeichnet kann mit dem \button{Routenplaner} konfiguriert werden..

\menulabel{\blink\,\button{Endanflugrechner}} 
Die Höhe des Zieles ist hierbei die Ankunftshöhe des Zieles, kann aber auch größer sein, falls die Wegführung durch die in der Aufgabe deklarierten Wegpunkte dies erfordern. Die Routen-Funktion ist in folgenden Flugmodi verfügbar:

Goto Mode, deklarierte (aktive) Aufgaben und Abbruch modus.

\begin{center}
\includegraphics[angle=0,width=1.0\linewidth,keepaspectratio='true']{figures/route3.png}
\end{center}

Bei der Routenberechnung wird die Strecke unter Berücksichtgung der Flugzeugpolaren zeitoptimiert berechnet. Standardmäßig ist die Routenfunktion ausgeschaltet. Sie kann aktiviert werden unter  \config{routemode} Berücksichtigung von Lufträumen, Gelände, Wolkenuntergrenze.

Geländekollisionen in vertikaler Richtung werden verhindert durch Einstellung der Gelände-Sicherheitshöhe. \config{safetyterrain} Hierbei ist keine zusätzliche Höhe als Puffer vorgesehen.

Es kann sein, daß manche Routen in einer Ankunft am Ziel höher als erwartet enden. Dies kann zum Beispiel geschehen, wenn der Zielpunkt exakt hinter einem relativ hohen Berg liegt. Zur Vermeidung von horizontalem Einfliegen in einen Luftraum wird mit ein Sicherheitsabstand von 250 m ohne weitere vertikaler Reserven benutzt.
Richtig deklarierte Routen werden im Flugverlauf über oder unterhalb Lufträumen und auf jeden Fall um Lufträume herum führen.

\warning
Wenn der MacCready-Wert größer als 0, dann ist Steigend während einer Route gemäß der MC-Theorie ''erlaubt''. \config{routeceiling}  Die maximale Höhe des Steigens  ist auf  500 m unterhalb der Wolkenuntergrenze limitiert, kann jedoch in der Konfiguration abgeschaltet werden.

Ein Steigen über die maximal vorgeschriebene Höhe des Start- und Zielpunktes (siehe oben) wird ''bestraft'' mit einer entsprechend schlechteren Steigrate als der aktuell eingestellte MacCready-Wert. 


Einige Einschränkungen und Limitierung bezüglich des Routenplanungssystems sind hier aufgelistet:

\begin{itemize}
\item Wenn Kurbeln notwendig ist, um das Ziel zu erreichen,  wird angenommen, daß das Steigen zu Beginn der Route ist. 

Hintergrund: eine Einstellung von  MacCready gleich 0 bedeutet, daß (streng nach der MC-Theorie) kein Steigen erwartet wird!  Andernfalls würde die Rechenalgorythmen in  die Irre geführt werden!   
\item Strecken mit Steigen im Geradeausflug werden auf einer konstanten Höhe angenommen, gleichbedeutend mit vielen kleinen Steigen verteilt entlang der Strecke
\item Kurven und Schlenker zwischen den einzelnen Streckensegmenten mit Ablagen größer 90$^\circ$ sind erlaubt
\item Fehler des Rechenalgorithmus innerhalb der Route können dazu führen, daß der Anflug in einem direkten Anflug auf das Ziel zurückgeführt wird.
\end{itemize}
%finished für v6.5 date 09/02/2013 OH

%%%%%%%%%%%%%%%%%%%%%%
\input{ch04_xc_tasks.tex}
%%%%%%%%%%%%%%%%%%%%%%
\chapter{Der Segelflugrechner}\label{cha:glide}
In diesem Kapitel wird beschrieben, wie das Programm intern arbeitet und es ist wirklich empfohlen,
sich hiermit zu beschäftigen. Nur dann können einige der Anzeigen bzw.\ Berechnungen von textsf{XCSoar's}
wirklich richtig gedeutet werden.
Ein Mindestmaß an Überlandflugerfahrung  wird voraus gesetzt, insbesondere sollte die MC-Theorie
halbwegs verstanden sein.

Dennoch ist es problemlos möglich mit dem Programm 'durch die Gegend' zu fliegen und sich von ihm in
vielen Dingen unterstützen zu lassen.

\section{Die verschiedenen Flugmodi}\index{Flugmodi!Unterschiede}
Je nach Flugmodus berechnet \textsf{XCSoar} die Daten teils unterschiedlich und gibt diese in frei
definierbaren Datenfenstern, den sog.\ Infoboxen wieder. Als Flugmodus werden hier Kurbeln (thermalling),
einfacher Vorflug (cruise) oder Endanflug (final glide), einer besondern Art des Vorfluges,
nämlich auf den letzten Wegpunkt also den Zielpunkt einer Aufgabe hin,  unterschieden.

\textsf{XCSoar} kann automatisch zwischen Kurbel-  und Vorflugmodus unterscheiden. Die Umschaltung
der verschiedenen Berechnungsmethoden hierbei geschieht automatisch.  Nach ca.\ 30 Sekunden Kurbelzeit
schaltet die Software automatisch auf den Kurbelmodus. Nach 30 Sekunden Geradeausflug schaltet das
Programm auf den Vorflugmodus. Es besteht auch die Möglichkeit diese Umschaltung manuell zu erzwingen.\todonum{Wie?}

Ist der letzte Wegpunkt (Zielpunkt) aktiv, wird in den Endanflugmodus geschaltet.
Der Endanflugmodus wird ebenfalls aktiviert, wenn eine Aufgabe abgebrochen wird, so daß hier immer
auf die erreichbaren Punkt gerechnet wird.

Ein kleines Symbol in der rechten unteren Ecke zeigt den jeweiligen Flugmodus an:

\begin{tabular}{c c c c}%{c c c c}
\includegraphics[angle=0,width=0.75cm,keepaspectratio='true']{figures/mode_cruise.png} &
\includegraphics[angle=0,width=0.75cm,keepaspectratio='true']{figures/mode_climb.png} &
\includegraphics[angle=0,width=0.75cm,keepaspectratio='true']{figures/mode_finalglide.png} &
\includegraphics[angle=0,width=0.75cm,keepaspectratio='true']{figures/mode_abbruch.png}\\
(a) & (b) & (c) & (d)
\end{tabular}

\begin{description}
\item[Vorflug (a)]   Das Flugzeug ist nicht am Kurbeln  und es ist entweder keine Aufgabe aktiv, oder
der aktive Wegpunkt ist nicht der Aufgaben-Zielpunkt
\item[Kurbeln (b)]  Das Flugzeug ist derzeit am Kurbeln (auch wenn es dabei nicht steigt oder gar sinkt!)
\item[Endanflug (c)]  Das Flugzeug ist nicht am Kurbeln und der aktive Wegpunkt ist der Zielpunkt der Aufgabe.
\item[Abbruch (d)]  Dieser manuelle Aufgabenabbruch zeigt an, daß sich das Programm in einem  Modus befindet, in dem ausschließlich Außenlandemöglichkeiten auf landbaren Plätzen, Flugplätzen und markierten Feldern  dargestellt und berechnet werden, die sich in den entsprechenden Datenbanken befinden..(see Section~\ref{sec:abort-resume-task})
\end{description}

Die jeweiligen Berechnungen von \textsf{XCSoar} sind direkt abhängig vom jeweiligen Flugmodus. Die
Anzeige der Daten, insbesondere die der Infoboxen wechseln komplett und frei definierbar von Modus zu
Modus. Auch der Zoom-Modus kann beim Umschalten der Modi automatisch geändert werden.

Zusätzlich zu diesen Anzeige-Modi können mehrere Sätze an Infoboxen frei benannt und belegt werden,
welche während des Fluges unabhängig vom jeweiligen Flugmodus aufgerufen werden können.

Der Endanflugmodus ersetzt den normalen Vorflugmodus sobald der Flieger oberhalb der notwendigen
Endanflughöhe ist. Die hierzu benötigte Höhe ist maßgeblich abhängig vom eingestellten MC-Wert, aber
auch die Höhe über Grund und Sicherheitshöhen werden berücksichtigt.

Wird während eines Endanfluges mit dem Kurbeln begonnen, wird\textsf{XCSoar} in den Kurbelmodus umschalten und
sobald der Endanflug wieder fortgeführt wird, zurück in den Endanflug-Modus geschaltet.

\section{MacCready setting}

Der  MacCready Wert kann auf verschiedene Weisen eingegeben werden:
\begin{itemize}
\item Von den Menupunkten
\begin{quote}
\bmenu{Konfig.}\blink\bmenu{MC $+$}

\bmenu{Konfig.}\blink\bmenu{MC $-$}
\end{quote}
\item Bei  Touchscreen-Geräten (Android, PDA), kann auch die entsprechende Infobox direkt angewählt
werden und mit den Hoch/Runter Knöpfen (Android) bzw. der Schaltwippe (PDA)der Wert geändert werden.
\item Ist \textsf{XCSoar} an ein intelligentes Vario wie z.B.\ das Vega angeschlossen, dann
ändert die MC Einstellung im Vario auch die in \textsf{XCSoar}.
\end{itemize}

Als zusätzliche Option ist ein Automatik-MC-Mode verfügbar und in Kap. ~\ref{sec:auto-maccready} beschrieben.

\section{Flugzeugpolare}

Eine große Anzahl an Polaren ist in \textsf{XCSoar} bereits vorhanden und kann
somit sofort  ausgewählt werden. Wenn ein Flugzeugtyp nicht vorhanden sein
sollte, so können z.B. Näherungswerte der vorhanden  Flugzeuge benutzt werden.

In diesem Falle ist es ratsam, sich an den im Internet auf bekannten Seiten
vorhandenen Polaren zu beschaffen und entsprechend anzupassen. Siehe hierzu
auch \config{polar}.

Die Polare kann während des Fluges angepaßt werden, um z.B. Mückenbefall einzustellen
und so die Gleitleistung entsprechend zu verschlechtern oder aber entsprechend Wasserballast entsprechend zu verbessern.

Wie bekannt und von jederman sicher schon leidvoll erfahren, bewirkt
Mückenbefall und/oder  Regen eine teils drastische  Verschlechterung der
Gleitleistung des Flugzeuges. Bislang sind Rechner bzw. Programme noch nicht in
der Lage, diese automatisch zu erkennen, sodaß es allein in der Macht und
Erfahrung des Piloten liegt, eine Einschätzung der Verschlechterung vorzunehmen
und diese als Parameter (in Form eines Prozentwertes) dem Programm mitzuteilen.

Hierbei entsprechen 100\% Mücken Faktor einem sauberen Profil (linke Seite), 50\% bedeuten eine Verdopplung der
Sinkrate des Flugzeuges - verglichen mit dem sauberen Profil.

\begin{center}
\includegraphics[angle=0,width=\linewidth,keepaspectratio='true']{figures/cut-clean-dirty-polar.png}
\end{center}

Es ist bekannt, daß Mückenbelastung die Flugleistungen eklatant verschlechtern können, mir ist ein Fall bekannt, in dem die
Verschmutzung so drastisch war, daß sogar eine bekanntermaßen "harmlose" ASK21 bei der Landung einfach durchsackte und
dem Piloten einige Wochen im Krankenhaus verschaffte\dots Ein anderer Pilot berichtete während es gleichen Fluges davon, daß seine
Mückenputzer nicht ausgefahren werden konnten, da diese an den Massen der Läuse hängen blieben\dots

Dies alles vorausgesetzt, ergeben sich praktisch resultierende Verschlechterungen der Gleitleistung und somit Verschiebung
der Polare der Polare für den "worst case" von 30 - 70\%.

Es sind sicherlich einige "Probeflüge" notwendig, um die entsprechende Verschlechterung durch Mückenbelastung einschätzen
und entsprechend al Prozentangabe in den Rechner eingeben zu können, die vor allem vor dem Hintergrund,
daß alle Flugzeugprofile sich hierin unterscheiden.

Der Wert für den Ballast  wird in \xc eingestellt als eine prozentuale Einstellung des \textsl{maximal möglichen Ballastes}.
\index{Ballast-Einstellung}\index{Wasserballast-Einstellung}
Abhängig von der Konstruktion des Segelflugzeuges bzw.\ insbesondere seiner Polaren kann diese Einstellung  durch das
jeweilige Gewicht des Piloten in großem Rahmen variieren.

So kann z.B. wenn ohne Wasserballast geflogen wird, die Einstellung dennoch auf z.B.\ 10\% eingestellt werden,
um dem wahren Abfluggewicht des Flugzeuges näher zu kommen.


%{\it DIAGRAM SHOWING GLIDE POLAR, 0\% BALLAST AND 100\% BALLAST}

Die zur Berechnung herangezogene Polare und das eingegebene Gesamtgewicht kann
u.a. im Analyse-Dialog betrachtet werden, welches hier später behandelt wird.

\section{Flug Einstellungen (dialog)}\label{sec:basic-sett-dial}
In diesem Menü können die folgenden Einstellungen vorgenommen werden:


\menulabel{\bmenut{Konfig.}{1/2}\blink\bmenut{Flug}{Einstellungen}}
\begin{itemize}
\item Wasserballast
\item Flächenbelastung
\item Mückenbelastung
\item QNH
\item max. Temperatur
\end{itemize}


%\begin{center}
%\includegraphics[angle=0,width=0.45\linewidth,keepaspectratio='true']{figures/XCS64-Flug-Einstellungen.png}
%\end{center}

Mithilfe der Mückeneinstellung ('clean') kann während des Fluges nachjustiert werden, wie sich die Polare
durch Mückenbefall verschlechtert.  Ein Setzen auf 100\% 'clean' bedeutet, daß die Software annimmt,
das Flugzeug bzw. das Profil ist top sauber. \tip

Eine Einstellung von 50\%  wird die Polare um 50\% verschlechtern, was in etwa der Verdopplung
der Sinkrate bei der entsprechenden Geschwindigkeit entspricht.

Die Ballast Einstellungen werden benutzt, um  die Polare entsprechend der Flächenbelastung
verschieben zu können. Der Ballast wird in Litern angegeben, und sollte \emph{vor} dem Flug
korrekt angegeben werden.
\marginpar{\includegraphics[angle=0,width=\linewidth,keepaspectratio='true']{figures/XCS64-Flug-Einstellungen.png}}

Dieser Dialog kann vor und während des Fluges benutzt werden, um das QNH einzustellen.
Die Software benutzt diese Werte um während des Fluges die entsprechenden Flugflächen
zu errechnen. Wenn mit einem intelligenten unterstützten Vario bzw. Höhenmesser verbunden,
kann wird die Höhe in diesem Dialog eingestellt werden. Das macht es sehr einfach,
das QNH einzustellen, wenn die Höhe des Flugplatzes bekannt ist.


Die maximale vorhergesagte Boden-Temperatur kann hier eingegeben werden und wird dann für die
Konvektionsvorhersagen verwendet. (siehe Abschnitt~\ref{sec:convection-forecast})
Hieraus werden dann Höhe der Konvektionsschicht und Basis der Wolken ermittelt.

\tip Es ist möglich, \textsf{XCSoar} so zu konfigurieren, das beim Start als erstes dieser Dialog erscheint\todonum{..und wie ?..}

Beim Systemstrat wird, nachdem ein GPS als solches erkannt ist und über einen Drucksensor verfügt,
das QNH automatisch ermittelt.  Diese Korrektur setzt das QNH so, daß die Druckhöhe mit der
Geländehöhe übereinstimmt.

Das QNH wird nur dann aktualisiert, wenn das Flugzeug für mehr als 10 Sekunden auf dem Boden
steht! (um den Abgleich mit der Geländedatei zu ermöglichen.)

Sollte \textsf{XCSoar} im Fluge neu gestartet werden, so ist das QNH nicht mehr aktuell! \index{QNH!Ermittlung!Korrektur}
Die Aktualisierung kann weiterhin ausschließlich dann erfolgen, wenn Geländedatei aktuell und in
Ordnung ist - darin sind ja die Geländehöhen MSL enthalten...

\section{Sollfahrt}\index{Sollfahrt}

Wenn das Programm mit einem intelligenten Vario betrieben wird, welches die angezeigte
Fluggeschwindigkeit (IAS) berechnet, dann werden am rechten Rand der Anzeige Geschwindigkeitspfeile
angezeigt. Wenn das Flugzeug schneller als die optimale Geschwindigkeit fliegt, dann zeigen grüne Pfeile
nach oben (ziehen !), wird zu langsam geflogen, zeigen rote Pfeile nach unten. Wird mit optimaler
Geschwindigkeit geflogen, so erscheinen keinerlei Pfeile.

%{\it DIAGRAM SHOWING SPEED COMMAND CHEVRONS}

Abhängig von der jeweiligen Konfiguration erscheinen die Geschwindigkeitspfeile entweder
am rechten Rande des Bildschirmes oder aber in der Vario-Anzeige.

\section{Sollfahrt - Empfohlene Geschwindigkeit}\index{Sollfahrt!MC oder Delphinstil}

\textsf{XCSoar} berechnet kontinuierlich zwei Sollfahrten gemäß der Polare des Flugzeuges und der
MC-Theorie:

\begin{description}
\item[MacCready Geschwindigkeit]  Diese Geschwindigkeit ist die optimale Vorfluggeschwindigkeit nach der
MC Theorie in ruhender Luft, unter Berücksichtigung des Windes, wenn im Endanflug.
\item[Delphin-Stil- Geschwindigkeit]  Dies ist die optimale Vorfluggeschwindigkeit unter
    Berücksichtigung von aufsteigender und fallender Luft sowie unter Berücksichtigung des Windes,
    wenn im Endanflug.
\end{description}

Diese Geschwindigkeiten haben nichts mit der Geschwindigkeit beim Kurbeln zu tun!
Der Pilot kann diemaximal zulässige Manövergeschwindigkeit im Konfigurationsbereich
eingeben, welche die empfohlenen Geschwindigkeiten anhand der MC-Werte in einen
realistischen Bereich eingrenzen.

Je nach Pilot gibt es verschiedene Vorlieben:

manche bevorzugen dies sogenannte "block-speed", d.h., sie fliegen zwischen zwei Bärten mit konstanter
Geschwindigkeit\index{Block-Speed}, streng nach der MC-Theorie; andere dagegen
variieren auch zwischen zwei Bärten, gemäß des "Delphin"-Flugstieles\index{Deplhin-Stil}, wobei die
Geschwindigkeit permanent angepasst wird.


\begin{maxipage}
\begin{center}
\includegraphics[angle=0,width=0.8\linewidth,keepaspectratio='true']{figures/figure_speed_to_fly.png}
\end{center}
\end{maxipage}

In der Konfigurationseinstellung "Block-Speed-to-fly" (~\ref{sec:final-glide})
kann eingestellt werden, ob der Delphinstil, oder der "pure" MC-Wert zur
Berechnung herangezogen werden soll.

Die \infobox '' V opt'' zeigt ebendiese Geschwindigkeit, je nach
ausgewählter Methode an.\index{V opt}

Falls an das VEGA Vario angeschlossen, werden die Töne des Varios im Sollfahrt-Modus
genau anhand dieses Wertes ausgegeben.


\section{Sollfahrt  mit Risikofaktor}\label{sec:speed-fly-with}
Viele Piloten verringern den MC-Wert und somit die Sollfahrt wenn Sie während der Aufgabe niedriger werden.
 Die interne Sollfahrtberechnung kann mit einem Risikofaktor belegt werden, welcher je nach erreichter Flughöhe in beiden Fällen
 (Delphin oder Block-Speed) den MC Wert automatisch verringert, je mehr man sich dem Boden nähert.

Die Theorie hierfür ist grob geschrieben in einem Aufsatz von John
  Cochrane, ``MacCready Theory with Uncertain Lift and Limited
  Altitude'' {\em Technical Soaring} 23 (3) (July 1999) 88-96.
  \url{http://faculty.chicagogsb.edu/john.cochrane/research/Papers/newmcred.pdf}

Ein konfigurierbarer Parameter, $\gamma$ (`der STF Risiko Faktor '\footnote{STF  = Speed To Fly $\Rightarrow$  Sollfahrt},
in den Konfigurationseinstellungen auf der Seite "Endanflugrechner") erlaubt die Kontrolle, wie der Risiko-MC-Wert berechnet wird.

Der $\gamma$ Faktor bestimmt die Beziehung aus dem aktuell eingestellten MC-Wert als Funktion der Höhenbeziehung.
Der Höhenbeziehung ist bestimmt als das Verhältnis zwischen aktueller Höhe über Grund ($h$) und Basishöhe über Grund ($h_{top}$).

Somit bestimmt $\gamma$ einen Faktor, bestehend aus maximal zu erreichbaren Höhe (Basis minus Geländehöhe)
und der Höhe, bei der man beginnt, über eine evtl. Außenlandung bzw. Abbruch der Aufgabe nachzudenken.

Demgemäß bewirkt ein niedriger Faktor $\gamma$ eine höhere Außenlandetoleranz gegenüber größeren Werten von $\gamma$. Der als Default
gesetzte Wert v0n $\gamma=0$ bewirkt keinerlei  Anpassung, der Risiko MC-Wert ist identisch dem eingedrehten MC-Wert.

Für $\gamma=1$ wird der Risiko-MC-Wert linear mit $h/h_{top}$ angepasst.
Alle Werte dazwischen werden leicht variiert angepaßt, sodaß der der Risisko-MC nur dann auch klein wird, wenn man wirklich niedrig ist.

Kleine Werte von $\gamma$ werden bevorzugt, wenn die Piloten mit geringer werdenden Höhe nicht langsamer werden möchten und so ein
Außenlanderisiko bewußt in Kauf nehmen, höhere Werte von $\gamma$ werden von vorsichtigen Piloten bevorzugt,
resultierten jedoch (aufgrund niedrigerer MC-Werte) in einer niedrigeren Schnittgeschwindigkeit.

Ein Wert von  $\gamma=0.3$ ist empfohlen.

\begin{center}
\includegraphics[angle=0,width=\linewidth,keepaspectratio='true']{figures/riskmc.png}
\end{center}

\section{Sicherheitshöhen}\label{sec:safety-heights}

In \xc werden drei Sicherheitshöhen deklariert und benutzt, welche ein abgestuftes Maß von Sicherheit in die Berechnungen einfließen lassen.

Die Sicherheitshöhen sind:
\begin{description}
\item[Ankunftshöhe]
Dies ist die Ankunftshöhe über Grund über dem aktiven Wegpunkt, bei welchem das Flugzeug  eine sichere Platzrunde
zur Landung erreichen wird, zusätzlich einer gewissen Reserve.

Dieser Wert wird im Endanflug benutzt und ist bestimmend für die Anzeige und das Erreichen von alternativen landbaren Feldern.


\item[Überlandfreiheit "Terrain  clearance"]
Dies ist die Höhe über Grund, unterhalb derer jeder berechnete Gleitpfad zu nicht
angepassten Höhen des Weiterfluges führen wird.

Die Überlandfreiheit beeinflußt die Reichweitenanzeige  und, falls der Endanflug zu irgendeinem
Punkt unterhalb der Höhe über Grund führen sollte, erscheint ein rotes Kreuz als Zeichen für den "Aufschlagpunkt".

Sollte das Geländemodell ungültig sein oder außerhalb der Reichweite, dann wird dies Geländewarnmerkmal deaktiviert.

\item[Aufbruchshöhe "Break-off height"]
Dies ist die Höhe über Grund, bei der sich der Pilot Gedanken machen sollte, den Überlandflug abzubrechen
bzw.\ sich mit einen Außenlandeacker beschäftigen sollte.

Derzeit wird diese Höhe noch nicht innerhalb der Berechnungen benutzt, aber für spätere Zwecke schon eingeführt.
\end{description}

\begin{maxipage}
\begin{center}
\includegraphics[angle=0,width=\linewidth,keepaspectratio='true']{figures/figure_terrain.png}
\end{center}
\end{maxipage}

\warning
Diese Höhen können auf  Null gesetzt werden, es ist jedoch grundsätzlich empfohlen, mit einer gewissen Reserve als Sicherheit
zu fliegen, da kein bisher auf dem Markt verfügbarer Rechner oder aber Software in die Zukunft bzw.\  in das Wetter schauen kann.
(Ich denke mal, das wird auf absehbare Zeit auch so bleiben -\textsf{OH})

\textsf{XCSoar} benutzt zur Berechnung der Ankunftshöhen die MSL eines jeden Punktes aus dem Wegpunkt-File, oder aber,
falls hierin nichts eindeutiges drin vermerkt ist, die Daten aus dem Höhenmodell des Kartenmaterials (also des XCM-Files)

\textbf{Die angezeigte Ankunftshöhe  des nächstliegenden, landbaren ausgewählten Wegpunkte werden standardmäßig
mit MC=0  und mit dem besten Gleiten aus der eingestellten Polare errechnet, der Wind wird hierbei mit berücksichtigt.
Ein Sicherheits-MC-Wert kann und sollte bestimmt und in den Konfigurationseinstellungen eingestellt werden, wie im Kapitel
oben beschrieben.}



Landbare Felder werden nur als erreichbar angezeigt, wenn die Ankunftshöhe oberhalb  der Ankunftssicherheitshöhe liegt und der
Gleitpfad zu keinem Punkt durch Geländeerhöhungen bzw. der Überlandfreiheit (s.o.) begrenzt beschränkt ist.


Wenn der Endanflug durchs Gelände führen würde, ist, Wolkenstraßen oder Hangwind o.ä. ausgenommen,
ziemlich sicher, daß erneut gekurbelt werden muß, um sicher (also mit den eingestellten Werten) den
Wegpunkt erreichen zu können.

Während der Berechnung der Ankunftshöhen aller landbaren Felder/Plätze kann ein ebenfalls
Sicherheits-MC- Wert eingestellt werden. Dieser Wert ist standardmäßig auf Null gesetzt.

Größere Werte führen zu konservativeren (sichereren) Anflügen.

\section{Endanflugrechner}\index{Endanflugrechner}

Der Endanflugrechnern benutzt mannigfaltige Informationen und Quellen um die Ankunftshöhe und Zeit zu berechnen.
Diese sind im einzelnen:

\begin{itemize}
\item Die Flugzeugpolare.
\item Windrichtung und -geschwindigkeit
\item Entfernung und Kurs zum Zielpunkt
\item MC-Wert
\item Höhe des Zielpunktes
\item Benutzerdefinierte Sicherheitshöhen
\item Die Totalenergie des Flugzeuges, falls das Flugzeug mit einem enstprechenden Gerät ausgestattet, dies mit \xc
verbunden ist und die IAS angibt.
\end{itemize}

Aus den oben genannten Parametern, können zwei Höhen errechnet werden:

\begin{description}
\item[Benötigte Höhe]
Diese Berechnung gibt die insgesamt benötigte Höhe um das ausgewählte Ziel zu erreichen,
zusätzlich vom Piloten bestimmter Sicherheitshöhen.
\item[Höhendifferenz]
Diese Kalkulation errechnet die benötigte Höhendifferenz zum Zielpunkt  plus der Höhe des
Ziels (MSL) minus der Flughöhe des Flugzeuges (MSL).

Das Ergebnis entspricht also entweder der aktuellen Flughöhe über dem Gleitpfad oder aber der Ankunftshöhe am Ziel.
Wenn keine Höhe des Wegpunktes in der Wegpunkt-Datenbank (z.B. $GER-WP.txt$) hinterlegt ist, wird \textsf{XCSoar}
die Höhe des Geländes aus dem Gelände File (z.B.: $GER-High.xcm$) benutzen.
\todonum{und wenn nicht vorhanden? Gibt es eine Fehlermeldung?}
\end{description}

Die Berechnung des Endanfluges ist soweit fortgeführt worden, daß sowohl die Höhen als auch die Höhendifferenzen
berechnet werden, um die komplette Aufgabe zu erfüllen. Auch \textsf{XCSoar} kann also "um die Ecke rechnen"

Diese Höhendifferenz wird permanent am linken Rand des Displays sowohl numerisch als auch als roter,
grüne oder orangene  Pfeile nach unten oder oben dargestellt.

Diese Höhe ist korrigiert um die Energie-Höhe (potentielle Energie), aufgrund der Tatsache, daß potentielle
Energie in kinetische Energie umgewandelt werden kann - abzüglich Verlusten.

Die kinetische Energie, welche vom Flugzeug in Höhe umgewandelt werden kann (Hochziehen) wird aus der
Differenz der TAS zur Geschwindigkeit des besten Gleitens gebildet. Diese Berechnung ist am genausten wenn
entsprechende Daten vorhanden sind (Abgriff vom Fahrtenmesser o.ä.), andernfalls wird die True Airspeed
aus der Wind- und Übergrundgeschwindigkeit ermittelt. Dazu mehr im folgenden Teil

\section{Anzeige der benötigten Höhe für die ganze Aufgabe}\index{Benötigte Höhe für ganze Aufgabe}\index{Endanflugspfeile}

Auf der linken Seite der Karte wird über die gesamte Flugdauer hinweg die für die gesamte Aufgabe benötigte
Höhe sowohl als Pfeil als auch in numerischer Form dargestellt.  Wenn der Flieger über der minimal erforderlichen
Höhe \emph{zum Erreichen der Aufgabe} ist, erscheint ein grüner, nach oben gerichteter Pfeil.
 Ist der Flieger zu tief, erfolgt die Anzeige ganz ähnlich, aber in rot und nach unten.


Bei mehr als 500m positiver oder negativer Höhe wird zusätzlich eine abgesetzte Pfeilspitze in entsprechender
Farbe dargestellt; dies dient lediglich der augenblicklcihen Erkennung der Lage.

\emph{Sollte erreichbare Landepunkte in Reichweite sein, die aktuelle Höhe jedoch nicht ausreichen, um die Aufgabe zu beenden,
so werden die Pfeile orange angezeigt.}


\begin{center}
\includegraphics[angle=0,width=0.5\linewidth,keepaspectratio='true']{figures/XCS64-Final-Above-Below.png}
\end{center}

Die Skalierung dieser Pfeile der Endanflugsanzeige beträgt $+/-$ 500 m.

\subsection*{Anzeige der benötigten Endanflugshöhe auf dem Bildschirm}\index{Endanflug, benötigte Höhe, Anzeige}

Die Anzeige der benötigten Endanflugshöhe wurde derart gestaltet, daß auf einen Blick der
Einfluß des MC-Wertes sehr übersichtlich erkenntlich wird.

\emph{Die vertikalen Pfeile zeigen die notwendige Höhe zum Beenden der Aufgabe
beim jeweils gewählten MC-Wert und gleichzeitig  bei MC$=0$ an.}

Hierzu sind die Pfeile in den Farben grün, rot und orange dargestellt und zusätzlich ist eine entsprechend
abgesetzte gefärbte Pfeilspitze dargestellt, die ab einer zusätzlich notwenigen oder verfügbaren Höhe
von 500m erscheint.

Halbtransparente Pfeile der gleichen Farbe und halben Breite am linken Rand des Displays geben den entsprechenden
Wert für MC$=0$ an.

Die eingeblendete Zahl neben den Pfeilen gibt die notwendige Höhe für den aktuell gewählten  MC-Wert an.

Hier ein paar Beispiele für verschiedene Szenarien:

\begin{description}
\item[Über Endanflug]
Über dem Endanflugshöhe bei  MC$=M$ und MC$=0$:

Bei der derzeitigen MC Einstellung beträgt die Reserve $132m$ wird auf MC$=0$ umgeschaltet gibt es etwas mehr Reserve.
\marginpar{\includegraphics[angle=0,width=1.0cm,keepaspectratio='true']{figures/XCS64-Complete-Above.png}}

\item[Über Endanflug]

über Endanflughöhe bei MC$=0$ aber unter Endanflughöhe bei  MC$=M$, und :

Hier zeigt  das Display eine Situation, bei der die vorhande Höhe  mit dem eingestellten MC-Wert nicht ausreicht,
um die Aufgabe zu beenden (roter Pfeil nach unten). Für MC$=0$ jedoch stellt der Rechner das Gelingen
der Aufgabe in Aussicht (halbtransparenter  Pfeil ganz links nach oben) und gibt dafür eine Reserve von 152$m$ an.
\marginpar{\includegraphics[angle=0,width=1.0cm,keepaspectratio='true']{figures/XCS64-Above-0-Below-MC.png}}
In dieser Situation kann der Pilot entscheiden, ob er den Bart verlässt und den Endanflug
mit einer geringeren MC-Einstellung fortführt, oder er weitersteigen möchte.

Hier ist es sinnvoll, die Auto-MC Einstellung anzuschalten, die den MC-Wert automatisch auf den
optimalen MC-Wert einstellt. Hierdurch ist es extrem einfach für den Piloten, die ermittelte Steigrate
mit dem MC-Wert zu vergleichen. Wenn das erreichte Steigen unter das MC-Niveau fällt,
sollte der Bart verlassen werden

\item[Unter Endanflughöhe]
Unter Endanflughöhe  bei MC$=M$, und etwas weniger als 500m  unterhalb bei MC$=0$
\marginpar{\includegraphics[angle=0,width=1.0cm,keepaspectratio='true']{figures/XCS64-Final-below-0.png}}
Das Display zeigt an, daß beim aktuellen MC-Wert die Endanflughöhe nicht erreicht ist,
bei MC$=0$, der jedoch nur weniger als 500m fehlen.
Mit dem gewählten MC-Wert fehlen satte 733m.

\item[Unter Endanflughöhe]
Unter Endanflughöhe bei  MC$=M$, und bei MC$=0$
\marginpar{\includegraphics[angle=0,width=1.0cm,keepaspectratio='true']{figures/XCS64-Final-below-Complete.png}}
Die Aufgabe kann so nicht beendet werden, auch nicht bei MC$=0$.
Die roten Pfeile zeigen, daß eine Außenlandung wahrscheinlich ist,  zu hoch gepokert \texttt{:-(}
\end{description}

Und hier mal ein Beispiel mit einer Komplettansicht der MovingMap:
\begin{center}
\includegraphics[angle=0,width=7cm,keepaspectratio='true']{figures/finalglide-allbelow.png}
\end{center}
Hier hat man sich  verpokert.
Der Endanflug reicht weder mit dem eingestellten MC noch mit MC$=0$ aus, man erreicht weder den gewählten Wegpunkt \textsf{Damme},
dort gibt es deutlich vorher eine Geländekollision (gekennzeichnet durch das orange Kreuz auf der Karte),
und der Ausweichflugplatz \textsf{Achmer} liegt gaaanz knapp außerhalb der Reichweite.

DG $\rightarrow$ Dumm Gelaufen \verb":-("

\section{Ermittlung der Schnittgeschwindigkeiten }\label{sec:task-speed-estim}

Viele von \textsf{XCSoar's} internen Algorythmen benutzen \emph{Annahmen} der benötigten Zeit,
einen bestimmten Punkt zu erreichen.
(Es sind ja Berechnungen ''in die Zukunft''  aus bisher gesammelten Informationen.

Diese Information wird in einigen Infoboxen benutzt, weiterhin auch in AAT-Aufgaben
sowie zur Warnung bei Erreichen nach Sonnenuntergang.

Der Rechner nimmt an, daß die erreichbare Überlandflug-Schnittgeschwindigkeit
gleich ist, wie die nach der klassischen MC-Theorie, Wind mit einberechnet.

Diese Annahme wird benutzt um Ankunftszeiten und Aufgabenankunft zu berechnen.

Die folgenden Aufgaben-Geschwindigkeiten sind definiert bzw. werden kalkuliert:
\begin{description}
\item[Bisher erreichte Aufgabengeschwindigkeit (Task speed achieved)]  Dies ist die \emph{aktuell
erreichte} Schnittgeschwindigkeit, kompensiert um die Höhe Aufgabenstart \todonum{was heißt das exakt?}
\item[Mittlere Aufgabengeschwindigkeit (Task speed average)] Dies ist die angenommene
Schnittgeschwindigkeit, kompensiert um die Höhe, welche zur Beendigung der Aufgabe benötigt wird.
\item[Verbleibende Geschwindigkeit (Task speed remaining)]  Dies ist die angenommene Schnittgeschwindigkeit,
welche sich exakt nach der MC-Theorie ermittelt für den Task ergibt.
\item[Aktuelle Aufgabengeschwindigkeit (Task speed instantaneous)]  Dies ist der augenblickliche
angenommene Schnitt über die ganze Aufgabe
Wenn mit aktuell eingestelltem MC-Wert gekurbelt wird, ist dieser Wert
identisch mit der angenommenen Aufgabenschnittgeschwindigkeit.
Wenn in Bärten kleiner als der eingestellte MC-Wert gekurbelt wird, oder aber vom Kurs
abgewichen wird,  wird dieser Wert kleiner sein als die angenommene Schnittgeschwindigkeit.
Beim Vorfliegen in ruhiger Luft mit optimaler Geschwindigkeit wird diese Geschwindigkeit
gleich der angenommenen Aufgaben-Schnittgeschwindigkeit sein.

Dieser Wert, welcher als {\InfoBox} dargestellt werden kann, ist nützlich als eine kontinuierliche Anzeige
der Überlandperformance. Er wird nicht in anderen internen Kalkulationen zu Berechnungen herangezogen
\end{description}

Für AAT-Aufgaben wird anhand der MC-Einstellungen automatisch die optimale Position
der Wendepunkte errechnet.  \tip Wenn jeder Wendepunkt auf  'auto'  gesetzt wurde,  berechnet
\textsf{XCSoar} die Position der Wendepunkte derart, daß die AAT - Aufgabe nicht später als z.B. 5
Minuten nach der vorgegebenen Aufgabenzeit beendet wird.

Analog hierzu, wird ein zusätzlicher MC-Wert '\emph{MC-erreicht}' {\em achieved MacCready} berechntet.
Dieser Wert wird so ermitteltt, indem vom erreichten Schnitt zurückgerechnet wird,
auf den MC-Wert, welcher dieselbe bisher erreichte Schnittgeschwindigkeit ergibt.

Der Wert ist größer als der aktuell eingestellte MC-Wert, wenn in stärkeren Bärten gekurbelt
wird, oder aber z.B. unter Wolkenstraßen geflogen wird.

Der 'MC-\emph{erreicht}' -Wert wird im Aufgaben-Dialog verwendet.

Die ermittelten Geschwindigkeitswerte für die bisher erreichten Schnitte sind höhenkompensiert,
sodaß Steigen und Sinken mit in die Ermittlung der mittleren Aufgaben-Schnittgeschwindigkeit eingehen:
Angenommen, zwei Flieger fliegen dieselbe Aufgabe.
Flieger A ist schneller vorgeflogen, verliert entsprechend Höhe für die Geschwindigkeit.
Flieger B ist hinter Flieger A, aber er ist höher und er wird Zeit sparen, da er weniger kurbeln muß, um
die Endanflughöhe zu erreichen.

Beim Fliegen von AAT Aufgaben können sich die Geschwindigkeitsangeban beim Einfliegen in
den Sektor ändern. Der Grund liegt darin,  daß evtl. der Pilot den Wendepunkt manuell
verschoben  oder den Sektor geändert hat. Da sich damit die zu fliegende Strecke ändert,
ändert sich zwangsläufig der Schnitt.


\section{Optimaler Vorflug-Kurs}

Um während des Vorfluges zwischen den einzelnen Wendepunkten (nicht im Endaflug!) den Weg-bzw.
Zeitverlust zu minimieren, wird \textsf{XCSoar} einen optimalen Weg, den "optimalen Vorflug-Kurs"
vorschlagen. Dieser Kurs ist optimiert, insofern als er die Windabdrift beim Geradeausflug als auch beim
Kurbeln mit einberechnet. Dementsprechend wird der MC-Wert ein anderer sein, als bei der klassischen
MC-Theorie.


\begin{center}
\begin{maxipage}
\centering
\def\svgwidth{0.8\linewidth}
\includegraphics[angle=0,width=0.8\linewidth,keepaspectratio='true']{figures/figure_optimal_cruise.png}
\end{maxipage}
\end{center}

Der optimale Vorflugkurs wird als dicker blauer Pfeil auf dem Schirm angezeigt und er empfiehlt dem Piloten
exakt dieser Richtung nachzusteuern.

Wenn z.B. die Karte auf 'Kurs Oben'  (Track-Up) Anzeige eingestellt ist, steuere so, daß der dicke blaue Pfeil
genau nach oben zeigt.

Der optimale Kurs wird unter Berücksichtigung des Windes errechnet und zwar in der Art, daß sich die
schnellstmögliche Ankunft am Ziel ergibt. Wenn der Wind zu vernachlässigen ist oder sich der Rechner im
Endanflugmodus befindet, wird der blaue
Pfeil deckungsgleich mit der schwarzen Linie, die die Richtung zum nächsten Punkt angibt, sein.

Die Berechnung und Anzeige des optimalen Kurses ist einzigartig in \textsf{XCSoar}. Normalerweise,
wenn zwischen Bärten vorgeflogen wird, empfehlen Segelflugrechner, den Kurs direkt zum nächsten Wendepunkt.
Idealerweise aber sollten sie den Windversatz mit berücksichtigen, sodaß der minimale Weg zurückgelegt werden
muß, mit dem Nebeneffekt, schneller zu sein.

Sollte während des Endanfluges doch noch Kurbeln notwendig sein, so wird der Flieger auf jeden Fall
irgendwie versetzt, was der Rechner registriert und somit den track entsprechend anpaßt. Nach einigen
Bärten wird der Kurs daher eher kurvig bzw. zackig  angezeigt.

Für den Fall, daß das Ziel als Wegpunkt aktiv ist und man sich oberhalb der Endanflughöhe befindet, ist
ein Kurbeln nicht notwendig, daher ist dies einfache Schema optimal.

\section{Auto MacCready}\label{sec:auto-maccready}\index{MC-Optimierung}
\textsf{XCSoar} kann den MC-Wert automatisch anpassen, um den Piloten zu entlasten:
Dazu sind zwei Methoden bzw. Einstellungen möglich:
\begin{description}
\item[Endanflug]  Während des Endanfluges wird der MC-Wert so optimiert, daß das Ziel in der
\emph{kürzest möglichen Zeit} erreicht wird.

Bei OLC-Sprint tasks wird der MC-Wert dagegen so angepaßt, daß \emph{eine
möglichst große Strecke} in der verbleibenden Zeit zurückgelegt wird und daß
das Ziel in der vorgegebenen Höhe erreicht wird!
\item[Erwartetes mittleres Steigen]
Wenn man sich nicht im Endanflug befindet, wird der MC-Wert so gewählt, daß er dem
Mittelwert aller aller bisher erflogenen Bärte entspricht!
\item[Beides] Zusätzlich können beide Methoden benutzt werden, sodaß vor
Erreichen des Endanfluges -also während der ganzen Aufgabe- das erwartete mittlere
Steigen herangezogen wird und im Endanflug optimiert auf die
 Ankunftszeit (streng nach MC-Theorie)  gerechnet wird.
\end{description}

(Zu deustch: einmal rechnet er besten Schnitt, einmal größte Strecke\dots im
dritten Falle kombiniert er beides )

Ein- bzw. ausgeschaltet wird die Auto MC -Funktion mit

\begin{quote}
\bmenu{Konfig.}\blink\bmenut{Mac}{Auto}
\end{quote}

In den Konfigurationseinstellungen ist  'Beides' als Standard voreingestellt.


Falls Auto MacCready angeschaltet wurde, erscheint 'AUTO'  in der MC-Infobox
anstelle von 'manuell'. Gleichzeitig erscheint in der Vario-Anzeige 'AutoMC' anstelle von 'MC'.

Diese Auto MC-Einstellungen werden hier hier im Anschluß beschrieben:

\subsection*{Endanflug}\index{Endanflug}
Wenn man sich über dem Gleitpfad befindet, so wird man in der Regel den MC-Wert etwas höher einstellen,
um schneller anzukommen. Indem man dies tut, setzt man automatisch voraus, daß die zu noch erwartenden
Bärte auch stärker werden.

Befindet man sich unterhalb des Gleitpfades, so wird man den MC-Wert etwas niedriger einstellen.
Damit nimmt man automatisch an, daß auch die Bärte entsprechend schlechter werden, in denen gekurbelt wird.

Die Funktion \textsl{Auto MacCready} nimmt diese Einstellung automatisch und kontinuierlich vor.
Normalerweise ist es sinnlos, diese Einstellung vor annähernd Erreichen der Endanflughöhe  zu wählen, da man sich
zu Begin des Fluges erheblich unterhalb der Endanflughöhe befindet und der Rechner mit dieser Einstellung dann
-logischerweise- den MC-Wert auf Null stellen würde.


\begin{maxipage}
\begin{center}
\includegraphics[angle=0,width=0.8\linewidth,keepaspectratio='true']{figures/figure_auto_maccready.png}
\end{center}
\end{maxipage}

\subsection*{Erwartetes mittleres Steigen}

Diese Einstellung stellt den MC-Wert auf den gemittelten Steigwert aller bisher erflogenen Bärte.
Damit wird automatisch auch die Zeit, welche in den Bärten verbracht wurde, berücksichtigt.
Der Wert wird nach jedem Verlassen eines Bartes aktualisiert.

Da die MC-Theorie genau dann optimal funktioniert, wenn der eingestellte MC-Wert exakt gleich dem
nächsten -leider unbekanntem- Bartsteigen, führt diese Theorie mitunter zu suboptimalen Ergebnissen,
z.B. Vorfluggeschwindigkeit zu langsam, falls sich die Bärte deutlich bessern, oder auch zu zu schnellem Vorflug,
falls sich die Wetterbedingungen verschlechtern.

Gleichzeitig,  falls der Pilot in über längere Zeit zu schwachem Steigen kurbelt,
wird dies das errechnete mittlere Steigern verringern, und so den Piloten veranlassen, auch
weiterhin in zu schwachen Bärten zu kurbeln.


\textsf{\textcolor[rgb]{0.00,0.00,0.50}
{Die genaue Kenntnis der Schwächen bzw. dieser Limitierungen
der MC-\textsl{Theorie} sollte  dem Piloten vertraut sein und ihn dazu veranlassen, sich mit dem
Gesamtsystem bestehend aus Wetter, Flugzeug, Piloten können und Software eingehend zu beschäftigen
und seine Entscheidungsfindung demgemäß zu treffen. Der Rechner/die Software allein kann nicht fliegen
und auch das Wetter nicht vorhersagen! Der Pilot kann wenigstens aus dem Cockpit schauen und eine
Regenfront erkennen!}}

\section{Analyse Dialog}\index{Analyse Dialog}

Der Analyse-Dialog kann z.B. dazu benutzt werden, um z.B. die Polare zu checken.
Die Funktion ist erreichbar über:
\begin{quote}
\bmenu{Info}\blink\bmenu{Analyse}
\end{quote}

Die Polare-Seite zeigt einen  Graphen der Polare mit den aktuellen Mücken und Ballast-Einstellungen.
Weiterhin ist das beste Gleiten und die dementsprechende Geschwindigkeit abzulesen, sowie das
minimale Sinken und die demenstprechende Geschwindigkeit.
Das vorher eingestellte  Startgewicht ist ebenfalls hier  abzulesen.

Mit \button{Einstellungen} wird die entsprechende Seite aufgerufen, auf der genau diese o.g.\ Werte geändert werden können.

Die Flugzeugpolaren-Seite  des Analyse-Dialoges zeigt die totalenergiekompensierte Sinkrate bei der
entsprechenden Geschwindigkeit, wenn der Rechner mit einem intelligenten Vario verbunden ist.

Dies erlaubt es dem Piloten z.B. Testflüge in ruhiger Luft durchzuführen, um die Flugzeugpolare exakt zu vermessen.
Wird nun die gemessene mit der eingegebenen Polaren verglichen, erlaubt dies eine optimale Anpassung des
Programmes an das jeweilige Flugzeug bei der jeweiligen Beladung und dies unter Berücksichtigung der optimalen
Klappenstellung und anderen Optimierungsversuchen wie z-.B. Ruderspaltabdichtungen etc.

Die Daten werden ausschließlich erfaßt wenn die G-Rate zwischen 0,9 und 1,1 ist.
Man sollte also sehr ruhig fliegen, um die Aufzeichnungen nicht zu beeinflussen

\begin{center}
\includegraphics[angle=0,width=0.65\linewidth,keepaspectratio='true']{figures/XCS64-shot-glidepolar.png}
\end{center}

\section{Hinweise zum Flug}\index{Hinweise zum Flug}
Folgende Meldungen werden je nach Bedarf während des Fluges kontinuierlich in einem
Meldefenster auf dem Bildschirm eingeblendet:

\begin{itemize}
\item Erwarte zu frühe Aufgabenankunft (bei AAT Aufgaben)
\item Erwarte Ankunft nach Sonnenuntergang
\item Erhebliche Windänderungen
\item Flugzeug befindet sich gemäß Berechnung und Einstellungen  über/unter dem Gleitpfad.
\end{itemize}

%%%%%%%%%%%%%%%%%%%%%%
\input{ch06_atmosphere_and_instruments.tex}
%%%%%%%%%%%%%%%%%%%%%%
\input{ch07_airspace_and_flarm.tex}
%%%%%%%%%%%%%%%%%%%%%%
\chapter{Avionik und Zusatzgeräte}\label{cha:avionics-airframe}

In diesem Kapitel wird beschrieben, wie diverse Geräte, wie z.B.\ ein GPS oder ein Vario und eventuelle weitere Sensoren anzuschließen sind. 
\warning  Da es sich bei \xc um Software handelt, kann an dieser Stelle nicht erwartet werden, wie diese Geräte hardwaremäßig, also mit Kabeln etc.\ anzuschließen sind. Bauanleitungen finden sich hier nicht.

Auf die Integration von \fl wird im Kapitel~\ref{cha:airspace}  eingegangen; Variometer werden in Kap.~\ref{cha:atmosph}  behandelt.
%%%%%%%%%%%%%%%%%%%%
\section{Batteriezustand}

Die meisten \textsf{PDA} sind so ausgelegt, nur ab und zu angeschaltet zu werden, also nicht für den Dauerbetrieb gedacht. Sie verfügen demgemäß über eher schwache Akkus mit magerer Laufzeit, wenn man diese mit der Dauer eines guten Überlandfluges  ($>5h$) vergleicht. 
Aus diesem Grunde wird sehr empfohlen, \textsf{PDA} und andere Geräte mittels eines separaten Akkus zu versorgen. Diese Spannungsversorgung sollte grundsätzlich von qualifiziertem Personal erfolgen, weiterhin entsprechend abgesichert und mit einem Schalter zur Trennung vom Stromkreis versehen sein.

Der größte Verbraucher eines \textsf{PDA} ist normalerweise die LCD-Hintergrundbeleuchtung, welche in den meisten Fällen notwendig ist, damit die Geräte über ein im Sonnenlicht gut erkennbares Display verfügen. Zwar gibt es Geräte mit transflektivem Display, welche bei Sonnenschein sogar brillianter werden (z.B.\ die A600-Serie von Asus, alte IPAq, oder Dell Streak), doch diese stellen die Ausnahme dar und sind inzwischen selten auf dem Markt zu bekommen.  In den allermeisten Fällen wird daher die Beleuchtung auf ''voll an'' stehen. 
Im Gegensatz dazu, kann bei einem EFIS-System, wie z.B.\ dem \al die Beleuchtung auf kleinster Stufe stehen.


Wird der \textsf{PDA}s mit dem internen Akku betrieben wird, kann  \xc einen niedrigen Batteriezustand detektieren und entsprechend abschalten, um den aktuellen Zustand des RAM (Speicher) abzuspeichern. Weiterhin kann z.B.\ nach einer gewissen Zeit an Untätigkeit ein leerer Bildschirm  dargestellt werden (eine Art Bildschirmschoner), um Energie zu sparen.  
Um wieder in den normalen  Anzeigemodus zu gelangen, wird einfach einer der Hardwareknöpfe betätigt. Wenn vom System eine Statusmeldung hierzu vorgesehen wurde, wird diese dementsprechend aktiviert/inaktiviert.

Ein andere Möglichkeit, Strom zu sparen ist, manche Features abzuschalten; so nimmt z.B.\ die ständige Neuberechnung und das Zeichnen des Terrains während des Fluges signifikante Rechenleistung der CPU und somit auch Akku-Kapazität in Anspruch.

Beim \al/VEGA-System wird die externe Versorgungsspannung im Status-Display des System-Dialogfensters angezeigt.  Hierzu auch Abschnitt~\ref{sec:system-status-dialog}).

Für andere Plattformen, ist eine \infobox{Batterie} InfoBox vorgesehen, welche über den Zustand informiert.
(z.B.\ vorhandene Kapazität, interner/externer Batteriebetrieb, Ladezustand ein/aus)

\subsection*{GPS Status}\index{GPS!Status}

\xc benötigt 3D-GPS-Fixe für die Berechnung von Navigationsaufgaben. Das bedeutet, daß mindestens vier Satelliten erkannt werden müssen.

Der Status des GPS wird am unteren Rand des Displays als kleine Icons dargestellt:

\begin{center}\begin{tabular}{c c}%{c c}
\includegraphics[angle=0,width=0.75cm,keepaspectratio='true']{icons/gps_acquiring.pdf} & \includegraphics[angle=0,width=0.75cm,keepaspectratio='true']{icons/gps_disconnected.pdf}\\
(a) & (b)
\end{tabular}\end{center}

\begin{description}
\item[(a)\quad Warte auf gültigen GPS-Fix]  Das GPS wird vom Programm richtig erkannt. Zur Funktion wird aber ein besserer Empfang benötigt. Es kann sein, daß ein 2D-Fix vorhanden ist, welcher aber nicht für alle Berechnungen ausreichend ist. Solange kein 3D-Fix erkannt wird, erscheint ein kleines Flugzeug als Symbol.

\item[(b)\quad GPS nicht verbunden]  Es kann keine Verbindung zum GPS hergestellt werden.
 (GPS ausgeschaltet, Kabelbruch , RTD/TXD vertauscht, falsche Baud-Rate, falscher COM-Port)
\end{description}\index{GPS!Fehler!Verkabelung}

Falls ein GPS für mehr als eine Minute keine gültigen Daten sendet, wird \xc die Datenverbindung automatisch zurücksetzen, neu aufgebaut und wiederum auf gültige Daten gewartet. Dies wird solange fortgesetzt, bis gültige Daten zur Verarbeitung vorhanden sind. 

Diese Methode hat sich als die zuverlässigste bewährt, wenn Kommunikationsfehler auftreten.

\xc kann zur Redundanz bis zu zwei GPS-Quellen verwalten\index{GPS!Redundanz}
Diese GPS Quellen werden auf der NMEA-Anschluß-Seite konfiguriert
\menulabel{\bmenut{Konfig.}{2/3}\blink\,\bmenut{System}{Einstellung}\\[5pt]\mbutton{Einstellung}\blink\,\button{\footnotesize NMEA-Anschluß}} und als Gerät A bis F bezeichnet. 

Hier können sowohl serielle Schnittstellen (COM-Port, bei älteren Geräten), TCP-Ports (für z.B.\ die Anbindung an das Netzwerk), UDP-Ports, Geräteinterne Ports (bei Android) oder IoIo Ports (Schnisttelle von Android auf seriell) etc.\ angewählt und mittels entsprechender Treiber konfiguriert werden.  


Falls während des Betriebes das primäre GPS ausfällt, wird auf das sekundäre GPS zurückgegriffen. Wenn beide Geräte gültige Fixe erzeugen, wird grundsätzlich das primäre Gerät benutzt und das sekundäre ignoriert.  Aus diesem Grunde ist es dringend vorzuziehen, das beste und zuverlässigste Gerät als primäres Gerät (Gerät A) zu definieren.

\subsection*{GPS Höhe}

Manch ältere GPS  (und auch manche neue!) geben die Höhe nicht relativ zum MSL aus, sondern beziehen sich auf den WGS84-Ellipsoid.  \xc detektiert dies und muß anhand einer internen Tabelle jede Koordinate umrechnen. Diese Umrechnung ist nicht notwendig, wenn \fl oder \al benutzt werden, welche die Höhe korrekt bezogen auf MSL ausgeben.
%%%%%%%%%%%%%%
\section{Schalter als digitale Eingänge}\index{Digitale Eingänge}\index{Externe Schalter}

\xc unterstützt Sensoren und Schalter, welche an einem Hauptrechner angeschlossen sind, um z.B. Alarme auszugeben. Die Zustände dieser Schalter bzw. Eingänge können auf folgende Arten von \xc verarbeitet werden:

\begin{description}
     \item[Serielle Schnittstelle]  Einige ''intelligente'' Varios, wie das \textsf{triadis} VEGA  können etliche Eingänge verwalten und senden diese über das NMEA-Protokoll zu weiterverarbeitender Soft- oder Hardware wie z.B.\ an \xc oder aber an das EFIS System \al.
     \item[1-Draht-Gerät]  \textsf{triadis}\al und das VEGA Variometer unterstützen den 1-Draht Bus und können so digitale und analoge(!) Signale verarbeiten
     \item[Bluetooth Gerät]  Sehr viele Pocket \textsf{PC}s unterstützen Bluetooth, sodaß bspw.\ ein Game-Pad mit mehreren  Hardwareknöpfen unterstützt werden kann. 
     Dies eignet sich aber wohl eher als Interface für die Bedienung von \xc anstelle für den Eingang von Flugzeugmeldungen wie Fahrwerks, Klappen oder Stall-Warnungen.
     \textsl{Die Zuverlässigkeit der Taster/Eingänge sollte hier immer im Vordergrund stehen}.
        Dennoch -für Bastelfans- es ist nicht ausgeschlossen, hierüber digitale Eingänge zu realisieren.
\end{description}

In einer an eigene Bedürfnisse angepaßten Datei z.B.\ \texttt{Flugzeug-Meldungen.txt} wird eine Tabelle erstellt, in der die jeweiligen Eingänge gelistet und beschrieben werden. Ein Standardsatz von Ereignissen innerhalb dieser Datei  kann z.B.\ folgendes enthalten:
\begin{itemize}
  \item Landeklappen
  \item Wölbklappenposition (positive/Landestellung, neutral, negativ/Vorflug)
  \item Fahrwerk
\end{itemize}

Dieser Datensatz kann in kommenden Versionen auch durch Motor- bzw- Spritwarnungen erweitert werden.
Speziell bei Verwendung des VEGA können komplexere Meldungen erstellt werden, welche logische Eingänge verknüpfen und geschlossen ausgeben wie zum Beispiel: ''Landeklappen ausgefahren und Fahrwerk verriegelt''. 
Hierzu sollte die VEGA Dokumentation im Detail herangezogen werden.
%%%%%%%%%%%%%%%%%%%%%%%%%%%%%%%%%%%%%%%%
\section{Schalter-Dialog (\textbf{nur} bei VEGA Vario)}

Wenn das VEGA Variometer angeschlossen ist, erscheint gibt es ein zusätzliches
 \menulabel{\bmenut{Konfig}{1/2}\blink\bmenus{Vega}}
Menü für die am VEGA angeschlossenen Schalter und Eingänge. Ohne VEGA ist dieser Menü-Button grau, also nicht verfügbar. 


Dieser Dialog wird in Echtzeit aktualisiert, so daß der Pilot \menulabel{~~~~~~~~~~~~~~\bmenut{Flugzeug}{Schalter}} direkt 
nachverfolgen kann, wie seine Meldung verarbeitet wird und ob diese korrekt sind. 
Sinnvoll z.B. bei einem Klappencheck vor dem Start.


\begin{center}
\includegraphics[angle=0,width=0.6\linewidth,keepaspectratio='true']{figures/dialog-switches.png}
\end{center}
%%%%%%%%%%%%%%%%%%%%%%%%%%%%%%%%%%%%%%%
\section{Slave Modus, Doppelsitzermodus, Durchschleifen von Daten}\index{Slave-mode}\index{Dosi-Modus}

Einer der im NMEA-Anschluß Menü befindlichen Treiber (''NMEA-Output'')
bietet den 
\menulabel{\bmenut{Konfig}{3/3}\blink\,\bmenut{NMEA}{Anschluss}} 
sog.\  ''Slave-Modus'', welcher verwendet werden kann,  um z.B. zwei Geräte  (\al oder \textsf{PDA})  im Master-Slave Modus miteinander zu verbinden. 
\menulabel{\qquad\quad\blink\,\button{Einstellung}} 
Im als Master definierten Gerät muß dazu unter \button{NMEA Anschluss}der Treiber des zweiten  Gerätes (Gerät B) als NMEA OUT gesetzt werden.
Hierdurch werden alle ein- und ausgehenden Datensätze  durchgeschleift und an das als Slave definierte Gerät weitergereicht.
Im Slave-Gerät muß dann als Gerät A z.B.\ \fl als Treiber angewählt werden.

Als Beispiel seien im linken Bild der Master, rechts der Slave dargestellt, welche von einem \fl gespeist werden.:


\begin{center}%ab V6.4
\includegraphics[angle=0,width=0.45\linewidth,keepaspectratio='true']{figures/config-nmea-ms-master.png}\quad
\includegraphics[angle=0,width=0.45\linewidth,keepaspectratio='true']{figures/config-nmea-ms-slave.png}
\end{center}

%\begin{center}% bis V 6.25
%\begin{tabbing}
%Anschluß\quad\=  COM1  hspace{10em}   \=  Anschluß\quad\=  COM1\kill
%\textbf{Master-Gerät:} \>             \> \textbf{Slave-Gerät:}\\[0.5em]
%\textsf{Gerät A}       \>             \> \textsf{Gerät A}\\
%Anschluß\quad\>  COM1  \>                Anschluß\quad\>  COM1\\
%Baudrate\> 19200       \>                Baudrate\> 19200\\
%Treiber                \>           \al\>                           Treiber\> \al\\[0.75em]
%
%\textsf{Gerät B}       \>                         \>  \textsf{Gerät B}\\
%Anschluß\>  COM2       \>           Anschluß\>  COM1\\
%Baudrate\> 19200       \>           Baudrate\> 19200\\
%Treiber\> NMEA OUT     \>           Treiber\> \textsl{beliebig}\\[0.75em]
%\end{tabbing}
%\end{center}



Auf diese Art und weise werden beide Geräte mit den identischen Daten versorgt, so, als ob sie direkt vom \fl, VEGA, o.ä.\ kämen. 
Auf gleiche Weise können zwei \al mit identischen Daten versorgt werden, wenn nur eine Datenquelle zur Verfügung steht.

Zur entsprechenden Verkabelung bitte die entsprechenden Handbücher der Geräte konsultieren.

%%%%%%%%%%%%%%%%%%%%%%%%%%%%%%%%%%%%%%%%%%%%%%%%%%55
\section{Systemstatus / angeschlossene Geräte}\label{sec:system-status-dialog}\index{Status!System}\index{Status!angeschlossene Geräte}

Der Systemstatus-Dialog (siehe Kap.~\ref{sec:dialog-windows}) wird hauptsächlich zum Systemcheck benutzt und hier vor allem, ob die Verbindung zu den angeschlossenen \menulabel{\bmenut{Info}{2/3}\blink\,\bmenus{Status}}
Geräten voll etabliert und OK ist. 
Im Unterpunkt \button{System} sind die entsprechenden Daten zu finden.
%%%%%%%%%%%%%%%%%%%%%%%%%%%%%%%%%%
\section{Mehrere Verbindungen zu externen Geräten}\index{Externe Geräte}

Es können bis zu zwei externe Geräte konfiguriert werden, welche parallel angeschlossen und betrieben werden (einige \textsf{PDA} haben zwei serielle Schnittstellen, Verbindungen per Bluetooth können prinzipiell mehrere unabhängige Verbindung etablieren)

Wenn  beide der Geräte GPS-Signale zur Verfügung stellen können, dann wird das primäre, erste Gerät von \xc ausgewählt und das zweite bleibt ignoriert. Sowie die erste GPS -Quelle aus irgendeinem Grunde versagt, wird automatisch auf die zweite Quelle umgeschaltet - solange, bis die erste Quelle wieder gültige GPS-Daten übermittelt. (fallback)

Das gleiche gilt für alle Werte wie z.B. barometrische Höhe, Vario, Geschwindigkeit etc:
\xc bevorzugt grundsätzlich das primäre Gerät, und greift auf das sekundäre Gerät zurück, sowie das Primäre versagt.

\textit{Beispielkonfiguration:}
Primäres Gerät ist Cambridge CAI 302, das sekundäre Gerät sei \fl -- diese Kombination bietet von beiden Geräten das Beste.
%%%%%%%%%%%%%%%%%%%%%%%%%%
\section{Verwaltung der angeschlossenen Geräte, NMEA-Anschluß etc.\ }\index{NMEA!Info}
%%%%%%%%%%%%%%%%%%%%%%%%%%%%%%%%%%%%%%

Hier können angeschlossene Geräte und deren Ausgänge betrachtet werden.

Der Button \button{Wiederverbinden} dient dazu, das gerät mit \xc neu \menulabel{\bmenut{Konfig}{2/3}\blink\bmenut{NMEA}{Anschluss}} 
wiederzuverbinden. Dies geschieht normalerweise automatisch, aber in manchen Fällen kann es sinnvoll sein, dies manuell zu versuchen - bspw.\ Fehlersuche\dots

Der Button \button{Flug Donwload} ist nur verfügbar, wenn ein IGC-Logger an \xc angeschlossen ist. \ref{sec:supported-varios}
Wenn anklickt wird, erscheint eine Liste aufgezeichneter Flüge, welche entsprechend geladen werden kann. Die Datei wird geladen und in das \texttt{logs}-Unterverzeichnis des \texttt{XCSorData } - Verzeichnis kopiert.

\button{Bearbeiten} ermöglicht die Bearbeitung der NMEA-Quellen und Anschlüsse

Der Button \button{Verwalten} ist nur dann verfügbar, wenn ein VEGA oder ein CAI 302 angeschlossen ist. Hier werden spezielle Eigenschaften dieser beiden Geräte unterstützt. Bspw.\ das Löschen des Flugspeichers beim CAI 302. \warning Genau hier kann es Verzögerungen kommen, da teilweise ein Firmware-Fehler des CAI 302 auftritt.

Mit \button{Überwachung} werden die Ausgaben der angeschlossenen Geräte in Klartext  ausgegeben\footnote{Hoffe, das ist richtig so}. 
Sinnvoll zum Debuggen bzw.\ finden von Fehlern bei ''komischen'' GPS-Quellen.
%finished für v6.5 date 09/02/2013 OH

%%%%%%%%%%%%%%%%%%%%%%
\input{ch09_quickstart.tex}
%%%%%%%%%%%%%%%%%%%%%%
\input{ch10_infobox_reference.tex}
%%%%%%%%%%%%%%%%%%%%%%
\chapter{Konfiguration}\label{cha:configuration}
\xc ist ein hoch konfigurierbares Segelflugrechnerprogramm und kann sehr flexibel auf die jeweiligen Vorlieben
und Bedürfnisse angepasst werden.

Dies Kapitel beschreibt die Konfigurationsmöglichkeiten und  -optionen.
\section{Übersicht der Konfiguration}\index{Konfiguration}
Mögliche  Änderungen der Konfigurationsvoreinstellungen innerhalb \xc, welche hier behandelt werden:
\begin{itemize}
\item Ändern von  Konfigurationseinstellungen. Dies ist die Art von Konfiguration, die die meisten
    Benutzer anwenden werden, da die internen Konfigurationen teilweise genauere Kenntnis benötigen.
    Mithilfe dieser Möglichkeiten können so gut wie alle notwendigen  Einstellungen vorgenommen werden.
    Das Hauptaugenmerk in diesem Kapitel beschäftigt sich daher mit diesen Konfigurationsvoreinstellungen.
 \item Ändern der Benutzersprache, oder wenigstens ändern der Texte innerhalb der
     Informationsfenster.
\item Ändern der Zuordnung von Buttons und Menüs. Es können Inhalt und Struktur der Buttonmenüs
    geändert werden.
\item Ändern oder Zuordnen von Aktionen, welche stattfinden sollen, wenn bestimmte Ereignisse
    auftreten.
\item Definieren, wie lange Statusmeldungen erscheinen und/oder welcher Sound dazu gespielt werden
    soll.
\end{itemize}
Da es sich bei \xc  um ein quelloffenes OpenSource System handelt, ist die Beschreibung aller
Möglichkeiten der Einstellungen in diesem Kapitel so gut wie unmöglich. Prinzipiell kannst Du \xc auch
selber kompilieren und wirklich \textbf{alles} ändern.

Solltest Du ein ernstes Interesse daran haben, wirklich detailliert in \xc  einzusteigen, um eventuell Deine
ganz eigene Oberfläche zu gestalten, so empfehlen wir,  Dich mit dem \xc-Wiki  zu beschäftigen. Hier der
Link: \url{http://www.xcsoar.org/trac/wiki}
%%%%%%%%%%%%%%%%%%%%%%%%%%%%%%%%%%%
\section{Ändern der Voreinstellungen}
Es gibt einen Menge an Einstellungen, welche vom Benutzer verändert werden
können. Bis auf das generelle Layout des Programmes kann
\menulabel{\bmenut{Konfig.}{2/3}\blink\bmenut{System}{Einstellung}} sogut wie alles verändert und
an eigene Vorlieben angepasst werden.

Du solltest es tunlichst unterlassen, an diesen Einstellungen während des Fluges
''herumzufummeln'', hier sollte Deine Aufmerksamkeit voll und ganz dem Fluge und
Deiner Umgebung gewidmet sein. \warning

 Alle Einstellungen sollten bereits vorher am Boden erfolgt sein, so daß derartige
Änderungen in der Luft nicht mehr notwendig sind.

Die Konfiguration wird über zwei Ebenen aufgerufen, in denen weitere Fenster aufgerufen werden können.
Die Dialoge zum Ändern  der Voreinstellungen sind dann auf mehrere Seiten verteilt.

\begin{center}
\includegraphics[angle=0,width=0.5\linewidth,keepaspectratio='true']{figures/config-menu.png}
\end{center}

Wenn Du eine Änderung vorgenommen hast und diese speichern willst, drücke auf \button{Schließen} oder
den Power-Knopf des \al .

Ein weiterer Tastendruck bringt Dich zurück auf die normale Kartenoberfläche.

\tip Wenn alles zu Deiner Zufriedenheit eingestellt wurde, ist es angeraten, ein eigenes Profil
abzusichern, um diese Einstellungen jederzeit problemlos restaurieren zu können. Damit können u.a. auch
andere Flieger Ihre eigenen, ganz speziellen Vorliegen für einen Schnellzugriff abspeichern.
Auch, falls dem PDA mal  die Batterie ausgeht, ist zur schnellen Wiederherstellung der persöhnlichen
Einstellungen wenigestens das Sichern eines solchen Files eine große Hilfe.

Falls ein oder mehrerer Konfigurationsfiles \verb''xcsoar-registry.prf''  bestehen, so wird beim Start als allererstes nach
einem dieser Files zur Auswahl gefragt.

Im Kap.~\ref{cha:data-files} werden die diversen Formate dieser Files mit ihren
Auswirkungen auf das System besprochen.

Um die Suche bzw. Auswahl und Übersicht einfacher zu machen, bleibt,
wenn kein File benutzt bzw. angelegt werden muß, bleibt das Auswahlfeld einfach leer.

Das Hauptkonfigurationsfenster (System Einstellung) kann im ''Basis'' oder
''Experten''-Modus durch Anklicken eines Häkchen ''Experte'' bearbeitet werden.
\sketch{figures/config-expert.png}\index{Experten-Modus}\index{Basis-Modus}

''Basis''-Modus sind einige der weniger für die Funktion wichtigen Parameter für den
einfacheren Überblick verborgen. 

\tip In der folgenden Beschreibung werden alle die
Parameter, die nur im Expertenmodus sicht- und auswählbar sind, mit
einem$^{\textcolor{blue}{\star}}$ versehen.
%%%%%%%%%%%%%%%%%%
\section{Standortdateien}\index{Standortdatei}\index{Konfiguration!Neuer Flugplatz - neue Karte}\index{Konfiguration!Fliegerlager}\label{sec:site-files}
Dieser Abschnitt beschreibt die wichtigsten Files, welche konfiguriert werden sollten, wenn
Du Dich an einem neuen Platz befindest - beispielsweise in einem Fliegerlager.

\begin{description}
\item[XCSoar Daten Pfad]  Ganz oben im Fenster steht der Pfad, an dem diese Dateien \textsl{erwartet} werden.
Steht ein pfad dort, und sind keine Dateien vorhanden, bleiben alle weiteren Felder leer.
Alle Daten sollten in einem Unterordner\verb "\XCSoarData" enthalten sein.
Es ist extrem wichtig, und als allererstes durchzuführen!

Der Pfad kann sich auf der Festplatte, der SD-Karte oder im ''normale'' Speicher
Deines Gerätes befinden.
\item[Kartenddatenbank]  Name des Kartenfiles (\verb"*.xcm") In diesem File sind alle Daten zur
Karte, Topologie, das Höhenmodell  und die wichtigsten Daten der Wegpunkte enthalten.
Das File ist intern als ein \verb''*.zip'' File aufgebaut.
\item[Wegpunkte]  Das primäre Wegpunkt-File. Wenn kein File ausgewählt wird,
werden die Wegpunktdaten aus der Kartendatenbank (\verb"*.xcm"), s.o.\ benutzt
(sofern dort vorhanden).
\item[Weitere Wegpunkte$^{\textcolor{blue}{\star}}$]  Sekundäre Wegpunktdatei. Hier können z.B. Wegpunkte und spezielle
Angaben für Wettbewerbe hinterlegt werden.
\item[Beobachtete Wegpunkte$^{\textcolor{blue}{\star}}$]  Wegpunktdatei in der spezielle Wegpunkte gespeichert werden,
für die erweiterte Berechnung wie Ankunftshöhe etc.\  in der Karte (also auf dem Bildschirm)
angezeigt werden. Die bekommen dann die bekannten grünen Rahmen.
Sinnvoll für z.B.\ bekannte und zuverlässige Thermikauslösepunkte, Bergrücken, Kraftwerke etc.\
\item[Lufträume]  Primäres Luftraumdatei. Wenn nichts angewählt wird, werden die Daten aus der
Kartendatenbank (\verb''*.xcm'') benzutzt (sofern vorhanden).
\item[Weitere Lufträume$^{\textcolor{blue}{\star}}$]  Sekundäre Luftraumdatei. Hier  können z.B.\ Lufträume wie
Wettbewerbsgebiete oder andere eigene bearbeitete Luftraumdaten  eingegeben werden.
\item[Wegpunktdetails$^{\textcolor{blue}{\star}}$]  Hier kann eine Datei angegeben werden, in der vor allem
 zu  Flugplätzen eingegeben werden können.
\end{description}

Luftraum-Dateien definieren besondere Lufträume, die der Benutzer separat hinzufügen
kann.\index{Luftraum!Angepaßte Dateien} Es können bis zu zwei Luftraumdateien
angegeben werden. Die erste Datei (die primäre) beinhaltet die
allgemeingültigen Lufträume, im zweiten --so ist es vorgesehen-- sollten NOTAM und ähnliches
abgespeichert werden. Es kann aber genausogut auf Wettbewerben als separates File
angegeben werden, in dem Ausnahmen und andere Daten abgespeichert werden.
\sketch{figures/config-site.png} Diese Dateien sind frei erstellbar und bearbeitbar und werden benutzt,
sofern sie den Konventionen bzw. dem erforderlichen Format entsprechen.


Die XCM Datenbank ist derzeit der gängige Weg,
Karten wie Toplologie, Gelände und Topgraphie einzugeben. Vor der Version 6.4 von \xc waren  hierzu noch
andere Files anzugeben, die explizit die Toplologie bzw. das Gelände enthielten.
Die Möglichkeit, dies einzeln anzuwählen wurde mit Einführung der Version 6.4 fallen gelassen.

Das XCM-File beinhaltet derzeit all diese Files: Gelände, Topologie, Wegpunkte und Lufträume.
Wenn diese Files im XCM File vorhanden sind, dann muß kein primäres Wegpunktdatei
angewählt werden und \xc nimmt die Wegpunkte aus der XCM-Datenbank.

Es steht Euch frei, beliebige Dateien zu erstellen und einzufügen, welche dann anschließend
anstelle der Kartenbankdatei benutzt werden soll. Mehr zu den Kartendateien in Kap.~\ref{sec:map}.
%%%%%%%%%%%%%%%%%%
\section{Kartenanzeige / Ausrichtung}\label{sec:map-projection}
Hier kann eingestellt werden, wie die Kartenanzeige der Moving Map relativ zum Segelflugzeug
ausgerichtet und dargestellt wird.
\begin{description}
\item[Vorflug Ausrichtung]  \label{conf:orientation} Einstellung für den Geradeausflug: \\
  {\bf Kurs nach oben}:   Die Kartenanzeige wird so gedreht, daß der Kurs nach oben gedreht ist.\\
  \sketch{figures/config-map_projection.png}
  {\bf Flugrichtung nach oben}:   Die angezeigte Karte orientiert sich anhand der Flugrichtung - diese wird nach oben ausgerichtet.\\
  {\bf Norden oben}:   Die Kartenanzeige bleibt oben nach Norden ausgerichtet; das
  Segelgflugzeugsymbol  wird in Kursrichtung gedreht. Windeinfluß ist hierbei berücksichtigt. \\
  {\bf Ziel nach oben}:   Die Kartenanzeige wird gedreht, sodaß die Richtung zum nächsten   Zielpunkt immer nach oben ausgerichtet ist.
\item[Vorflug Ausrichtung] Die Beschreibung der einzelnen Möglichkeiten ist identisch wie bei Vorflug-Ausrichtung.
\item[Steigflug Zoom]  \label{conf:circlingzoom}
Umschaltung zwischen Auto-zoom und Fix Zoom für Kurbel- und Vorflug-Modus..
  Wenn aktiviert, dann wird bei der Kartendarstellung automatisch beim Umschalten auf Steigflug der Zoom vergrößert,
  beim Zurückschalten auf Sollfahrt wieder verkleinert.
\item[Kartenreferenzpunkt] Bestimmt die Richtung, in welcher das Segelflugsymbol
aus der Mitte heraus verschoben wird. \textsl{Achtung:} \achtung Erscheint nur bei Ausrichtung ''Norden oben''\\
  {\bf Keine}: Keine Anpassung. Segelflugzeugsymbol bleibt fix an der Position (wie unten angegeben)  stehen.\\
  {\bf Kurs}: Nimmt den aktuellen mittleren Kurs als Basis. (um z.B.\ mehr Platz in Flugrichtung auf der Karte zu haben) \\
  {\bf Zielpunkt}: Nimmt den Zielpunkt als Basis der Verschiebung. (verschiebt mehr in Richtung Zielpunkt)
\item[Seglersymbol Plazierung]  \label{conf:gliderposition} Position des Segelflugzeugsymboles
vom unteren Rand in Prozent.
\item[Max. auto Zoom Entfernung]  Obere Grenze für die Auto Zoom Entfernung
\end{description}
%%%%%%%%%%%%%%%%%%
\section{Kartenanzeige / Elemente}\label{sec:map-elements}
Hier werden diverse Anzeigen wie Kurs-Spuren, Symbole und Luftverkehr konfiguriert werden:
\begin{description}
\item[Kurs über Grund] Zeigt eine Linie des geflogene Kurses projeziert auf den Boden (Bodenspurlinie)
   {\bf Aus}: Zeigt die Bodenspurlinie niemals\\
   {\bf Ein}: Zeigt die Bodenspurlinie immer\\
   {\bf Auto}: Zeigt die Bodenspurlinie nur, wenn es eine signifikante Abweichung des Flugweges zum Kurs gibt.
\item[FLARM Flugverkehr ]
   {\bf Ein}:  \label{conf:flarm-on-map} Anzeige des F\fl Flugverkehres auf der Moving Map.\\
   {\bf Aus}: FLARM - Flugverkehrsanzeige ausgeschaltet.
\item[Spurlänge\textcolor{blue}{$\star$}] \label{conf:snailtrail} Einstellung, ob und wie lang die Spur hinter dem
Segelflugzeugsymbol dargestellt werden soll. \\
\sketch{figures/config-map_elements.png}
   {\bf Aus}: Spur wird nicht gezeigt. \\
   {\bf Lang}: Eine lange Spur wird gezeigt (ca. 60 Minuten Flugweg).\\
   {\bf Kurz}: Eine kurze Spur wird angezeigt (ca. 10 Minute Flugweg). \\
   {\bf Voll}: Die komplette Flugspur seit Start wird angezeigt.
\item[Spurdrift$^{\textcolor{blue}{\star}}$] \label{conf:traildrift} Einstellung, ob die Flugspur im Steigflugmodus
mit dem Wind verschoben wird. Wenn ausgeschaltet, wird die Flugspur ohne Winddrift angezeigt.\\
   {\bf Ein}: Spurdrift eingeschaltet\\
   {\bf Aus}: Spurdrift ein ausgeschaltet.
\item[Spurtyp\textcolor{blue}{$\star$}] \label{conf:snailtype} Bestimmt Art und Aussehen der Spur.\\
  {\bf Vario \#1}: Während des Steigens wird die Spur grün und in dicker werdender Linie dargestellt,
  beim Sinken wird die Linie braun und dünner dargestellt. Nullschieber wird als graue Linie dargestellt.\\
  {\bf Vario \#1 (mit Punkten)}: Dieselbe Anzeige und Farbschema, diesmal aber gepunktete Darstellung
beim Sinken. \\
  {\bf Vario \#2}: Die Farbe für Steigen ist von orange bis rot, Sinken von hellblau bis dunkelblau. Nullschieber sind gelbe Linien.\\
  {\bf Vario \#2 (mit Punkten)}: Dieselbe Anzeige und Farbschema, diesmal aber gepunktete Darstellung
  beim Sinken.\\
  {\bf Höhe}: Die Spur zeigt entsprechend des Farbschemas die geflogene Höhe an.
\item[Skalierte Spur$^{\textcolor{blue}{\star}}$] \label{conf:trailscaled} Wenn eingeschaltet, wird die Spur anhand des Variosingals
dünner(Saufen) oder breiter (Steigen) dargestellt .
\item[Angabe zu Umwegkosten\textcolor{blue}{$\star$}]  Wenn der Steuerkurs vom geplanten Ziel abweicht können vordem
Flugzeug in Flugrichtung Zahlen eingeblendet werden. Dies Zahlen geben am Ort ihres Auftretens auf der Karte
in Prozent vom geraden Weg zum Ziel den Umweg an. Eine $13$ hieße also die Entfernung zum angeflogenen
Wegpunkt beträgt das $1,13$ fache der Entfernung direkt zum Ziel, wenn man über diese $13$
an der Stelle auf der Karte zum Ziel flöge.
\item[Flugzeugsymbol$\ast$]  Darstellung des Flugzeugsymboles auf der Karte \\
  {\bf Einfach}: Eine einfache Liniendarstellung \\
  {\bf Einfach (groß)}: Eine vergrößerte einfache Darstellung für bessere Sichtbarkeit auf kleinen Displays. \\
  {\bf Detailliert}: Eine aufwendigere Darstellung (gerendert). \\
  {\bf Hängegleiterr}: Ein vereinfachter Hängegleiter, weiß mit schwarzer Kontur. \\
  {\bf Paraglider}: Ein vereinfachter Paragleiter, weiß mit schwarzer Kontur.
\item[Wind-Pfeil$^{\textcolor{blue}{\star}}$]  Determines the way the wind arrow is drawn on the map. \\
  {\bf Aus}: Keine Darstellung des Windes auf der Karte. \\
  {\bf Pfeilspitze}: Zeigt  nur die Spitze des Pfeiles. \\
  {\bf Ganzer Pfeil}: Zeigt einen ganzen Pfeil mit gestrichelter Linie
\end{description}
%%%%%%%%%%%%%%%%%%
\section{Kartenanzeige / Wegpunkte}\label{sec:waypoint-display}
Hier können Einstellungen zur Darstellung von Wegpunkten und deren Details
vorgenommen werden.

\begin{description}
\item[Beschriftungsformat]  diese Einstellungen beeinflussen die Darstellung der Wegpunktebeschreibungen,
  welche an jedem Punkt auf der Karte zu sehen sind.\label{conf:labels}
  Es gibt vier verschiedene Formate:\\
  {\bf Vollständiger Name}: Der komplette Name des Wegpunktes wird gezeigt. \\
  {\bf Erstes Wort aus Namen}: Das erste Wort des Namens bis zum ersten Leerzeichen wird angezeigt.
   \sketch{figures/config-map_waypoint.png} \\
  {\bf Ersten 3 Buchstaben}: Die ersten 3 Buchstaben werden angezeigt. \\
  {\bf Ersten 5 Buchstaben}: Die ersten 5 Buchstaben werden angezeigt. \\
  {\bf Keine}: Es werden keine Namen zu den Wegpunkten auf der Karte dargestellt.
\item[Ankunftshöhe$^{\textcolor{blue}{\star}}$] Bestimmt ob und wie wie die Ankunftshöhe  in der Wegpunktbeschreibung (s.o.) angezeigt wird. \\
  {\bf Keine}: Es wird keine Ankunftshöhe angezeigt. \\
  {\bf Geradliniger Gleitpfad}: Ankunftshöhe auf geradliniegem Gleitpfad wird angezeigt. Das Gelände wird hierbei  nicht berücksichtigt. \\
  {\bf Gleitpfad unter Berücksichtigung des Geländes}: Die Ankunftshöhe unter Berücksichtung der Geländehöhe wird Angezeigt.
  \achtung Dies benötigt den Reichweitenmodus ''mit Umweg'' bei den Routenplanereinstellungen. \\
  {\bf Geraden \& Gelände Gleitpfad}: Es werden beide Höhen angezeigt. Dies benötigt den Reichweitenmodus ''mit Umweg'' bei den Routenplanereinstellungen.  \\
  {\bf Benötigte Gleitzahl}: Zeigt die benötigte Gleitzahl an, welche zum gewählten Wegpunkt  erforderlich ist.
\item[Stiel der Beschriftung$^{\textcolor{blue}{\star}}$]  Die Form der ''Etiketten'' an den Wegpunkten.\\
    {\bf Abgerundetes Rechteck}: Beschriftung erfolgt in schwarzer Schrift auf weißem Hintergrund in einem abgerundeten Rahmen
    {\bf Umrissen}: Beschriftung erfolgt in weißer Schrift, schwarz umrissen.
\item[Sichtbarkeit der Beschriftung$^{\textcolor{blue}{\star}}$]  \label{conf:labelvisibility}
   Zeigt an, welche Wegpunkte auf der moving map mit Beschriftung (Name und Höhenangabe wie oben
   beschrieben)  versehen werden \\
   {\bf Alle}: Alle Wegpunkte bekommen die Beschriftung. \\
   {\bf Aufgabenwegpunkte und Landbare}: Alle Wegpunkte, die in der aktuellen Aufgabe gelistet sind,
   sowie alle landbaren werden mit Beschriftung versehen. \\
   {\bf Aufgabenwegpunkte}:  Nur Wegpunkte, die in der aktuellen Aufgabe gelistet sind, werden
   mit Beschriftung versehen. \\
   {\bf Keine}:  Keine Wegpunkte bekommen eine Beschriftung.
\item[Landbare Symbole]  \label{conf:waypointicons} Es gibt drei verschiedene Stilarten:
   Lila Punkt, (ähnlich WinPilot), S/W (ein hoch kontrastreicher Stil) und ein ''Verkehrsampel''-Stil.
   Mehr dazu in Kap.~\ref{sec:waypoint-schemes}.
\item[Detaillierte Landbare$^{\textcolor{blue}{\star}}$]  Einstellung für mehr Details der Wegpunkte\\
  {\bf Aus}: zeigt die gleichen Symbole für alle Wegpunkte\\
  {\bf Ein}: zeigt die Symbole mit zusätzlicher Angabe von Pistenlänge und Richtung
\item[Größe für Landbare$^{\textcolor{blue}{\star}}$]  Hier besteht die Möglichkeit, die Größe der landbaren    Wegpunkte in Prozent zu vergrößern oder zu verkleinern.
\item[Skalierte Bahnlänge$^{\textcolor{blue}{\star}}$] Ermöglicht, die Bahnlänge anhand der wirklichen Bahnlänge im Verhältnis zu anderen
   Flugplatzbahnlängen darzustellen.\\
{\bf }: Zeigt eine fixe Bahnlänge für alle Wegpunkte\\
{\bf}: Skaliert die Darstellung anhand der realen Bahnlänge.\\

\end{description}
%%%%%%%%%%%%%%%%%%
\section{Kartenanzeige / Gelände}\label{sec:terrain-display}
Auf dieser Seite kann das Erscheinungsbild des Geländes auf der Moving Map eingestellt werden.
Die Auswirkungen der Einstellungen können sofort auf einem kleinen Ausschnitt  \sketch{figures/config-terrain.png}
der Karte beobachtet werden.

\begin{description}
\item[Gelände darstellen]  Zeigt eine digitalisierte Darstellung des Geländees unter Berücksichtigung eines Höhenmodelles.
\item[Topographie darstellen]  Zeigt topographische Eigenschaften der Karte wie Seen, Flüsse, Straßen, Städte etc.\
\item[Geländefarben]  Hier gibt es eine Auswahl an verschiedenen Farbmodi für das Gelände. Es sollte ausprobiert werden,
welche Darstellung am besten gefällt. Vor allem die Darstellung des gebirgigen Raumes kann hier gut vorher durchprobiert werden.
Es gibt aber auch eine ICAO-Darstellung.
\item[Hangschattierung$^{\textcolor{blue}{\star}}$]  \label{conf:shading} Das Gelände kann unterschiedlich schattiert werden
um z.B. Sonnenseite, Lee bzw. Luvseite anzuzeigen. Mögliche Anzeigen:\\
{\bf Aus}: Aus. Keine Schattierung. Gleiche  Helligkeiten für alle Seiten, nur die Höhe wird analog des
Höhenmodelles farblich unterschieden.\\
{\bf Fixiert}: Die Sonnenposition wird fix von Nordwesten angenommen.\\
{\bf Sonne}: Der sonnenbeschienene  Hang wird heller dargestellt, Schattenseite ist dunkel.\\
{\bf Wind}: Der Luv-Hang wird hell dargestellt, der Lee-Hang dunkler.\\ \index{Luv und Lee Darstellung}
\item[Geländekontrast $\ast$]  Hier kann die Einstellung der Tönung (Phong-Tönung) der Geländedarstellung
vorgenommen werden.
Größerer Werte sind für das Relief sinnvoll, kleinere Werte für eher
steile Berglandschaften (z.B. Alpen).
\item[Geländehelligkeit$^{\textcolor{blue}{\star}}$]  Einstellung der Helligkeit (weißton) der Geländedarstellung
\end{description}

Die zur Auswahl stehenden Geländefarbpaletten sind unten aufgelistet:

\begin{longtable}{c c c c}
\includegraphics[angle=0,width=3.0cm,keepaspectratio='true']{figures/ramp-terrain-flatlands.png}&
\includegraphics[angle=0,width=3.0cm,keepaspectratio='true']{figures/ramp-terrain-mountanous.png}&
\includegraphics[angle=0,width=3.0cm,keepaspectratio='true']{figures/ramp-terrain-icao.png}&
\includegraphics[angle=0,width=3.0cm,keepaspectratio='true']{figures/ramp-terrain-grey.png}
\end{longtable}

\begin{longtable}{c c c c}
\includegraphics[angle=0,width=3.0cm,keepaspectratio='true']{figures/ramp-terrain-imhof4.png}&
\includegraphics[angle=0,width=3.0cm,keepaspectratio='true']{figures/ramp-terrain-imhof7.png}&
\includegraphics[angle=0,width=3.0cm,keepaspectratio='true']{figures/ramp-terrain-imhof12.png}&
\includegraphics[angle=0,width=3.0cm,keepaspectratio='true']{figures/ramp-terrain-imhofatlas.png}
\end{longtable}
%%%%%%%%%%%%%%%%%%
\section{Kartenanzeige / Luftraum}\label{sec:airspace-display}

Hier kann die Darstellung von Lufträumen und mögliche Vorwarnzeiten
 beeinflußt werden.

\begin{description}
\item[Luftraumanzeige]  Hier können die Anzeigemodi für die Lufträume in
Abhängigkeit von der Höhe konfiguriert werden.
Die Luftraumfilter ermöglichen ebenfalls das Filtern von Lufträumen und Warnungen für jede
einzelne Luftraumklasse. \\
  {\bf Alles}: Es werden alle Lufträume angezeigt.\\
  {\bf bis max. Höhe}: Nur die Lufträume unterhalb dieser einstellbaren Höhe werden dargestellt.\\
  {\bf Auto}: Alle Lufträume in einem konfigurierbarem Höhenband um das Luftfahrzeug herum werden angezeigt.
\sketch{figures/config-airspace.png} \\
  {\bf Alles unter mir}:  Wie ''Auto'', aber Anzeige sämtlicher Lufträume unterhalb des Flugzeuges.
%\item[Clip altitude] For clip mode, this is the altitude below which airspace   is displayed.
%\item[Margin]  For auto and all below airspace mode, this is the height   above/below which airspace is included.
\item[Warnungen]  {\bf Ein /Aus}:  Einschalten oder Abschalten der Warnungen für Lufträume.
\item[Vorwarnzeit$^{\textcolor{blue}{\star}}$]  {\bf sec, min}: Einstellbare Zeit für vorhergesagten Luftraumeintritt, zu der eine Warnung erfolgen wird.
\item[Bestätigungszeit$^{\textcolor{blue}{\star}}$]  {\bf sec, min}: Für diese Zeit wird die Weiderholung einer Luftraumwarnung ausgesetzt.
\item[Benutze schwarze Kontur$^{\textcolor{blue}{\star}}$] Zeichnet einen schwarzen Rahmen um jeden Luftraum anstelle der entsprechenden Luftraumfarbe
\item[Luftraum Füllmodus$^{\textcolor{blue}{\star}}$]  {\bf }:  Hier kann definiert werden, wie die Lufträume gefüllt werden.\\
   {\bf Voreinstellung}: Sucht anhand der Geräteperformance automatisch die beste Art der Darstellung heraus.
    Normalerweise wird hier --bis auf das PPC2000-Betriebsystem-- der Füllmodus ''Rahmen'' bevorzugt.\\
    {\bf Ganz ausfüllen}:  Transparently fills the airspace colour over the whole area. \\
    {\bf Rahmen}: Zeichnet einen satten Rahmen um den halbtransparenten Luftraum. \\
\item[Luftraumtransparenz$^{\textcolor{blue}{\star}}$]  {\bf Ein /Aus}: Wenn eingeschaltet, dann werden Lufträume transparent dargestellt.
\end{description}

Auf dieser Seite gibt es am unteren Rande zwei weitere Schaltflächen  \button{Filter} und
 \button{Farben} mit denen für jede einzelne Luftraumklasse Farbe, Rahmen und Muster
 eingestellt werden kann.
 Außerdem ist möglich in den Filtereinstellungen die Sichtbarkeit zu regulieren/festzulegen.
Abhängig von der Hardware kann es sein, daß nicht alle Transparenzeigenschaften gezeigt
werden können.

\subsection*{Farben}
Mit dieser Funktion wird die Farbe, Rahmen des Luftraumes,  Dicke des Rahmens und
anderes eingestellt.

Zuerst einen Luftraum auswählen, dann doppelkicken und mit \button{Farbe der Umrandung}  bzw.
\button{Füllfarbe ändern} die gewünschten Eigenschaften zuordnen.

\subsection*{Filter}
Die Filterfunktion ist beschrieben in Kap.~\ref{sec:airspace-filter}.

%%%%%%%%%%%%%%%%%%
\section{Endanflugrechner/Sicherheitsfaktoren}
Hier werden Einstellungen für die Sicherheitshöhen und deren Verhalten in den Berechnungsalgorythmen vorgenommen.

\begin{description}
\item[Ankunftshöhe] Sicherheitshöhe. \index{Sicherheitshöhe!Einstellung} Diese  gibt die Höhe über Grund am Zielpunkt an, in der das
Flugzeug für eine sichere Landung ankommen sollte. Gesetzt auf Null bedeutet, daß man am Zielplatz in Null Meter Höhe ''aufschlägt''.
\item[Geländefreiheit]  \label{conf:safetyterrain} Sicherheitshöhe. Eine einstellbare Höhe, die das Flugzeug während des Endanfluges
nicht unterschreiten sollte. Vermeidet damit ''in den Berg fliegen'', da \xc dann vorschlägt weiter zu Kurbeln. (Man kann natürlich auch
drumherum fliegen\dots mit dem Rotenplaner\dots)\\
\todonum{nur im Endanflug? - nicht im cruise-mode??}
Mehr zu Sicherheitshöhen, deren Bedeutung und Definition  im Kapitel~\ref{sec:safety-heights}.
\item[Alternativen Modus]  \label{conf:alternatesmode} Bestimmt, wie beim Abbruch einer Aufgabe die Wegpunktalternativen
innerhalb des Wegpunktdialoges sortiert und vorgeschlagen werden. \\
{\bf Einfach}: Die Alternativen werden nur Ankunftshöhe sortiert ausgegeben. Der erste Punkt auf der Liste ist der nächstliegende
und preferierte.\\
{\bf Aufgabe}: Die Sortierung berücksichtigt  zusätzlich die Kursrichtung, sodaß der nächstgelegene Platz in Kursrichtung
preferiert wird. \\
{\bf Heimat}:  Ähnlich wie die ''Aufgabe''-Sortierung, aber  mit dem \textsl{Heimatplatz} als Bezugskurs.   Diese Sortierung berücksichtigt
auch Wegpunktalternativen in Richtung des konfigurierten \textsl{Heimatplatzes}  - welcher \textsl{nicht unbedingt der Startpunkt} sein muß \achtung
\item[Polaren Verschlechterung$^{\textcolor{blue}{\star}}$] Einstellung für eine permanente Verschlechterung der Polare. Einstellung auf $0 \% $ bedeutet keine Verschlechterung,
$50 \% $ bedeutet, daß die Sinkrate \textbf{verdoppelt} ist (!)\index{Polare!Auswirkung des Faktors}
\item[Sicherheits MC$^{\textcolor{blue}{\star}}$]  Wenn Sicherheits MC aktiviert ist, dann wird dieser Wert \textit{nach einem Abbruch einer Aufgabe} für die Kalkulation von Reichweite, Alternativen und Ankunftshöhen an Flugplätzen und Landefeldern benutzt.\label{safety-MC} 

\item[Vorflug Risikofaktor$^{\textcolor{blue}{\star}}$]
 Der Vorlfugrisikofaktor reduziert den MC-Wert für die Vorfluggeschwindigkeit mit fallender Flughöhe um das Risko einer
 Außenlandung zu mindern. Wenn zu $0,0$ gesetzt, bedeutet dies keine Anpassung an die aktuelle Flughöhe, gesetzt auf $1,0$
wird der MC Wert linear bezogen auf die aktuelle Höhe reduziert, als Bezugspunkt gilt dabei die maximal während des
Fluges erreichte Höhe.
Ein Wert von ca. $0.3$ ist empfohlen.  Mehr dazu auch in Kap.~\ref{sec:speed-fly-with}.
\end{description}

%%%%%%%%%%%%%%%%%%
\section{Endanflugrechner / Endanflugrechner}\label{sec:final-glide}
Über diese Seite können die Routinen des Endanflugrechners von \xc  beinflußt werden.

\begin{description}
\item[MC Optimierung]  Diese Option setzt fest, welcher MC-Algorythmus verwendet wrd.
Mehr dazu in Kap.~\ref{sec:auto-maccready}. \\
   {\bf Endanflug}: Stellt den MC Wert auf die schnellstmögliche Ankunft ein.
   Bei OLC - Aufgaben werden die Einstellungen  so gewählt, daß \sketch{figures/config-glidecomputer.png} eine möglichst große Strecke innerhalb der verbleibenden Zeit zurückgelegt wird.\\
   {\bf Erwartetes mittleres Steigen}: Die MC Einstellungen basierend auf den bisher ermittelten Steigwerten gesetzt.\\
   {\bf Beides}: Benutzt ''Erwartetes mittleres Steigen'' während der Aufgabe und wechselt im Endanflug automatisch auf  ''Endanflug''.
\item[Blockgeschwindigkeit$^{\textcolor{blue}{\star}}$] {\bf Ein / Aus}: Wenn eingeschaltet, wird die Sollfahrt für den Vorflug empfohlen, welche für Gleiten ohne vertikale Luftbewegung gilt.
Andern falls werden die Sollfahrten nach dem Delphin-Stil vorgeschlagen, es wird dann die vertikale Luftbewegung mit in die Sollfahrtberechnung
einbezogen.
\item[Nav. mit barometrischer Höhe$^{\textcolor{blue}{\star}}$] {\bf Ein / Aus}: Wenn eingeschaltet und ein barometrischer Höhenmesser angeschlossen ist, dann wird die barometrische Höhe für alle Navigationsfunktionen verwendet. Ansonsten wird die GPS--Höhe benutzt.
\item[Wechsel Steig-/Vorflugmodus$^{\textcolor{blue}{\star}}$] {\bf Ein / Aus}: Wenn ein Vega Variometer angeschlossen ist und diese Option aktiviert ist, wird mit der Wölbklappe zwischen Kreis- und Vorflug umgeschaltet.
Das bedeutet, daß  beim Umwölben sofort der entsprechende Modus aktiviert wird (auch beim Borgelt B50 verfügbar)
\item[Gleitzahl Mittelwertzeitintervall$^{\textcolor{blue}{\star}}$] {\bf sec, min}: Hier kann entscheiden werden, welche Zeitspanne für die Mittelwertbildung benutzt werden soll.
Die Mittelwertbildung wird grundsätzlich in Echtzeit vorgenommen. Berechnet wird das erreichte Höhen zu Streckenverhältnis
innerhalb der hier ausgewählten Zeitspanne.

Wenn Du zum Beispiel wegfliegst und nach zwei Minuten zum gleichen Punkt zurückkehrst und Du hast eine Zeitspanne von zwei Minuten eingestellt,
dann nimmt der Integrator die Strecke innerhalb dieser zwei Minuten annehmen, welche aber dann Null (Punkt A = Punkt B $\Rightarrow \bar{AB} = 0$ )
ist und somit wird die Gleitzahl auch Null.

Normalerweise ist für Segelflugzeuge ein Wert von 90 - 120 sec.\ zu bevorzugen,
für Paragleiter sind 15 sec.\ OK. Kleinere Werte werden als Ergebnis immer näher an den
aktuellen Wert herankommen, größere Werte immer weiter an den Wert gemittell über den gesamten Flug.
(Klar, bei einer Mittelwertbildung\dots) Andere kommerziell erhältliche Instrumente benutzen Werte von ca. 120 Sekunden.
\item[Geschätzte Winddrift$^{\textcolor{blue}{\star}}$]  {\bf Ein / Aus}: Berücksichtigt die Abdrift durch den Wind während des Kurbelns. Dies wird die Ankunftshöhe auf Gegenwindschenkeln reduzieren
und kann mitunter zu interpretationswürdigen Ergebnissen führen,
wenn man sich im Endanflug unterhalb des Gleitpfades befindet, MC auf Null gestellt hat, und so also noch kurbeln \textsl{muß}.
Nach der MC-Theorie heißt MC=0, daß kein kurbeln mehr stattfindet. Der Algorythmus befindet sich so in einer Zwickmühle, denn einerseits wird mitgeteilt, daß nicht
mehr gekurbelt wird, andererseits tut man es doch\dots
\achtung Für das ''normale, althergebrachte Verhalten'' anderer Systeme sollte man sich überlegen, dies auf ''Aus'' stehen zu lassen.
\end{description}

%%%%%%%%%%%%%%%%%%
\section{Endanflugrechner / Wind} \label{sec:final-wind}
Auf dieser Seite werden die Windeinstellungen vorgenommen. Details zu den verschiedenen Berechnungen in Kap.~\ref{sec:wind-estimation}
\begin{description}
\item[Windberech]  \label{conf:autowind}  Auswahl des entsprechenden modus für Ermittlung des Windes in \xc \\
  {\bf Manuell}: Wenn der Algorythmus ausgeschaltet ist, wird allein die Eingabe des Piloten benutzt. \\
  {\bf beim Kreisen}: Benutzt das GPS zur Ermittlung des Windes durch Berechnung der Abdrift beim Kurbeln.\\
  {\bf ZickZack}: ZickZack-Modus benötigt ein intelligentes Vario  mit einem Ausgang für die TAS.\\
  {\bf Beides}:  Benutzt den ZickZack und GPS-Modus zur Ermittlung des Windes.
\item[Externer Wind]  Wenn eingeschaltet und ein entsprechendes gerät am \xc angeschlossen ist, dann wird der Wind von einem
externen Gerät in die internen Routinen übernommen.
\end{description}


%%%%%%%%%%%%%%%%%%
\section{Endanflugrechner / Routenplaner}\label{sec:final-route}
Seite für die Einstellungen für die Reichweite und Routenplaner-Optimierungen.

\begin{description}
\item[Routenplaner Modus]  \label{conf:routemode} {\bf Keine, Gelände, Luftraum, Beides}
  Einstellung, welcher Modus für die Routenplanung benutzt werden soll. Für Details bitte in Kap.~\ref{sec:route} nachschauen.
\item[Pfadsuche mit Steigen$^{\textcolor{blue}{\star}}$]  {\bf Ein / Aus}: \label{conf:routeclimb} Wenn dies aktiviert und MC größer Null, dann erlaubt der Routenplaner
  Steigphasen zwischen der aktuellen Position und dem Ziel.
\item[Wolkenuntergrenze$^{\textcolor{blue}{\star}}$]  \label{conf:routeceiling} Wenn aktiviert wird bei der Routenplanung  das Steigen durch die Wolkenbasis begrenzt. Sie wird mit 500 m über der momentanen Flugzeughöhe und der Thermikhöhe angenommen.
  Wenn nicht aktiviert wird mit unbegrenztem Steigen geplant.
\item[Reichweiten Modus]  \label{conf:turningreach} Die Reichweitenberechnung kann Umwege durch vorhandene Hindernisse berücksichtiegn.
  Hier werden die Einstellungen dazu vorgenommen:\\
  {\bf Aus}: Reichweite wird nicht berechnet.\\
  {\bf Direkt}: Reichweite wird in geradem Weg vom Flugzeug aus berechnet.\\
  {\bf Mit Umweg}: Die Reichweitenberechnung wird unter Berücksichtigung von Hindernissen vorgenommen.
\item[Erreichbare Anzeige]  \label{conf:gliderange} Einstellung, wie die Reichweite auf der moving map dargestellt wird.\\
  {\bf Aus}: Keine Darstellung des Gleitbereiches.\\
  {\bf Grenzlinie }:Zeichnet eine gestrichelte linie um den aktuell erreichbaren Bereich.\\
  {\bf Schattiert }: Verdunkelt das Gelände außerhalb der aktuellen Reichweite etwas
\item[Erreichbare Polare$^{\textcolor{blue}{\star}}$]  \label{conf:reachpolar}
   Einstellung der Gleiteigenschaften des Flugzeuges für die Berechnung von Reichweite, Ankunftshöhen, Aufgabenabbruch und alternativen Landeplätzen.\\
  {\bf Aufgabe}: Verwendet die Polare aus der Aufgabe\\
\todonum{welche Polare - die ermittelte, oder eingegebene?}
  {\bf Sicherheits MC}: Verwendet den eingegebenen Sicherheits-Mc-Wert.
\end{description}


%%%%%%%%%%%%%%%%%%
\section{Anzeigen / FLARM, etc.} \label{sec:flarmandother-gauge}

\begin{description}
\item[FLARM Radar]  \label{conf:flarmdisplay} Anzeige der \fl -Rose auf dem Kartenddisplay.\\
{\bf Ein}: Aktiviert die Anzeige des FLARM Radars. Die Flugrichtung des Ziels relativ zum eigenen Steuerkurs wird als Pfeilkopf dargestellt.
Ein Dreieck, das nach oben oder nach unten zeigt die Höhe des Zieles relativ zur Eigenen an.
\sketch{figures/flarmrose.png}\\
{\bf Aus}: Flarmanzeige des Luftverkehrs  bleibt aus.
\item[Schließe FLARM automatisch$^{\textcolor{blue}{\star}}$] {\bf Ein / Aus}: \\
{\bf Ein}: Eingeschaltet, wird der FLARM Dialog automatisch geschlossen, sobald kein Verkehr da ist.\\
{\bf Aus}: Beläßt den Dialog offen, auch wenn kein Verkehr mehr da ist.
\item[Zentrierhilfe] \label{conf:thermalassistant} {\bf Ein / Aus}: Aktiviert die Anzeige des Thermikassistenten (Zentrierhilfe).
\item[Thermikprofil] \label{conf:thermalband} {\bf Ein / Aus}: Aktiviert die Anzeige des Thermikprofils über der Karte.
\item[Endanfluganzeige für MC 0$^{\textcolor{blue}{\star}}$] {\bf Ein / Aus}:
Wenn aktiviert zeigt der Endanflug Anzeigebalken einen zusätzlichen, zweiten Pfeil für die bei MC 0 Einstellung benötigte Höhe um das Ziel zu erreichen.
\end{description}

In allen Modi deutet die Farbe des Zieles auf den Level der potentiellen Gefahr an (grün, gelb, rot\dots)


%%%%%%%%%%
\section{Anzeigen / Vario}\label{sec:vario-gauge}

Hier können \textsl{dezimale Anzeigen} auf dem Vario-Anzeige-Bereich  ein- und ausgeschaltete werden.
Mithilfe des Zeigers (ganz unten) kann auch ein Mittelwert-Zeiger an oder ausgeschaltet werden,
\index{Vario!Elemente} diesen auf der analogen Ringdarstellung zu platzieren.\index{Vario!Aussehen}
\begin{description}
\item[Geschwindigkeitspfeile$^{\textcolor{blue}{\star}}$]  \label{conf:variogauge} Ob Pfeile für die Sollfahrt  in der Varioanzeige dargestellt werden sollen.\\
{\bf Ein}: Wenn aktiviert weisen im Vorflugmodus  Aufwärtspfeile zum langsamer Fliegen (Ziehen--nach oben ) an,  Abwärtspfeile zum schneller Fliegen an (Drücken--nach unten).
{\bf Aus}:  KeineAnzeige.\\
\item[Zeige Mittelwert$^{\textcolor{blue}{\star}}$]  {\bf Ein / Aus}: Ob das mittlere Steigen angezeigt werden soll.
Im Vorflug wird auf mittlere Nettovario Anzeige geschaltet.
\item[Zeige MacCready$^{\textcolor{blue}{\star}}$] {\bf Ein / Aus}: Anzeige des  MacCready-Wertes  ein oder aus.
\item[Zeige Mücken$^{\textcolor{blue}{\star}}$] {\bf Ein / Aus}: Anzeige der Mückenbeladung in Prozent.
\item[Zeige ballast$^{\textcolor{blue}{\star}}$] {\bf Ein / Aus}:Anzeige des Wasserballast ein oder aus. Anzeige in Prozent.
\item[Zeige brutto$^{\textcolor{blue}{\star}}$] {\bf Ein / Aus}: Anzeige des Bruttosteigens ein oder aus.
\item[Mittelwert Zeiger$^{\textcolor{blue}{\star}}$] {\bf Ein / Aus}: Falls aktiviert ist in der Varioanzeige eine hohle Mittelwertnadel zu sehen.
Im \textsl{Vorflug} zeigt diese Nadel das gemittelte Nettovario,
beim \textsl{Kurbeln} wird das mittlere Bruttovario angezeigt.
\todonum{Besser als Integrator bezeichnen??}
\end{description}

%%%%%%%%%%
\section{Anzeigen / Audio-Vario}\label{sec:audiovario-gauge}

Hier werden Einstellungen für die Tonausgabe des Varios vorgenommen.\index{Vario!Tonausgabe}
\label{conf:audiovariogauge}

\begin{description}
\item[Tonvario]  {\bf Ein / Aus}:  Ein oder Ausschalten der Tonausgabe des Varios
\item[Volumen]  Einstellung der Laustärke.
\item[Tonausblendung Ein]  {\bf Ein / Aus}: Unterdrückung der Tonausgabe, wenn das TEiegn (oder Sinken) innerhalbeines betimmten Wertes um Null ist.
\item[Niedrigster Ton$^{\textcolor{blue}{\star}}$]  Frequenz des Tones bei größtem Saufen.
\item[Null-Ton$^{\textcolor{blue}{\star}}$]  Frequenz des Tones bei Nullschieber.
\item[Höchster Ton$^{\textcolor{blue}{\star}}$]  Frequenz des Tones bei größtem Steigen.
\item[Tonausblendung min. Steigen$^{\textcolor{blue}{\star}}$]  Wenn die Tonausblendung aktiviert ist, gibt das Vario unterhalb dieses Grenzwertes Töne aus.
\item[Tonausblendung max. Steigen$^{\textcolor{blue}{\star}}$]  Wenn die Tonausblendung aktiviert ist, gibt das Vario oberhalb dieses Grenzwertes für das Steigen Töne aus.
\end{description}


%%%%%%%%%%%%%%%%%%
\section{Aufgabenvoreinstellungen  / Regeln}

Auf dieser Seite werden Regeln für Aufgaben global voreingestellt.
rules. \label{conf:taskrules}\index{Aufgabe!Regeln Voreinstellung}

\begin{description}
\item[Max. Abfluggeschwindigkeit$^{\textcolor{blue}{\star}}$] Einstellung der maximalen Abfluggeschwindigkeit über die deklarierte Abfluglinie.
   Wenn kein Limit vorgegeben ist, sollte der Wert auf $0$ gesetzt werden.
\item[Max. Abflug-Geschwindigkeitstoleranz$^{\textcolor{blue}{\star}}$] Maximal tolerierte Geschwindigkeitstoleranz der vorgegebenen Abfluggeschwindigkeit.
   Für Null-Toleranz zu $0$ setzen.
\item[Maximale Abflughöhe$^{\textcolor{blue}{\star}}$]  Maximale Abflughöhe bei Überfliegen der Abfluglinie.
   (MSL oder GND möglich--s.u.) Für Start ohne Höhenbegrenzung zu $0$ setzen.
\item[Max. tolerierte Abflugüberhöhung$^{\textcolor{blue}{\star}}$]  Maximal tolerierte Höhe über der vorgegebenen Abflughöhe
   Wenn keine Toleranz erlaubt, zu $0$ setzen.
\item[Abflughöhe bezogen auf$^{\textcolor{blue}{\star}}$]  Bezugshöhe der max. Abflughöhen:\\
    {\bf MSL}: Abflughöhe ist bezogen auf Meereshöhe.\\
    {\bf AGL}: Abflughöhe ist bezogen auf Höhe über Grund.
\item[Min. Ankunftshöhe$^{\textcolor{blue}{\star}}$]  Minimale Ankunftshöhe zum Überqueren der Ziellinie / Zielkreises
    (AGL or MSL). Wenn keine Abweichung erlaubt ist, zu $0$ setzen.
\item[Zielhöhenref.$^{\textcolor{blue}{\star}}$] Bezugshöhe der Ankunftshöhe:\\
    {\bf MSL}: Ankunftshöhe ist bezogen auf Meereshöhe.\\
    {\bf AGL}: Ankunftshöhe ist bezogen auf Höhe über Grund.
\item[On-Line Contest]  Einstellungen zu den Optimierungsalgorythmen des OLC.
   Die Implementierung enstpricht der ofiziellen Ausgabe vom 23.September 2010. \\
   {\bf OLC FAI}: Gleichlautend zu den FAI-Dreiecksregeln. Zwei Wenden, wobei Star-und Zielpunkt identisch sind.
   Kein Schenkel darf kürzer sein als 28\% der Gesamtstrecke. Bei Dreiecken größer als 500Km
   darf kein Schenkel größer sein als 25\% der Gesamtstrecke und länger als 45\% andernfalls kein
   Schenkel kleiner als 28\%. Ankunftshöhe darf nicht  tiefer sein als Starthöhe -1000m.\\
   {\bf OLC Classic}: Bis zu sieben Punkte einschließlich Start und Ziel, Ankunftshöhe nicht
   mehr als 1000m unter der Abflughöhe. \\
   {\bf OLC League}: Der aktuelle Contest mit Sprint-Wettbewerbsregeln.
   Aus dem OLC Classic Dreieck wird das 2,5 Stunden Fenster mit dem besten Schnitt herausgschnitten.
   Ankunftshöhe darf nicht unter Abflughöhe liegen.\\
  {\bf OLC Plus}: Kombination aus Classic und FAI Regeln. 30\% der FAI Punkte werden zur
   Classic Wertung dazugezählt \\
  {\bf XContest}: \todonum{noch zu schreiben} \\
  {\bf DHV-XC}: \todonum{noch zu schreiben} \\
  {\bf SIS-AT}: \todonum{noch zu schreiben} \\
  {\bf FFVV NetCoupe}: Der FFVV NetCoupe {\it ''libre''} Wettbewerb.
\item[Wertungs-Vorhersage]  {\bf Ein / Aus}: Wenn eingeschaltet, wird angenommen, daß der nächste Wegpunkt erreicht wird und
die daraus sich ergebende Wertung errechnet.
\end{description}


%%%%%%%%%%%%%%%%%%
\section{Aufgabenvoreinstellungen  / Wendepunkttypen}

Einstellungen für Start, Ziel und Wendepunkte können hier global konfiguriert werden.
Diese Werte werden bei Erstellung einer neuen Aufgabe als Standard genommen.
Die Feineinstellung  ist im Aufgabeneditor vorzunehmen.
Mehr dazu in Kap~\ref{cha:tasks}. \index{Aufgabe!Regeln!Voreinstellungen}


\begin{description}
\item[Startpunkt] Standard Einstellungen für neu zu erstellende Aufgaben.\\
{\bf  Abfluglinie}: Ein geradliniges Tor. Wird auf der Map mit einem Halbkreis entgegengesetzt zum ersten Schenkel
dargestellt. Überquere diese Linie entsprechend der Aufgabenvorgaben aus dem Sektor heraus.\\
{\bf  Abflugzylinder}: Ein Zylinder um den Startpunkt herum. Von innen nach außen überfliegen.\\
{\bf FAI-Abflug-Quadrant}: Ein 90$^\circ$ Sektor mit 1km Radius. Überfliege de Flanken vom inneren her für den Abflug.\\
{\bf  BGA Abflugsektor}:Ein 180$^\circ$ Sektor mit 5km Radius. Verlasse den Sektor in irgendeine Richtung für den Abflug.
\item[Torbreite] Standard Radius oder Torbreite in Km für neu zu erstellende Aufgaben.
\item[Zielpunkt] Standard Ziel-Typ für neu zu erstellende Aufgaben.\\
{\bf Ziellinie}: Überfliege diese Linie nach Vorgabe der WB-Leitung.\\
{\bf Zielzylinder}: Fliege in diesen Zylinder.\\
{\bf FAI Zielquandrant}: Ein 90$^\circ$ Sektor mit 1Km Radius. Überfliege die flanken für gültigen Einflug.
\item[Radius] Standard Radius oder Torbreite in Km für neu zu erstellende Aufgaben.
\item[Wendepunkt] Standard Wendepunkt-Typ für neu zu erstellende Aufgaben.\\
{\bf Wendepunkt-Zylinder}: Ein Zylinder. Ein Punkt im Zylinder gilt als gültig, gewertet wird der Mittelpunkt\\
{\bf Schlüssellochsektor (DAEC)}: Jeder Punkt innerhalb eines 500m Zylinders vom Zentrum oder 10Km innerhalb
des 90$^\circ$ Sektors. Gewertet wird der Mittelpunkt.(deustch)\\
{\bf BGA festgelegter Kurssektor}: Jeder Punkt innerhalb 500m Zylinder, oder 20Km  in einem 90$^\circ$ Sektor.
Gewertet wird das Zentrum des Zylinders.(britisch)\\
{\bf BGA Erweiterungsoption festgelegter Kurs}:Jeder Punkt innerhalb 500m Zylinder, oder 10Km  in einem 180$^\circ$ Sektor.
Gewertet wird das Zentrum des Zylinders.(britisch)\\
{\bf FAI Quadrant}: Ein 90$^\circ$ Sektor mit unendlicher Seitenlänge. Überfliege eine Flanke. Gewertet wird vom Eckpunkt.
\item[Radius] Standard Radius oder Torbreite in Km für neu zu erstellende Aufgaben.
\item[Aufgabe] Standard Aufgaben-Typ für neu zu erstellende Aufgaben.\\
{\bf Racing }:Geschwindigkeitsaufgabe um Wendepunkte. Kann für FAI Abzeichen oder auch Rekordversuche  verwendet werden. Alle OZ-Typen sind möglich.\\
{\bf  AAT}: Geschwindigkeitsaufgabe über zugeweisene Gebiete. Mindestflugzeit wird vorgegeben. Beschränkt auf Zylinder und Sektorgebiete. Gewertet wird bester Schnitt.\\
{\bf  FAI Abzeichen / Rekorde}: FAI-Regeln. Erlaubt nur FAI Start-, Ziel- und Wendepunkttypen. Für Abzeichen und Rekorde. Aktiviert die FAI Höhe für die Endanflugbnerechnung.
\item[AAT min. Zeit] Die bei AAT-Aufgaben mindestens zu fliegende Zeit. Vorgabe durch WB-Leitung.
\item[Optimierungs-Spielraum$^{\textcolor{blue}{\star}}$] Sicherheitszeitraum für AAT-Optimierung.
Als Schutz gegen ''zu früh ankommen''. Der Optimierungsalgorythmus versucht, die Aufgabe innerhalb der vorgegebenen Mindestflugzeit plus dieser Zusatzzeit zu bneenden.
Im ''Optimiert''-Modus  werden dabei die AAT-Punkte automatisch verschoben, um diese Zeit einhalten zu können..
\end{description}

%%%%%%%%%%%%%%%%%%
\section{Aussehen / Sprache, Eingaben}\label{sec:interface}

Auf dieser Seite können Interaktionen und Aussehen der Menüs von \xc beeinflußt werden.


\begin{description}
%\item[Auto Blank\textcolor{blue}{$\star$}]  This determines whether to blank the display after a long
%  period of inactivity when operating on internal battery power (visible for some mobile
%  devices only).
\item[Ereignisss$^{\textcolor{blue}{\star}}$]  Die Eingabeereignisdatei bestimmt, wie \xc auf bestimmte Eingaben wie z.B.\ einen Knopfdruck oder einer
Eingabe eines externen Gerätes reagiert.
\item[Sprache]  Die Sprache (2 stellige Abkürzungen als Auswahl) der Oberfläche -- übersetzt aus dem Englischen.
{\bf Automatisch} wählt die Sprache anhand der Betriebssystemeinstellung aus.
\item[Nachricht$^{\textcolor{blue}{\star}}$] Die Statusdatei legt fest, welche Klänge auf welche Ereignisse hin abgespielt werden
und wie lange die verschiedenen Meldungen auf dem Bildschirm sichtbar bleiben.
\item[Menü Zeitabschaltung$^{\textcolor{blue}{\star}}$] Diese Zeit bestimmt, wie lange ein Menü ohne weitere Eingabe des Benutzers auf
dem Bildschirm sichtbar bleibt, bevor es wieder ausgeblendet wird.
\item[Texteingabestil$^{\textcolor{blue}{\star}}$]  Auswahl, wie Text eingegeben werden kann. Es stehen drei Möglichkeiten zur Auswahl:
  Siehe hierzu Kap.~\ref{sec:textentry} für mehr Info dazu. \\
    {\bf Voreinstellung}: Die Standardeinstellung des Betriebsystemes des Gerätes wird benutzt.\\
    {\bf Tastatur}: Benutzt eine Bildschirmtastatur  für die Eingabe.  \\
    {\bf Ranglisten Stil}: Auswahl von Buchstaben durch unterstrichenen blinkenden Cursor.
\item[Haptische Rückmeldung$^{\textcolor{blue}{\star}}$]  {\bf Ein / Aus}: (Nur bei Android Geräten) Schaltet die Vibration des Gerätes bei Berührung der TouchScreen-Oberfläche  ein oder aus. Das ist das ''\texttt{brrt..}''\dots
\end{description}



Um die Schriften von \xc anzupassen, drücke auf den hier vorhandenen  \button{Schriften} Knopf.

\subsection*{Schriften}\index{Schriften!Anpassen}

Auswahl der verschiedenen Schriften für diverse Anzeigeelemente. Anwahl des beschriebenen
Elementes und Auswahl der dazu angebotenen Schrift, -größe und -format.

\sketch{figures/config-fonts.png}

Falls nicht vom Benutzer konfiguriert, werden voreingestellte Standardschriften verwendet.

%%%%%%%%%%%%%%%%%%
\section{Aussehen / Anordnung}\label{sec:interface-appearance}

Auf dieser Seite können weitere Details zum look and feel der Oberfläche
angepaßt werden.


\begin{description}
\item[InfoBox Geometrie]  Eine Auswahlliste von mehreren Anordnungen der InfoBox Seiten Je nach Gerät und persöhnlichem
Geschmack muß dies schlicht ausprobiert werden. Die angebene Zahl entspricht der Anzahl der InfoBoxen auf der entsprechenden InfoBoxSeite.
\item[FLARM Anzeige$^{\textcolor{blue}{\star}}$]  \label{conf:flarmradar-place}
Wen das \fl angeschaltet wird, kannst Du das kleine \fl Radar es hiermit an entsprechenden stellen einer InfoBoxSeite plazieren.
Als Standard ist ein ''Auto''-Setup vorhanden, was bedeutet, daß das Radar lediglich die InfoBoxen
überlappt und nicht einen Teil der MovingMap.
\item[Reiter Dialogstil] Einstellung, ob Icons oder Text auf beschrifteten Seiten angezeigt werden sollen.
Bspw. im Aufgaben Editor.
\item[Nachrichtenanzeige$^{\textcolor{blue}{\star}}$]  Einstellung, wo Nachrichten und Systemmeldungsfenster
erscheinen sollen.   {\bf Zentriert} oder in der {\bf Oben Links} Ecke.
\item[Dialogfenstergröße$^{\textcolor{blue}{\star}}$]  Einstellung über die Größe der Dialogfenster{\bf Volle Breite},  {\bf Skaliert},  {\bf Skaliert und zentriert} und {\bf fixiert} stehen zur Auswahl.
\item[Invertierte InfoBox$^{\textcolor{blue}{\star}}$]  Wenn {\bf Ein}, werden die InfoBoxen weiß auf schwarzem Hintergrund, wenn {\bf Aus} schwarz auf weißem Hintergrund dargestellt.
\item[Farbige InfoBoxen$^{\textcolor{blue}{\star}}$]  Wenn {\bf On}, haben manche InfoBoxen farbige Inhalte. Wenn {\bf Aus}, erfolgt die Darstellung in Schwarz/Weiß.
Zum Beispiel wird der aktive Wegpunkt-InfoBox  blau dargestellt, wenn sich das Flugzeug oberhalb des Gleitpfades für den Endanflug  befindet,
\item[InfoBox Rahmen$^{\textcolor{blue}{\star}}$]  Es gibt zwei Arten, die InfoBoxen zu umrahmen:\\
  {\bf Kasten}: Zeichnet einen Kasten um jede InfoBox.\\
  {\bf Tab}: Zeichnet einen Reiter über den Titel der InfoBox
  title. \\
\end{description}


%%%%%%%%%%%%%%%%%%
\section{Aussehen / InfoBox Seiten (oder einfache Seiten)}

Mithilfe dieser Einstellungen kann das Aussehen und die Belegung des InfoBoxenSeiten angepaßt werden.

Es können bis zu acht InfoBoxSeiten konfiguriert und benutzt werden, da.h. es können maximal
8x8 = 64 verschiedene Infoboxen während des Fluges betrachtet werden (Ob das sinnvoll ist, sei dahin gestellt).
Eine InfoBoxSeite ist nichts anderes als eine feste Tabelle mit in verschiedenen Anordnungen eingefügten Infoboxen.

Es gibt drei vorangepasste Seiten für das Kurbeln, den Vorflug, den Endanflug eines Seite welche
nur eine Karte ganz ohne InfoBoxen zeigt und eine Automatik- InfoboxSeite, welche je nach Flugmodus
(Vorflug/Kurbeln) automatisch vom System aktiviert wird.

\begin{description}
\item[Seite  1..3] Wähle aus, was Du meinst, was auf Deinen Seiten stehen sollte.
       Auswahl von  ''- - -'' bedeutet, daß die Seite inaktiv bleiben wird.
\item[Seite 4..8$^{\textcolor{blue}{\star}}$]  ''Experten'' können hier weitere Seiten für Ihre Zwecke definieren
\end{description}




%%%%%%%%%%
\section{Aussehen / InfoBox Modi (und InfoBox-Sets)}\label{sec:infobox_sets}

Auf dieser Seite werden InfoBoxen-sets erstellt, geändert und konfiguriert.
Im ''Experten'' Modus können bis zu acht Sets von InfoBoxen bearbeitet werden,
in der Basis - Konfiguration lediglich vier.

Hier werden die entsprechenden InfoBoxen den entsprechenden Positionen innerhalb
der InfoBoxSeiten zugeordnet. Doppelklick auf die entsprechenden Positionen öffnet die mögliche Auswahl.

Es gibt drei vorbereitete Sets von InfoBox-Seiten
bzw. InfoBox-Seiten, welche ''beim Kurbeln'', ''Vorflug'' und ''Endanflug'' benannt sind.

\begin{description}
\item[beim Kreisen]  vorbereitete Seite gedacht für das Kurbeln.\\
\item[Vorflug] vorbereitete InfoBoxSeite, vorbereitet für Vorflug\\
\item[AUX-4--5]Frei zur Verfügung stehenden InfoBoxSeiten.
\end{description}

Es können bis zu fünf weitere InfoBoxSeiten erstellt werden, welche
standardmäßig ''AUX4',''AUX5''\dots benannt sind.

Mit \button{Kopieren} ist es möglich den Inhalt der aktuellen InfoBoxSeite auf eine nächste zu übertragen.

Nach dem Anklicken öffnet sich die entsprechende Seite und kann mit den diversen InfoBoxen bestückt
und mit einem beliebigen Namen benannt werden (z.B.\ AAT-Planung) werden

\button{Endanflugmodus verwenden}  Bestimmt, ob das ''Endanflug'' InfoBox-Set für die
 ''Auto''-Seite benutzt werden soll.
%
\subsection*{InfoBox Seiten-Anpassung}
%
\begin{description}
\item[Name]  Hier kann ein Name für die gerade erstellte oder bearbeitete InfoBoxSeite angegeben werden.
\item[InfoBox]  Nummer der entsprechenden Infobox innerhalb der InfoBoxSeite. (s. Tabelle weiter unten)
\item[Inhalt]  Hier steht eine Beschreibung des aktuellen Inhaltes der schwarz hinterlegten infoBox.
\end{description}


\begin{center}
\includegraphics[angle=0,width=0.65\linewidth,keepaspectratio='true']{figures/XCS64-Creating-New-InfoBoxSeite.png}
\end{center}
%\menulabel{XCS64-Creating-New-InfoBoxSeite.png}

Schwarz hinterlegt ist hier die aktuell angeklickte Position der Infobox, entweder Doppelklick auf das schwarze Feld, um
den Inhalt festzulegen, oder aber anhand der vorgegebenen Numerierung vorgehen. Mehr Detail in Kap.~\ref{cha:infobox} für die genaue Beschreibung der InfoBoxen und deren Inhalt.
Die Tabelle unten zeigt einen kurzen Überblick über die Positionen der InfoBoxen im Hoch - und Querlayout.
\begin{center}\begin{multicols}{2}
\begin{tabular}{|c|c|}
\hline
1 & 7 \\
\hline
2 & 8 \\
\hline
3 & 9 \\
\hline
4 & 10 \\
\hline
5 & 11 \\
\hline
6 & 12 \\
\hline
\end{tabular}

\begin{tabular}{|c|c|c|c|}
\hline
1 & 2 & 3 & 4 \\
\hline
\hline
5 & 6 & 7 & 8 \\
\hline
\end{tabular}
\end{multicols}
\end{center}
%%%%%%%%%%%%%%%%%%
\section{Einstellung / NMEA-Anschluß} \label{conf:comdevices}\index{NMEA-Anschlüsse}\index{GPS!Liste der Geräte}\index{GPS!Treiber}

Auf dieser Seite werden die Anschlüsse der externen Geräte insbesondere der GPS-Quelle definiert und deren Parameter
angegeben. Falsche Einstellungen an dieser Stelle werden zu Fehlfunktionen führen. Es stehen ein große Menge an Gerätetreibern zur Verfügung, die aus der Auswahlliste am
oberen Rande des Fensters ausgewählt werden können.

Je nachdem, welches Gerät ausgewählt wurde, erscheinen in der unteren Hälfte des Fensters dazugehörige
Parameter zur Auswahl. \achtung Es werden daher hier alle Parameter gezeigt, unabhängig davon, ob sie bei manchen Geräten nicht erscheinen.
Die Liste ist in permanentem Wandel/Wachstum, daher kann es sein, daß nicht alle Gerät hier mit dediziertem Treibern aufgelistet sind.
Wenn das Gerät mit dem Vega verbunden werden soll, ist als Anschluß COM1 und 38400 baud zu wählen.

Standard ist COM1 und 4800 baud. Es können insgesamt 4 Geräte angeschlossen werden (Gerät A .. D).
So kann eines z.B.\ als GPS-Quelle benutzt werden und ein anderes als externes variometer.
Wenn kein weiteres Gerät angeschlossen werden soll, so ist im Auswahlfeld des entsprechenden Anschlusses ''Deaktiviert'' auszuwählen.
\xc wird diese Anschlüsse dann ignorieren und keine Zeit mit der Suche nach einem Gerät an dem Anschluß verbringen.
\sketch{figures/config-devices.png}

Es dürfen COM 0 bis 10 verwendet werden,  einschließlich einer TCP/IP Verbinung, welche z.B. für eine
Rechnerverbindung (\textsc{CONDOR} o.ä. benutzt werden kann. Welcher COM-Port benutzt werden soll, hängt maßgeblich vom angeschlossenen Gerät ab,
welches diesen Port zur Verfügung stellt und wie es dies tut (BlueTooth, serielles Kabel, IoIo, SDCard, CF Card etc.) Auf alle diese detaillierten
Eingänge hier im Handbuch einzugehen, überstiege den Umfang des Handbuches, daher bitte auf der \xc-Hompage nachschauen, oder aber
in einer der mailing-listen  (z.B.:  \url{http:/www.forum.xcsoar.org} im Forum um Hilfe bitten.

\begin{description}
\item[Anschluß]  Hier werden die vier möglichen Anschlüsse a..D mit dem entsprechenden Gerät belegt. (Auf das feld gehen und anschließend entsprechenden Treiber aus Liste (s.u.) auswhälen).
\item[Baudrate] Hier wird die Baudrate eingestellt, mit der das Programm mit den extern angeschlossenen Geräten kommuniziert.
\item[Bündel Baudrate]  Hier wird die Baudrate eingestellt, mit der das Programm mit den extern angeschlossenen Geräten kommuniziert, wenn Flüge zu deklarieren oder herunterzuladen sind..
Falscheinstellungen führen definitiv zu Nichtfunktionieren. Eine der häufigsten \index{FLARM!Baudrate einstellen} \index{GPS Quelle!Baudrate einstellen}Fehlerquellen bei Anfängern.
\achtung Flarm muß  mindestens mit 9600 oder 19200 eingestellt werden, um auch den Luftverkehr darstellen zu können! Eine Einstellung des \fl auf 4800 bietet ausschließluich die GPS-Daten!
\item[TCP Port]  Dieser Port ist zu wählen, wenn  im Winter mit \xc trainiert werden soll, z.B.\ indem man  mit dem
Segelflugsimulator \textsc{Condor} spielt.
\item[Treiber] Eine Liste von Treibern, welche speziell auf die gelisteten Geräte angepaßt wurden, um deren Funktion in größtmöglichem Umfang zu benutzen.
Bislang unterstützt werden: \al, Borgelt B50/B800, Cambridge CAI GPS-NAV, Cambridge CAI302, Compass C-Probe, Condor Soaring Simulator, Digifly Leonardo,  EW Logger, EW microRecorder,
\fl, FlyNet Vario, Flymaster F1, Flytec 5030 Brauniger, GT Altimeter (GliderTools), ILEC Sn10, IMI Erixx, LX / Colibri, Posigraph Logger, Vega, Volkslogger, Westerboer VW1150, Westerboer VW921/922, XCOm760, Zander/SDI.
Weiterhin gibt es den Treiber GENERIC, welcher ein Universaltreiber ist, sowie die Auswahl NMEA, die für die Durchschleifung des Singnales gedacht ist, wenn zwei Geräte an einer GPS Quelle installiert sind(z.B.\ im Doppelsitzer.
\item[Abgleich vom Gerät$^{\textcolor{blue}{\star}}$]  {\bf Ein/Aus}: Mit dieser Option werden im externen Gerät vorgenomme Einstellungen, Berechnungen und Daten wie MacCready, Mücken, und Wind vom externen Gerät zur Verwendung in den internen Routinen empfangen.(Synchronisierung)\index{Synchronisierung externer Geräte mit \xc }
\item[Abgleich zum Gerät$^{\textcolor{blue}{\star}}$]  {\bf Ein/Aus}: Mit dieser Option werden in \xc eingestellte Daten wie MacCready, Mücken, und Wind zum externen Gerät verschickt. (Synchronisierung)
\item[DumpPort$^{\textcolor{blue}{\star}}$]  Enable this if you would like to log the communication with the device.
\item[Prüfsumme ignorieren$^{\textcolor{blue}{\star}}$] {\bf Ein / Aus}: Wenn das GPS-Gerät ungültige Prüfsummen der Übertragung meldet, werden mit
dieser Einstellung  die Daten dennoch benutzt.
\end{description}


%%%%%%%%%%%%%%%%%%
\section{Einstellung / Polare}\index{Polare!Laden}

Für eine Großzahl von Flugzeugen sind bereits Polaren vorhanden; auf dieser Seite kann eine
Polare erstellt und/oder verändert werden. Es kann auch eine externe Polaren-Datei verwendet werden.
Das Format der Daten entspricht dabei dem WinPilot-Format, siehe auch Kap.~\ref{sec:glide-polar}.
\label{conf:polar}

\begin{description}
\item[Polar V] Drei Felder mit den entsprechenden Werten für die Geschwindigkeit, die als Stützpunkte für die interne Errechnung der Polare dienen.
Eine gute Wahl ist dabei für die erste Geschwindigkeit die absolute Spitze der Polaren, die zweite Geschwindigkeit die, wo die Polare noch eine gute ''Krümmung'' aufweist, die dritte Geschwindigkeit wird als die ausgewählt, von der ab die Polare sogut wie ''gerade''erscheint.
\item[Polar W] Dies sind die zugehörigen Sinkwerte der oben genannten Geschwindigkeiten. Diese Wertepaare  müssen 100\% zueinanderpassen, ansonsten erlebt man böse Überraschungen.
%  A good choice for the point triplet is one at the top most area of the polar,
%  the second at a  still very curved area and the third far out where the
%  curvature seems to disappear.
\item[Referenzmasse] Das ist das Gewicht, für welches die Polare gültig ist.
\item[Normalgewicht] Abfluggewicht (also mit Pilot, Batterien etc.) des Flugzeuges ohne Wasserballast.
Da \xc aus Datenschutzgründen leider noch nicht die Gewichte aller Piloten vorliegen, muß dieser Wert individuell angepaßt werden.
\item[Flügelfläche]  Flügelfläche des Flugzeuges.
\item[Manövergeschwindigkeit] Hier kann optional die Manövergeschwindigkeit angegeben werden, um dem Rechner unrealistische Vorfluggeschwindigkeiten zu verbieten.
(Z.B.\ unterwegs mit Hr. Ohlmann in den Anden mit 8m/s in der Welle ...)
\item[Handicap]  Der aktuelle DAEC-Index des Flugzeuges.
\item[Max. Ballast] Maximal aufzunehmender Wasserballast welchen \xc als 100\% nimmt.
Wenn dies keine Rolle spielt,  Null setzen.
\item[Ballast Ablaßzeit]  Die Zeit, in der der Wasserballast komplett abgelassen werden kann.
\end{description}
\index{Polare!erstellen}
Am unteren Rande dieses Fensters befinden sich noch drei weitere Felder:\\
\button{Import},  \button{Export},  \button{Liste}.
\begin{description}
\item[Import] Hier kann eine Polare aus einem File geladen werden. Achte auf das Format!
\item[Export] Hier kann oben erstellte Polare in einem File gespeichert. Das Format ist automatisch richtig.
\item[Liste] Hier ist die interne Liste der Flugzeuge, zu denen eine Polare existiert.
\end{description}

\tip Es ist sehr empfehlenswert, eine eigene Polare zu exportieren, da dann alle Eigenschaften darin gespeichert und wieder aufgerufen werden können.
Die Werte für  die Polare sollten mit Bedacht und Sorgfalt eingegeben werden, da sämtliche Berechnungen von \xc darauf beruhen!
%%%%%%%%%%%%%%%%%%
\section{Einstellung / Logger} \label{conf:logger}

Der interne \xc-interne Logger besitzt separat anpaßbare Zeitintervalle  für Kurbeln  und Geradeausflug.
Typischerweise wird beim Kurbeln ein etwas geringerer Wert benutzt als beim Geradeausflug, um ''schönere'' Auswertungen zu bekommen, sehr sinnvoll aber auch beim Kurbeln ''hart an der Grenze'' von Lufträume\dots

\begin{description}
\item[Zeitintervall beim Vorflug$^{\textcolor{blue}{\star}}$] {\bf 0..30sec}: Zeitintervall der Logger-Fixe  beim Vorflug.
\item[Zeitintervall im Steigen$^{\textcolor{blue}{\star}}$]  {\bf 0..30sec}: Zeitintervall der Logger-Fixe  beim Kurbeln.
\item[Datei Kurzname$^{\textcolor{blue}{\star}}$] Die Aufzeichnung wird in eine Datei geschrieben. Der Name der Datei kann dabei im Lang- oder Kurzformat der IGC erfolgen. Das Datum wird dabei kodiert:\\
 {\bf Kurzer Dateiname}: \verb''81HXABC1.igc''\\
 {\bf Langer Dateiname}: \verb''2008-01-18-XXX-ABC1.igc''\\
\item[Auto Logger$^{\textcolor{blue}{\star}}$] Ermöglicht den automatischen Start  der Aufzeichnung durch den Logger beim Start und das das Stoppen bei Landung.
Wenn mit Gleitschirmen geflogen wird, sollte das auf ''Aus'' stehen, da  die Aufzeichnungen durch die langsamen Übergrundgeschwindigkeiten erfolgen.
   {\bf Ein}:  Start/Stop eingeschaltet\\
  {\bf Aus}: Start/Stop ausgeschaltet\\
  {\bf Nur Start}: Startet automatisch, muß manuell ausgeschaltet werden.
\item[NMEA Logger$^{\textcolor{blue}{\star}}$] Einstellung, ob der NMEA-Logger bei Programmstart gestartet werden soll. Falls nicht,
kann immer noch manuell gestartet werden.
  {\bf Ein}:  Startet automatisch NMEA-Aufzeichnugen\\
  {\bf Aus}: Aufzeichnung des NMEA-Loggers muß manuell eingeschaltet werden.\\
\item[Flugbuch$^{\textcolor{blue}{\star}}$]  {\bf Ein / Aus}: Aufzeichnung aller Starts und Landungen durch \xc.
\end{description}


%%%%%%%%%%%%%%%%%%
\section{Einstellung / Logger info} \label{conf:logger_info}\index{Logger!Info}\index{Logger!ID}\index{Logger!Flugzeugkennzeichen}\index{Logger!Wettbewerbskennzeichen}

Auf dieser Seite kann der \xc -IGC-Logger programmiert  und Daten wie Kennzeichen, Logger-ID, Wettbewerbskennzeichen, Pilotenname etc.\ eingegeben werden.
\begin{description}
\item[Pilotenname]  Eingabe des Pilotennamens für den \xc-Software-Logger.
\item[Flugzeugtyp]  Eingabe des Flugzeugtypen für den \xc-Software-Logger.
\item[Flugzeugkennz.]  Eingabe des Kennzeichens für den \xc-Software-Logger.
\item[Wettbewerbskennung] Eingabe des Wettbewerbskennzeichens für den Software-Logger.
\item[Logger ID]  Dies ist die Logger ID des verwendeten und angeschlossenen Loggers(\fl, VOLKSLOGGER etc.)
\end{description}
\todonum{stimmt das?}

%%%%%%%%%%%%%%%%%%
\section{Einstellung / Einheiten}\index{Einheiten!Einstellung}

Auf dieser Seite werden die angezeigten Einheiten von \xc eingestellt.
Für die meisten Benutzer sind bereits Voreinstellung vorhanden, welche in der Vorauswahl angewählt werden können.
{\bf American}, {\bf Australian}, {\bf British}, and {\bf European}.

Es ist jedoch möglich, die angegebenen Werte nach Belieben zu mischen und so z.B.\ in einem eigenen Profil zu speichern.

\begin{description}\itemsep=0.8\itemsep
\item[Vorauswahl] Hier ist eine Liste von vordefinierten Sätzen von Einheiten für diverse Länder/Kontinente enthalten:\\
{\bf Benutzerdefiniert}: Freie Auswahl und Anpassung der unten angegebene Einheiten nach Belieben\\
{\bf Europäisch}: Auswahl europäischer Einheiten   \\
{\bf Britisch}:   Auswahl britischer  Einheiten\\
{\bf Amerikanisch}:   Auswahl amerikanischer Einheiten\\
{\bf Australisch}:   Auswahl australischer Einheiten
\item[Luftfahrzeug-/Windgeschwindigkeit\textcolor{blue}{$\star$}]  Einheiten für Wind, und Luftfahrzeug: mph, knots, km/h.
    In Aufgaben kann eine separate Einheit verwendet werden. (s.u.)
\item[Entfernungen$^{\textcolor{blue}{\star}}$] Einheiten für horizontale Entfernungen wie z.B.\  Distanz zum Wegpunkt, zum Ziel
    etc.\ : sm, nm, km.
\item[Steigen/Sinken$^{\textcolor{blue}{\star}}$] Einheiten für Vertikalbewegungen (Vario): Knoten,  m/s, ft/min.
\item[Höhen$^{\textcolor{blue}{\star}}$] Einheiten für Höhen: Fuß, Meter.
\item[Temperaturen$^{\textcolor{blue}{\star}}$] Einheiten für Temperaturen: $\circ$ C, $\circ$ F.
\item[Aufgaben-Geschwindigkeiten$^{\textcolor{blue}{\star}}$] Einheiten für Geschwindigkeiten innerhalb Aufgaben: mph, knots,
    km/h.
\item[Druck$^{\textcolor{blue}{\star}}$]  Einheiten für den Druck: hPa, mb, inHg.
\item[Geographische Länge und Breite$^{\textcolor{blue}{\star}}$]  Einheiten für Koordinaten, besser die angewandten Formate
    der Koordinaten:  Es werden diverse Formate unterstützt:  ''Grad/Minuten/Sekunden'' Format und
    deren dezimale Darstellungen,  und das UTM WGS 84 - Format.
\end{description}


%%%%%%%%%%%%%%%%%%
\section{Einstellung / Time}

Einstellen der Uhr mit Referenz zur UTC-Zeit.

\begin{description}
\item[Zeitzonenverschiebnung]  {\bf -13 ..+13h}: In diesem Feld wird die Zeitzonenverscheibung zur UTC zeit eingebeben. Die Differenz kann im 30Minuten-Intervall eingegeben werden.
Zur Kontrolle wird im Feld darunter die entsprechende Lokalzeit angegeben -- um Mißverständnisse zu verhindern.
\item[Lokalzeit]  Das ist die oben errechnete Lokalzeit -- nur zur Kontrolle.
\item[Benutze GPS-Zeit$^{\textcolor{blue}{\star}}$] {\bf Ein /Aus}: Einstellung des Empfanges des Zeitsingales über das angeschlossene GPS\\
{\bf Ein}: Wenn eingeschaltet wird die Uhr des Gerätes auf dem \xc läuft \textsl{bei jedem Fix} durch das empfangene Zeitsignal des GPS neu gesetzt.
\achtung
Nur sinnvoll, wenn das Gerät nicht über eine Echtzeituhr verfügt, oder aber das Zeitsingale oft ausfällt oder extrem abweicht.
{\bf Aus}: Es wird die geräteinterne Uhr benutzt.
\end{description}

%%%%%%%%%%%%%%%%%%
\section{Einstellung / Verfolgung}

''{\it Live}''-Tracking steht für die GPS-basierte Verfolgung und Darstellung aktueller Standorte.
Die Daten buzw. Positionen werden in Echtzeit an einen Server übermittelt, der aus den Daten
eine Spur des Kurses erstellt und auf einer Karte darstellt.

Diese ''Tracking-Option'' benötigt eine  Verbindung zu einem Handynetz, um die Daten übermitteln zu können.
Zur Zeit sind zwei Protokolle implementiert. ''{\em SkyLines}''  als ein StartUp Service von \xc selber,
Details hierzu auf  \url{http://skylines.xcsoar.org}.


Weiterhin das ''{\em LiveTrack24}'', welches für das Tracking System auf \url{http://www.livetrack24.com} benutzt wird.
Details dazu befinden sich auf der entsprechenden Homepage.

\begin{description}
\item[SkyLines] {\bf Ein}: Ermöglicht die Life-Verfolgung Deiner Position über Live-stream per `{\em SkyLines}'.
\item[Tracking Interval]  Das Zeitintervall für die Positionsübermitlung an den Tracking-Service.
''{\em SkyLines}''  hat ein sehr schlankes Protokoll implmentiert, sodaß selbst  ein 30-Sekunden-Intervall eine GPRS-Verbindung (ca. 12kbit/sec) nicht überlastet.
\item[Key] Hier wird der Schlüssel eingegeben, welcher auf der Seite \url{http://skylines.xcsoar.org/tracking/info}
erstellt wurde, um Dich zu identifizieren.
\\
\item[LiveTrack24]  {\bf EIN}: Hiermit wird der liveStream von  ''{\em LiveTrack24}'' eingeschaltet.
\item[Verfolgungsintervall] Das Zeitintervall für diesen Tracking-Service.
\item[Luftfahrzeugtyp]  Der Typ des Luftfahrzeuges (Segelflugzeug, Motorflugzeug etc.)
\item[Server] Internetaddresse des TrackingServers. Zur Zeit fest eingestellt.
\item[Benutzername] Wenn ein Zugang erstellt wurde, hier den entsprechenden Benutzernamen eingeben.
Andernfalls wird die Flugspur als `{\it Gast}'  und somit anonym übermittelt.
\item[Passwort] Das entsprechende Passwort zum oben angegebenen  Zugang.
\end{description}
%finished für v6.5 date 09/02/2013 OH

%%%%%%%%%%%%%%%%%%%%%%
\input{ch12_data_files.tex}
%%%%%%%%%%%%%%%%%%%%%%
\chapter{Historie und Geschichte}\label{cha:history-development}
\section{Produkt Historie}

\xc startete als kommerzielles Produkt, welches von Mike Roberts (UK) entwickelt wurde. 
Das Programm hatte ein ein paar Jahre lang erfreulichen Erfolg und ging und durch mehrere Versionen.
Persönliche Gründe zwangen ihn, nach der  {\bf Version~2} die Weiterentwicklung einzustellen, sodaß
er sich entschied, den Code im späten Jahre 2004 der Öffentlichkeit zur Verfügung und  als
\xc {\bf Version~3} unter die  GNU Public License  zur Verfügung zu stellen.

Auf einer Yahoo Group wurde eine Webseite erstellt und es fanden sich einige Entwickler, die sich zusammentaten,
hier weiterzuentwickeln; damit war das Projekt geboren.

Im März 2005 war das Programm substantiell überarbeitet, was in der {\bf Version~4.0} endete.

Zu ungefähr dieser Zeit wurde die Koordination der einzelnen Mitentwickler immer schwieriger, der Programmkode gestaltete sich
schwieriger und zeitintensiver, so daß das Projekt auf SourceForge portiert wurde, wo mithilfe eines Versionsmanagement-Systemes
der Programmkode  verwaltet werden konnte.

Im Juli 2005 ist die  {\bf Version~4.2} herausgegeben worden, in welcher einige Probleme mit der Kompatibilität von PDAs bzgl.\
der GPS Hardwarekonfiguration behoben wurde.

Im September 2005 kam stand die {\bf Version~4.5} zur Verfügung. Hier wurden erstmal größere Änderungen bzgl.\ der Benutzeroberfläche
und -Eingaben vorgenommen; die Einführung  des internen ''Ereignis-Systemes'' und erste die Möglichkeit zur Bedienung der 
Oberfläche in anderen Sprachen   wurden vorgestellt.s

Im April 2006 ist die {\bf Version~4.7} für den \al vorgestellt worden - Stabilität und Performance standen hier an erster Stelle.
Viele Fehler wurden beseitigt, eine neue Methode für die Behandlung von Menüein- und ausgeben wurde erstellt,
welche aus XML-Dateien basierte.

Im September 2006 ist die  {\bf Version~5.0} für alle Plattformen (\al, PC, PDA) erschienen.
Diese Version zeichnete sich durch etliche Neuerungen aus und ist als erste massiv im Fluge und in 
Simulationen getestet  worden.

Die {\bf Version~5.1.2} kam im September 2007 für alle Plattformen (\xc, PC, PDA) heraus.
Auch hier sind etliche Neuerung vor der Herausgabe  durch intensive Test im Fluge und in Simulationen geprüft worden.
Die wichtigsten Neuerungen waren: 

Einführung des JPEG2000-Formates für die Karten, Einführung des \fl-Radars, Einführung der
OLC-Unterstützung, bessere Stabilität, Genauigkeit der Aufgenberechnung, bessere Bedienbarkeit.
Das erste mal haben hier wirklich viele Nutzer Eingaben und Vorschläge zur Verbesserung gemacht,
welche  zum Großteil berücksichtigt werden konnten.

Februar 2008:  {\bf Version~5.1.6} vorgestellt. Viele Fehler konnten beseitigt werden. Deutliche Verbesserung und
Erweiterung Einführung von AAT-Aufgaben sowie der RASP-Wettervorhersage.

März 2009, {\bf Version~5.2.2} ausgeliefert. Verbesserungen in der Bedienung und einige maßgeblich Verbesserungen sind
vorgenommen worden, unter anderem:

IGC-Files werden für die Validierung digital signiert z.B.\ für den OLC. Das erste Mal konnte das Programm auf Windows-CE 
und PNA-basierte Geräte portiert werden.

\fl ist in die Kartendarstellung integriert worden und es wird von nun an die \fl-Net Datenbank unterstützt.
Entwickler konnten von nun an \xc auch vom Linux-PC einfach kompilieren

August  2009: {\bf Version~5.2.4} interne Fehlerbehebung und Erweiterungen.

Im Dezember 2010 ist die {\bf Version~6.0} herausgegeben worden.
Nach einer intensiven, kompletten  Überarbeitung von großen Teilen des Programmkodes konnte viele Stabilitäts- und Performance
Probleme gelöst werden. Die Startzeit konnte dramatisch reduziert werden {(\small das kann ich als User nur bestätigen\dots Anmerkung OH)}

Viele neue Erweiterungen und Funktionen wurden eingeführt, inklusive eines neuen TaskManagers (Aufgabenverwaltung), 
erweiterter AAT-Fähigkeiten, einer neuen \fl-Unterstützung und einer Zentrierhilfe. 

Viele neue Sprachen standen zur Verfügung, und neue Übersetzungen konnten von nun an auch von
den Benutzern einfach vorgenommen werden.

Diese Umcodierung erlaubte \xc nun nativ auf Unix/Linux Systeme wie z.B.\ auch Android Geräten zu laufen
und das bei gleichzeitiger Performance-Steigerung durch moderne Compilerund aktuelle Hardware (z.B. Handys oder Tablets.)

März 2011: {\bf Version~6.0.7b} ist rausgekommen, die erste Version, die offiziell  Android-Geräte untersützt.

Januar 2013{\bf Version 6.4.5.} Das erste komplette Handbuch auf Deutsch ist erschienen. (Danke, Hotte)

\section{Mach mit}

Der Erfolg diese Projektes ist das Resultat vieler einzelner Beitrage. Du mußt kein Software-Entwickler sein, um hier teilnehmen zu können.
Testberichte, Übersetzungen und Hinweise zu evtl.\ abnormalem Verhalten des Programme sind ebenso wichtig und gerne willkommen
im Projekt.

Generell gibt es fünf weiter Möglichkeiten, Projekt beizutragen, ohne Entwickler bzw. Programmierer zu sein:

\begin{description}
\item[1. Gib uns Rückmeldungen] Ideen, Vorschläge, Fehlerberichte und konstruktive Kritik sind gerne willkommen und werden
wo immer möglich berücksichtig.
\item[2. Hinweise zur Konfiguration] Da \xc derart flexibel in der Konfiguration ist, brauchen wir immer Vorschläge von anderen, wie
das Programm am besten zu bedienen ist. Auch wir sind --wie jeder-- nicht immer gefeit vor einer gewissen ''Betriebsblindheit'' .
Vorschläge zur Bedienung von InfoBoxen, deren Layout, Anordnung, Farbe - alles was hilfreich ist um das Programm so sicher
und einfach wie möglich zu bedienen, bedürfen einiger Überlegung vorab, um diese ''Designstudien'' anschließend den Entwicklern
vorzustellen, welche es letztlich realisieren und als Standard im Programm implementieren.
\item[3. Integrität externer Daten] Luftraumdateien und Wegpunktdateien sollten immer auf dem laufenden sein.
Hierzu sind oftmals Personen gefragt, die Kenntnisse der Lokalen Gegebenheiten haben (exakte Koordinaten, evtl. aktuelle Photos von Flug- und Landeplätzen).
\item[4. Werbung]  Je mehr Personen die Software benutzen, desto besser wird das Programm werden. Je mehr Leute uns Rückmeldungen geben, umso
besser kann das Program auf die allgemeinen Vorlieben angepasst werden und umso schneller werden evtl.\ auch versteckte Fehler erkannt.

Du kannst helfen, indem Du die ''Werbetrommel'' schlägst, bspw.\ indem Du in Deinem Verein einen kleinen Vortrag hältst, oder aber im Vereinsraum mit
dem Beamer Deinen letzten Flug in \xc vorstellst oder nachfliegen läßt, oder indem Du einfach einem Kumpel das Programm in drei Minuten auf 
dem Handy zum Spaß oder zur Probe installierst.
\item[5. Documentation]  Normalerweise und vollkommen klar - das Handbuch ist grundsätzlich nicht aktuell. Du kannst helfen, hier zu übersetzen oder zu 
schreiben. Oder einfach ein paar übersichtliche Kärtchen evtl.\ mit den wichtigsten ''Handgriffen'' erstellen, die Du an Deine Vereinskameraden 
verteilst.
\end{description}

\section{Philisophie des offenen Quellcodes (Open Source)}

Es gibt einige Vorteile, ein Programm wie \xc als Open Source Projekt anzulegen:

\begin{itemize}
\item Zuallererst: es handelt sich um kostenlose, freie Software, das heißt, jeder (nicht nur piloten) Pilot kann es sich installieren und probieren und testen,
ob es das ist, was er braucht. Jeder ist frei, sich das Programm auf den PC, MAC, Pinguin, Android, PDA, PNA, EFIS  oder was auch immer zu installieren
und jeder kann es nach Belieben weitergeben.

\item Du hast freien Zugang zum Programmquellcode, sodaß Du ihn in jedem anderen freien OpenSource Programm verwenden darfst.

\item Programmcode im Internet offen zu haben, bietet die Gewähr, daß viele Leute einen forschenden Blick auf den Code werfen können und Fehler schnell offenbar 
werden um so schnell und beseitigt werden zu können.

\item  Eine große Gruppe von Entwicklern steht zur Verfügung um Hilfe bei Fehlern zu geben schnell neue Funktionen auszuarbeiten

\item Open Source - Software, welche unter die GNU Public License fällt, kann nicht zu einem späteren Zeitpunkt zu ''Löhnware'' 
gemacht werden. Die Benutzung derartiger Software wird daher auf keinen Fall dazu führen, daß Du später in die Kostenfalle tappst.
Diese Software ist und bleibt frei, gratis, kostenlos -- aber nicht umsonst\dots.

\end{itemize}

Die ausführlichen, kompletten Bedingungen zur Lizensierung von \xc sind in Anhang~\ref{cha:gnu-general-public} in deutsch und in englisch abgedruckt.

Die Entwicklung von \xc ist, seit es OpenSource ist ins Netz gestellt wurde, komplett eine rein freiwillige Arbeit von ausschließlich Freiwilligen.
Das bedeutet nicht, daß evtl.\  einige individuelle Entwickler oder Organisationen  kommerzielle Unterstützung anbieten.
Der Geist diese Projektes jedoch läßt vermuten, daß in diesen fällen auch diese  ''Mitstreiter'' sicher sein werden, der Gemeinschaft
mit Ihrem Produkt oder Dienstleistung einen guten Dienst geleistet zu haben.

\section{Entwicklungpsozeß}

Wir versuchen, neue Funktionen so schnell wie möglich zu implementieren. Es muß hierbei natürlich abgewogen werden, das komplette Konstrukt
aus Bediener und interner Funktionalität nicht derart schroff geändert werden kann, daß ein Nutzer, welcher z.B.\ ein Update installiert sich nicht
mehr zurechtfindet und sich im schlimmsten Falle von \xc abwendet.

Das heißt, es war, als wir z.B.\ die neue Menüoberfläche in V4.5 eingeführt haben,  notwendig, mit der neuen Version  auch ein externes
File mit verteilen mußten, indem die althergebrachten Funktionen in einer Art Übersetzungstabelle enthalten waren.

\xc kann, wenn es während des Fluges benutzt wird, als ''kritische'' Software  (mission-critical) angesehen werden, da es sich um ein Echtzeit-System handelt.
Es wird daher sehr großer Wert darauf gelegt, daß diese Software ausgiebig von den Entwicklern und  freiwilligen Piloten im Fluge und in der ''harten ''Realität'' geprüft wird,
bevor eine neue Version an die Allgemeinheit herausgeht.

Die Test im Fluge und die Erfahrungsberichte sind ganz sicher die besten  Prüfungen für das Programm, aber es gibt auch eine ganze Reihe 
von Berichten von Testern, die mit \xc im Auto herumgefahren sind und bestimmt Szenarien durchgespielt haben.  Eine ganze Menge von 
Testberichten konnten durch Zusendung von  IGC-Files ausgewertet werden, wo sekundengenau das Verhalten von \xc nachgespielt werden konnte.

Natürlich tun wir alles, um einen Programmabsturz oder -hänger vermeiden, wenn dies jedoch während der Testphase geschieht hat die Beseitigung eines solchen 
Bugs oberste Priorität vor der evtl.\  Neueinführung von neuen Funktionen.
Stabilität des Programmes ist oberstes Gebot.


Es gibt daher für die immer weiter fortschreitende und sich extrem stark durchsetzende Android-Version eine Crash-Report-Verfolgung,
auf welche die Crash Reports der Androids, welche z.B.\  unter
\begin{tabbing}
{\small\texttt{/mnt/SDCard/XCSoarData/crash/..}}\\
\qquad\qquad{\small\texttt{../crash-2013-01-05-10-24-09-3998.txt}}
\end{tabbing}
oder
\begin{tabbing}
{\small\texttt{/mnt/extSDCard/XCSoarData/crash/..}}\\
\qquad\qquad{\small\texttt{../crash-2013-01-05-10-24-09-3998.txt}}
\end{tabbing}
abgelegt werden - je nachdem wie das File heißt und wo sich das XCSoarData-Verzeichnis befindet--
zur Begutachtung hochgeladen werden können. (Der Filename enthält das Datum sowie eine codierte Nummer)


Die Software Entwickler halten alle Kontakt miteinander über dies SourceForge-Entwickler-mailing- Liste:
\begin{quote}
\url{xcsoar-devel@lists.sourceforge.net}
\end{quote}
Wir versuchen, die Aktivitäten und Arbeiten so zu koordinieren, daß doppelte Arbeit und Konflikte vermieden 
und eine echte Teamarbeit zustande kommt.
Wenn Du mitmachen willst, sende einfach eine mail an einen der Software-Entwickler aus der Liste s.u.

\section{Die Basis der Benutzer}

Wer benutzt \xc? Gute Frage, schwer zu beantworten. 

Da niemand für diese Software bezahlt, und die meisten Leute das Programm 
anonym herunterladen oder kopieren, ist es schwer nachzuvollziehen,  wie viele Leute es auch tatsächlich nutzen.

Die Statistiken der c-Homepage zeigen für die Jahre von Juni 2005 bis Juni 2006 ca. 20 Downloads pro Tag an, und für 2006 bis 2007
80 Downloads pro Tag.

Wenn man sich anschaut, wie viele Leute die verschiedenen Karten und Topologie  herunterladen, kann man erkennen, daß
\xc in vielen Ländern auf nahezu jedem Kontinent benutzt wird.

\xc wird von vielen Benutzergruppen verwendet, es gibt Wettbewerbspiloten, genauso wie ''Couchflieger'',
welche es im Zusammenspiel mit Segelflugsimulatoren wie z.B.\ Condor verwenden, oder aber Anfänger und ''Lust''-Flieger.

In immer mehr Vereinen wird \xc eingesetzt, da es eine echte Multiplikatorfunktion für die Nachwuchsflieger besitzt  und vor allem
in allen Flugzeugen für geringste Kosten als vollwertige Navigationshilfe eingesetzt werden kann:


Ein alter Ipaq 38xx liegt bei Ebay bei ca. 15-40\euro{}. Dazu ein bißchen Kabel, ein Spannungswandler für 3,5\euro{} und eine Halterung 
und mit deutlich unter 100\euro{} ist man dabei.  
Einige Vereine bieten \xc - Einführungstage an, um den kompletten ''Trupp'' auf dem selben Level zu halten.

Hier zahlt sich auf beste Weise aus, daß \xc eben auch auf dem PC läuft und so problemlos auf einem Beamer Im Vereinsheim 
ausgestrahlt werden kann.

\section{Dank an}\label{sec:credits}

{\large\bf Software Entwickler:}
\begin{compactitem}
\item Santiago Berca \url{santiberca@yahoo.com.ar}
\item Tobias Bieniek \url{tobias.bieniek@gmx.de}
\item Robin Birch \url{robinb@ruffnready.co.uk}
\item Damiano Bortolato \url{damiano@damib.net}
\item Rob Dunning \url{rob@raspberryridgesheepfarm.com}
\item Samuel Gisiger \url{samuel.gisiger@triadis.ch}
\item Jeff Goodenough \url{jeff@enborne.f2s.com}
\item Lars H \url{lars_hn@hotmail.com}
\item Alastair Harrison \url{aharrison@magic.force9.co.uk}
\item Olaf Hartmann \url{olaf.hartmann@s1998.tu-chemnitz.de}
\item Mirek Jezek \url{mjezek@ipplc.cz}
\item Max Kellermann \url{max@duempel.org}
\item Russell King \url{rmk@arm.linux.org.uk}
\item Gabor Liptak \url{liptakgabor@freemail.hu}
\item Tobias Lohner \url{tobias@lohner-net.de}
\item Christophe Mutricy \url{xtophe@chewa.net}
\item Scott Penrose \url{scottp@dd.com.au}
\item Andreas Pfaller \url{pfaller@gmail.com}
\item Mateusz Pusz \url{mateusz.pusz@gmail.com}
\item Florian Reuter \url{flo.reuter@web.de}
\item Mike Roberts
\item Matthew Scutter \url{yellowplantain@gmail.com}
\item Winfried Simon \url{winfried.simon@googlemail.com}
\item Google Inc., incl. Tom Stepleton \url{stepleton@google.com}
\item Simon Taylor \url{simon.taylor.uk@gmail.com}
\item Matthew Turnbull \url{matthewt@talk21.com}
\item Paolo Ventafridda \url{coolwind@email.it}
\item James Ward \url{jamesward22@gmail.com}
\item John Wharington \url{jwharington@gmail.com}
\end{compactitem}

\vspace{1em}
{\large\bf Dokumentation:}
\begin{compactitem}
\item Daniel Audier \url{osteocool@yahoo.fr}
\item Monika Brinkert \url{moni@sunpig.de}
\item Kevin Ford \url{ford@math.uiuc.edu}
\item Claus-W. Häbel \url{c-wh@online.de}
\item Stefan Murry \url{smurry@ao-inc.com}
\item Helmut J. Rohs \url{helmut.j.rohs@web.de} 
\end{compactitem}

\vspace{1em}
{\large\bf Übersetzer:}
\begin{compactitem}
\item Milan Havlik
\item$\star$  Zdenek Sebesta
\item Tobias Bieniek \url{tobias.bieniek@gmx.de}
\item Niklas Fischer \url{nf@nordthermik.de}
\item Peter Hanhart \url{peter.hanhart@schoensleben.ch}
\item Max Kellermann \url{max@duempel.org}
\item Helmut J. Rohs \url{helmut.j.rohs@web.de}
\item Philipp Wollschlegel \url{folken@kabelsalat.ch}
\item$\star$  Thomas Manousis
\item Miguel Valdiri Badillo \url{catastro1@tutopia.com}
\item Alexander Caldwell \url{alcald3000@yahoo.com}
\item Diego Guerrero \url{iccarod@hotmail.com}
\item$\star$  Hector Martin
\item Andres Miramontes \url{amiramon@gmail.com}
\item$\star$  Romaric Boucher
\item Sylvain Burger \url{sylvain.burger@wanadoo.fr}
\item$\star$  Dany Demarck
\item$\star$  Zoran Milicic
\item$\star$  Sasa Mihajlovic
\item Gabor Liptak \url{liptakgabor@freemail.hu}
\item$\star$  Kalman Rozsahegyi
\item$\star$  Enrico Girardi
\item$\star$  Lucas Marchesini
\item$\star$  Rick Boerma
\item Joop Gooden \url{joop.gooden@nlr.nl}
\item Hans van 't Spijker
\item Michal Jezierski \url{m.jezierski@finke.pl}
\item$\star$  Mateusz Pusz
\item Luke Szczepaniak \url{luke@silentflight.ca}
\item Mateusz Zakrzewski
\item$\star$  Tales Maschio
\item Luis Fernando Rigato Vasconcellos \url{fernando.rigato@gmail.com}
\item Monika Brinkert \url{moni@sunpig.de}
\item Nikolay Dikiy
\item Brtko Peter \url{p.brtko@facc.co.at}
\item Roman Stoklasa \url{rstoki@gmail.com}
\item$\star$  Aleksandar Cirkovic
\item$\star$  Patrick Pagden
\item 'zeugma'
\item Morten Jensen
\item Kostas Hellas \url{kostas.hellas@gmail.com}
\item Alexander Caldwell \url{alcald3000@yahoo.com}
\item Xavi Domingo \url{xavi@santmodest.net}
\item Arnaud Talon
\item Adrien Ott \url{adrien.ott@gmail.com}
\item Matthieu Gaulon
\item Filip Novkoski \url{f1novkoski@gmail.com}
\item Szombathelyi Zolt\'an \url{szombathelyi.zoltan@main.hu}
\item \'Ur Bal\'azs \url{urbalazs@gmail.com}
\item Paolo Pelloni \url{paolo@paolopelloni.it}
\item Piero Missa \url{pieromissa@virgilio.it}
\item Masahiro Mori \url{mron@n08.itscom.net}
\item Jinichi Nakazawa \url{jin-nakazawa@wkk.co.jp}
\item Mike Myungha Kuh
\item Rob Hazes
\item Thomas Amland \url{thomas.amland@gmail.com}
\item Quint Segers
\item Wil Crielaars \url{kawa1998@home.nl}
\item Krzysztof Kajda
\item Micha\l{}   Tworek % Todo, l with dash in the middle
\item Tiago Silva
\item Mario Souza
\item J\'ulio Cezar Santos Pires \url{juliocspires@gmail.com}
\item Wladimir Kummer de Paula
\item Pop Paul \url{poppali1@yahoo.com}
\item Dobrovolsky Ilya \url{ilya_42@inbox.ru}
\item Mats Larsson \url{mats.a.larsson@gmail.com}
%\item ???????? ?????
\end{compactitem}

$\star$:  Beiträge zum  LK8000 Projekt (\url{http://www.lk8000.it/}) hinzugefügt

\vspace{1em}
{\large\bf Kode und Algorythmen wurden weiterhin integriert von:}
\begin{description}
\item[Ephemeris] Jarmo Lammi
\item[Shapelib] Frank Warmerdam \url{http://shapelib.maptools.org}
\item[Least squares] Curtis Olson \url{http://www.flightgear.org/~curt}
\item[Aviation Formulary] Ed Williams \url{http://williams.best.vwh.net/avform.htm}
\item[JasPer] Michael D. Adams \url{http://www.ece.uvic.ca/~mdadams/jasper/}
\item[Volkslogger support] Garrecht Ingenieurgesellschaft
\item[Circling wind analyser] Andr\'e Somers \url{http://www.kflog.org/cumulus/}.
\end{description}
             
%%%%%%%%%%%%%%%%%%%%%%
\printindex
%%%%%%%%%%%%%%%%%%%%%%
\appendix
%%%%%%%%%%%%%%%%%%%%%%
\chapter{GNU General Public License}\label{cha:gnu-general-public}
\input{gpl.tex}
%%%%%%%%%%%%%%%%%%%%%%
\listoftodos
%\todonum[inline]{alternates list}
%%%%%%%%%%%%%%%%%%%%%%
\end{document}
